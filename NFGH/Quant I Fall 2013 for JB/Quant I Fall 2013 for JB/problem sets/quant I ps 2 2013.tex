
\documentclass[11pt]{article}
%%%%%%%%%%%%%%%%%%%%%%%%%%%%%%%%%%%%%%%%%%%%%%%%%%%%%%%%%%%%%%%%%%%%%%%%%%%%%%%%%%%%%%%%%%%%%%%%%%%%%%%%%%%%%%%%%%%%%%%%%%%%%%%%%%%%%%%%%%%%%%%%%%%%%%%%%%%%%%%%%%%%%%%%%%%%%%%%%%%%%%%%%%%%%%%%%%%%%%%%%%%%%%%%%%%%%%%%%%%%%%%%%%%%%%%%%%%%%%%%%%%%%%%%%%%%
\usepackage[abbr]{harvard}
\usepackage{amssymb}
\usepackage{setspace,graphics,epsfig,amsmath,rotating,amsfonts,mathpazo}

\setcounter{MaxMatrixCols}{10}
%TCIDATA{OutputFilter=LATEX.DLL}
%TCIDATA{Version=5.50.0.2960}
%TCIDATA{<META NAME="SaveForMode" CONTENT="1">}
%TCIDATA{BibliographyScheme=BibTeX}
%TCIDATA{Created=Thursday, September 11, 2008 15:11:56}
%TCIDATA{LastRevised=Wednesday, October 02, 2013 12:51:09}
%TCIDATA{<META NAME="GraphicsSave" CONTENT="32">}
%TCIDATA{<META NAME="DocumentShell" CONTENT="Articles\SW\article.egan">}

\topmargin=0 in \headheight=0in \headsep=0in \topskip=0in \textheight=9in \oddsidemargin=0in \evensidemargin=0in \textwidth=6.5in
\input{tcilatex}
\begin{document}


New York University

Wilf Family Department of Politics

Fall 2013

\begin{center}
{\large \textbf{Quantitative Research in Political\ Science I}}

Professor Patrick Egan

\bigskip

\textbf{PROBLEM\ SET 2: Due in class on Monday October 7 at 10 a.m.}
\end{center}

\textit{A reminder: you may work with others in the class on this problem
set, and you are in fact encouraged to do so. \ However, the work you hand
in must be your own. \ Handwritten work is acceptable, but word-processed
work (e.g., using LaTeX) is preferred.}

\begin{enumerate}
\item WMS Exercise 3.22.

\item WMS Exercise 3.30.

\item WMS Exercise 3.33.

\item WMS Exercise 3.48.

\item WMS Exercises 3.64 and 3.65.

\item WMS Exercises 3.136 and 3.137.

\item WMS Exercise 4.53.

\item WMS\ Exercises 4.58 and 4.59. \ Rather than the statistical tables or
the WMS \textquotedblleft applet,\textquotedblright\ use the Stata functions 
\texttt{normal(z) }and\texttt{\ invnormal(p)} to solve these exercises. \
(If you need it, help for these functions is easily obtained through Stata's 
\texttt{help }command.) \medskip 
\end{enumerate}

\begin{center}
[THERE\ IS\ ONE\ MORE\ PROBLEM\ ON\ THE\ NEXT\ PAGE.]\newpage
\end{center}

\begin{enumerate}
\item[9.] Consider the following standard setup in formal models of
electoral competition, in which the utility voter $v$ derives from electing
candidate $a$ is written 
\begin{eqnarray*}
U_{v}\left( x_{a}\right)  &=&-\left( x_{v}-x_{a}\right) ^{2},\text{ where} \\
&&x_{v}\text{ is the voter's ideal policy on the real line, and} \\
&&x_{a}\text{ is the policy (also on the real line) candidate }a\text{ will
enact if elected.}
\end{eqnarray*}

(A concrete way to think about this, for example, is to consider $x_{a}$ and 
$x_{v}$ to be two different tax rates.)

\begin{enumerate}
\item \lbrack Easy; not a trick question.] \ Say that $a$ makes a binding
proposal during an election campaign to enact $x_{a}$ if elected. \ What
proposal (and therefore what policy) maximizes $v$'s utility?\medskip

Now consider the case where the voter is unsure about what $a$ will do if
elected. A reasonable way to model this scenario would be to consider $x_{a}$
a random variable with mean $\mu _{a}$ and variance $\sigma _{a}^{2}.$ \ In
this case, rather than evaluating $U_{v}\left( x_{a}\right) ,$ the voter
evaluates her expected utility, or $E\left[ U_{v}\left( x_{a}\right) \right]
.$\medskip

\item Supply an expression for $E\left[ U_{v}\left( x_{a}\right) \right] $
written only in terms of $x_{v},$ $\mu _{a},$ and $\sigma _{a}^{2}.$\medskip

\item What is $\frac{\partial E\left[ U_{v}\left( x_{a}\right) \right] }{%
\partial \sigma _{a}^{2}}$?\medskip

\item Have we made any assumptions about the distribution of $x_{a}$? \
\medskip 

\item As specified above, $U_{v}\left( x_{a}\right) $ is an example of what
is known as a \textquotedblleft concave utility function.\textquotedblright\
\ Based on your analysis here, why is it appropriate that agents with
concave utility functions are said to be \textquotedblleft risk
averse?\textquotedblright \medskip 

\item If we assume that voters are risk averse, do candidates have an
incentive to be vague in a campaign about the policies they'll enact if
elected? \ (HINT: Your response should explicitly refer to how $\sigma
_{a}^{2}$ affects $E\left[ U_{v}\left( x_{a}\right) \right] $.) \medskip
\end{enumerate}
\end{enumerate}

\end{document}


\documentclass[11pt]{article}
%%%%%%%%%%%%%%%%%%%%%%%%%%%%%%%%%%%%%%%%%%%%%%%%%%%%%%%%%%%%%%%%%%%%%%%%%%%%%%%%%%%%%%%%%%%%%%%%%%%%%%%%%%%%%%%%%%%%%%%%%%%%%%%%%%%%%%%%%%%%%%%%%%%%%%%%%%%%%%%%%%%%%%%%%%%%%%%%%%%%%%%%%%%%%%%%%%%%%%%%%%%%%%%%%%%%%%%%%%%%%%%%%%%%%%%%%%%%%%%%%%%%%%%%%%%%
\usepackage[abbr]{harvard}
\usepackage{amssymb}
\usepackage{setspace,graphics,epsfig,amsmath,rotating,amsfonts,mathpazo}

\setcounter{MaxMatrixCols}{10}
%TCIDATA{OutputFilter=LATEX.DLL}
%TCIDATA{Version=5.50.0.2960}
%TCIDATA{<META NAME="SaveForMode" CONTENT="1">}
%TCIDATA{BibliographyScheme=BibTeX}
%TCIDATA{Created=Thursday, September 11, 2008 15:11:56}
%TCIDATA{LastRevised=Wednesday, November 13, 2013 17:05:56}
%TCIDATA{<META NAME="GraphicsSave" CONTENT="32">}
%TCIDATA{<META NAME="DocumentShell" CONTENT="Articles\SW\article.egan">}

\topmargin=0 in \headheight=0in \headsep=0in \topskip=0in \textheight=9in \oddsidemargin=0in \evensidemargin=0in \textwidth=6.5in
\input{tcilatex}
\begin{document}


New York University

Wilf Family Department of Politics

Fall 2013

\begin{center}
{\large \textbf{Quantitative Research in Political\ Science I}}

Professor Patrick Egan

\bigskip

\textbf{PROBLEM\ SET 7: Due Monday, November 18 at beginning of class.}
\end{center}

\textit{A reminder: you may work with others in the class on this problem
set, and you are in fact encouraged to do so. \ However, the work you hand
in must be your own. \ Your work must be word-processed in order for you to
receive credit for the assignment.}

\bigskip

For this assigment, you are to use the \texttt{cpsnov2012.dta}\textit{\ }%
dataset found on our class website. \ This is the November 2012 Current
Population Survey, conducted by the U.S.\ Census Bureau with a nationally
representative sample of U.S. households. \ The codebook for the dataset (%
\texttt{cpsnov2012.pdf}) may also be found on our class website.\bigskip\ \ 

\textit{Note: }Most of these questions require that you analyze the CPS's
household income variable (\texttt{hefaminc}). \ It is coded at the ordinal
level, but the analyses require that it be at an interval level. \ To do
this, create a recoded version\ of \texttt{hefaminc }in which each case is
assigned a household income equal to the midpoint of its interval of \texttt{%
hefaminc. }For example, a household with income in the range of
\$5,000-\$7,500 should be assigned the value \$6,250, and so on.\bigskip 

\begin{enumerate}
\item The following scatterplot was produced by an analyst interested in
exploring the relationship between turnout (measured with variable \texttt{%
pes1}) and household income (\texttt{hefaminc}). \ The Stata command
generating the scatterplot was \texttt{scatter hefaminc pes1 .\FRAME{dtbpF}{%
5.22in}{3.7913in}{0pt}{}{}{ps7.tif}{\special{language "Scientific Word";type
"GRAPHIC";maintain-aspect-ratio TRUE;display "USEDEF";valid_file "F";width
5.22in;height 3.7913in;depth 0pt;original-width 13.1996in;original-height
9.5778in;cropleft "0";croptop "1";cropright "1";cropbottom "0";filename
'//POLFS1/pje202/t.Quant I Fall 2013/problem sets/ps7.tif';file-properties
"XNPEU";}}}

\bigskip

\begin{enumerate}
\item There are many, many things wrong with this figure with regard to both
accuracy and style. \ Name as many as you can.\bigskip

\item Construct a well-designed figure that best displays the relationship
between household income and turnout. \ This will require recoding variables
and thinking carefully about the levels at which both variables are
measured. \ Provide the Stata (or, if you prefer to use it, \textit{R})
commands you used to recode variables and construct the figure. In a brief
paragraph, explain why you made the choices that you did.\bigskip

\item In a few sentences, describe the relationship you see between income
and turnout.\bigskip
\end{enumerate}

\item Which relationship--that between income and turnout, or education and
turnout--best approximates a linear relationship? \ The variable to use for
educational attainment is \texttt{peeduca}. Note that it, like \texttt{%
hefaminc}, is coded at the ordinal level but you want to analyze it as an
interval-level variable. \ \bigskip

\item You want to investigate the relationship between country of birth and
current income. \ 

\begin{enumerate}
\item You wish to divide the sample into three groups: (1) those not born in
the U.S.; (2) those born in the U.S. but who have at least one parent not
born in the U.S.; and (3) those born in the U.S. with both parents born in
the U.S. \ Using the variables \texttt{hefaminc, penatvty}, \texttt{pemntvty}%
, and \texttt{pefntvty}, construct a boxplot which displays the distribution
of household income for each of these three groups. \ \textit{Hint}: doing
this will require creating new variables from \texttt{penatvty}, \texttt{%
pemntvty}, and \texttt{pefntvty.} \ You will also need to make choices about
how to deal with missing values. \ Justify any choices you make in a note
accompanying the figure.\bigskip

\item Using the proper statistical tests with an $\alpha =.05$, answer the
following questions. \ Note that additional recoding may be necessary.

\begin{enumerate}
\item Do native-born Americans have higher incomes than non-native born
Americans?

\item Do native-born Americans whose parents were born in the U.S. have
higher incomes than native-born Americans with at least one parent \textit{%
not }born in the U.S.?

\item Do Americans with a foreign-born father and a native-born mother have
lower incomes than Americans with a foreign-born mother and a native-born
father?\bigskip
\end{enumerate}

\item In a few sentences, describe your results. \ \texttt{\ \ }
\end{enumerate}

\item Take some time to acquaint yourself with the dataset. \ It will be
helpful to refer to the codebook as you do. \ Now pick three variables from
the dataset to illustrate omitted variable bias as we discussed in class. \
That is, illustrate how if the true model is%
\begin{equation*}
y=\beta _{0}+\beta _{1}x+\beta _{2}z+u
\end{equation*}%
but you estimate%
\begin{equation*}
y=\beta _{0}+\beta _{1}x+v,
\end{equation*}%
you obtain a biased estimate of $\beta _{1}.$[BEFORE\ PROCEEDING, SEE\ NEXT\
PAGE.]

\begin{enumerate}
\item First, show how omitting a variable $z$ can bias estimates of $\beta
_{1}$ in a \textit{positive} direction:

\begin{enumerate}
\item Pick three variables from the dataset: a dependent variable ($y$), an
independent variable ($x$) and a potential confound ($z$).

\item Theoretically justify your designations of $x$ and $y$ as independent
and dependent variables, and $z$ as a potential confounder.

\item Recode the variables if necessary.

\item Generate a correlation matrix with your three variables. \ Discuss how
the matrix indicates that $\widehat{\beta }_{1}$will be biased upward if you
estimate the model $y=\beta _{0}+\beta _{1}x+v$ when the true model is $%
y=\beta _{0}+\beta _{1}x+\beta _{2}z+u$.

\item Estimate the models $y=\beta _{0}+\beta _{1}x+v$ and $y=\beta
_{0}+\beta _{1}x+\beta _{2}z+u.$ \ Discuss how the results confirm your
expectations.
\end{enumerate}

\item Now repeat this process. \ But this time choose $x$, $y$ and $z$ that
illustrate how omitting a variable $z$ can bias estimates of $\beta _{1}$ in
a \textit{negative }direction.
\end{enumerate}
\end{enumerate}

\end{document}

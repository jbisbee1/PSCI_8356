
\documentclass[11pt]{article}
%%%%%%%%%%%%%%%%%%%%%%%%%%%%%%%%%%%%%%%%%%%%%%%%%%%%%%%%%%%%%%%%%%%%%%%%%%%%%%%%%%%%%%%%%%%%%%%%%%%%%%%%%%%%%%%%%%%%%%%%%%%%%%%%%%%%%%%%%%%%%%%%%%%%%%%%%%%%%%%%%%%%%%%%%%%%%%%%%%%%%%%%%%%%%%%%%%%%%%%%%%%%%%%%%%%%%%%%%%%%%%%%%%%%%%%%%%%%%%%%%%%%%%%%%%%%
\usepackage[abbr]{harvard}
\usepackage{amssymb}
\usepackage{setspace,graphics,epsfig,amsmath,rotating,amsfonts,mathpazo}

\setcounter{MaxMatrixCols}{10}
%TCIDATA{OutputFilter=LATEX.DLL}
%TCIDATA{Version=5.50.0.2953}
%TCIDATA{<META NAME="SaveForMode" CONTENT="1">}
%TCIDATA{BibliographyScheme=BibTeX}
%TCIDATA{Created=Thursday, September 11, 2008 15:11:56}
%TCIDATA{LastRevised=Friday, December 06, 2013 11:30:49}
%TCIDATA{<META NAME="GraphicsSave" CONTENT="32">}
%TCIDATA{<META NAME="DocumentShell" CONTENT="Articles\SW\article.egan">}

\topmargin=0 in \headheight=0in \headsep=0in \topskip=0in \textheight=9in \oddsidemargin=0in \evensidemargin=0in \textwidth=6.5in
\input{tcilatex}
\begin{document}


New York University

Wilf Family Department of Politics

Fall 2013

\begin{center}
{\large \textbf{Quantitative Research in Political\ Science I}}

Professor Patrick Egan

\bigskip

\textbf{PROBLEM\ SET 9: Due Friday, December 13 at 10 a.m.}
\end{center}

\textit{A reminder: you may work with others in the class on this problem
set, and you are in fact encouraged to do so. \ However, the work you hand
in must be your own. \ Your work must be word-processed in order for you to
receive credit for the assignment.}

\begin{enumerate}
\item Consider three random variables $X,$ $Y$ and $Z$, where%
\begin{eqnarray*}
cov(Z,Y) &<&0;\quad cov(Z,X)=0\text{; } \\
\text{and }cov(X,Y) &>&0\text{.}
\end{eqnarray*}%
\smallskip 

Assume we have a large number of observations of the joint distribution of
all variables from an i.i.d. random sample yielded by a data generating
process governed by the linear model%
\begin{equation*}
y=\alpha +\beta x_{i}+\delta z_{i}+u_{i}.
\end{equation*}%
\smallskip \qquad\ If we estimate the equation 
\begin{equation*}
\widehat{y_{i}}=\widehat{\alpha }+\widehat{\beta }x_{i},
\end{equation*}

\begin{enumerate}
\item What is the formula OLS uses to generate $\widehat{\beta }$? $\ $

\item What is $E\left( \widehat{\beta }\right) $?

\item Is $\widehat{\beta }$ is a biased estimate of the parameter $\beta $
due to the omission of $z$? \ Explain your answer in both words and
mathematics.\bigskip \bigskip

Now instead if we estimate the equation 
\begin{equation*}
\widehat{y_{i}}=\widehat{\alpha }+\widehat{\delta }z_{i},
\end{equation*}

\item What is the formula OLS uses to generate $\widehat{\delta }$? $\ $

\item What is $E\left( \widehat{\delta }\right) $?

\item Is $\widehat{\delta }$ is a biased estimate of the parameter $\delta $
due to the omission of $x$? \ Explain your answer in both words and
mathematics.\newpage

Now consider a different DGP governed by the model%
\begin{equation*}
y=\alpha +\beta x_{i}+\gamma w_{i}+u_{i},
\end{equation*}%
\bigskip\ where $W$ is a random variable with $cov(W,X)<0$ and $cov(W,Y)<0.$
\ If we estimate the equation 
\begin{equation*}
\widehat{y_{i}}=\widehat{\alpha }+\widehat{\gamma }w_{i},
\end{equation*}

\item What is the formula OLS uses to generate $\widehat{\gamma }$? $\ $

\item What is $E\left( \widehat{\gamma }\right) $?

\item Is $\widehat{\gamma }$ is a biased estimate of the parameter $\gamma $
due to the omission of $x$? \ Explain your answer in both words and
mathematics. \bigskip \bigskip 

Let's say that a measure of $X$ is not available and we thus have no choice
but to estimate $\widehat{y_{i}}=\widehat{\alpha }+\widehat{\gamma }w_{i}$
even though we know that the proper model is $y=\alpha +\beta x_{i}+\gamma
w_{i}+u_{i}.$ \ Say whether the following statements are TRUE\ or FALSE and
explain why.\bigskip

\item Our estimate of $\gamma $ will be biased upward.

\item An estimate of $\widehat{\gamma }>0$ leaves us quite confident that $%
\gamma >0.$

\item An estimate of $\widehat{\gamma }<0$ leaves us quite confident that $%
\gamma <0.$\bigskip
\end{enumerate}

\item For this question, please use the \textit{\ counties.dta} dataset you
used for Problem Set \#8. \ Analyze the following questions using Stata, but
answer them with a few sentences in plain English. \ Provide appropriate
Stata output as an attachment to your assignment.\bigskip 

\begin{enumerate}
\item You are interested in the relationship between population density and
crime in the state of California. \ Estimate the linear model $\widehat{%
\mathit{crimerate}}$ =$\widehat{\beta }_{0}+\widehat{\beta }_{1}$\textit{%
density} using \texttt{regress,}limiting your analysis only to counties in
California.\texttt{\ }Find $\widehat{\beta }_{1},$and provide a one-sentence
English-language interpretation of this coefficient.

\item Construct a plot of the estimated linear model $\widehat{\mathit{%
crimerate}}$ =$\widehat{\beta }_{0}+\widehat{\beta }_{1}$\textit{density}
using Stata's \texttt{twoway, scatter} and \texttt{lfit }commands, again
limiting your analysis only to counties in California.. \ For legibility
purposes, you may want to remove the legend using the \texttt{legend(off) }%
option.

\item A glance at the plot suggests that there is an outlier that may be
affecting your results in a substantial fashion. \ Describe what you see and
say how this might influence our estimate of $\widehat{\beta }_{1}.$

\item Modify the graph you constructed in (b) in a way that allows you to
identify the outlier. \ (Hint: to label points with a variable name, use the
option \texttt{mlab(\textit{varname})}with the \texttt{scatter }command,
where \texttt{\textit{varname}}\textit{\ }is the name of the variable you
want to use to create the labels).

\item Re-run your estimation, eliminating the outlying county. \ How did
your estimates change? \ Did they do so in the way that you expected?

\item Counties typically consist of many municipalities. \ If we were to run
this analysis using California \textit{municipalities} instead of California 
\textit{counties}, $VAR\left( \widehat{\beta }_{1}\right) $ would definitely
change in two very helpful ways for us. \ What are they and why would they
be helpful?
\end{enumerate}
\end{enumerate}

\end{document}

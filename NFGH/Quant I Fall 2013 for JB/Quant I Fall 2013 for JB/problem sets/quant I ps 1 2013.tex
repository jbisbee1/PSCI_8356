
\documentclass[11pt]{article}
%%%%%%%%%%%%%%%%%%%%%%%%%%%%%%%%%%%%%%%%%%%%%%%%%%%%%%%%%%%%%%%%%%%%%%%%%%%%%%%%%%%%%%%%%%%%%%%%%%%%%%%%%%%%%%%%%%%%%%%%%%%%%%%%%%%%%%%%%%%%%%%%%%%%%%%%%%%%%%%%%%%%%%%%%%%%%%%%%%%%%%%%%%%%%%%%%%%%%%%%%%%%%%%%%%%%%%%%%%%%%%%%%%%%%%%%%%%%%%%%%%%%%%%%%%%%
\usepackage[abbr]{harvard}
\usepackage{amssymb}
\usepackage{setspace,graphics,epsfig,amsmath,rotating,amsfonts,mathpazo}

\setcounter{MaxMatrixCols}{10}
%TCIDATA{OutputFilter=LATEX.DLL}
%TCIDATA{Version=5.50.0.2960}
%TCIDATA{<META NAME="SaveForMode" CONTENT="1">}
%TCIDATA{BibliographyScheme=BibTeX}
%TCIDATA{Created=Thursday, September 11, 2008 15:11:56}
%TCIDATA{LastRevised=Wednesday, September 25, 2013 12:24:32}
%TCIDATA{<META NAME="GraphicsSave" CONTENT="32">}
%TCIDATA{<META NAME="DocumentShell" CONTENT="Articles\SW\article.egan">}

\topmargin=0 in \headheight=0in \headsep=0in \topskip=0in \textheight=9in \oddsidemargin=0in \evensidemargin=0in \textwidth=6.5in
\input{tcilatex}
\begin{document}


New York University

Wilf Family Department of Politics

Fall 2013

\begin{center}
{\large \textbf{Quantitative Research in Political\ Science I}}

(POL-GA.1250)

Professor Patrick Egan

\bigskip

\textbf{PROBLEM\ SET 1: Due Monday, September 30 at beginning of class.}
\end{center}

\textit{A reminder: you may work with others in the class on this problem
set, and you are in fact encouraged to do so. \ However, the work you hand
in must be your own. \ Handwritten work is acceptable, but word-processed
work (e.g., using LaTeX) is preferred.}\bigskip 

\bigskip \textit{The following two questions are designed to get you nimble
with the sorts of proofs we'll be doing in class with scalar algebra and
summation signs:}

\begin{enumerate}
\item Prove that the sum of deviations of a set of measurements from their
mean is equal to zero, that is:%
\begin{equation*}
\underset{i=1}{\overset{N}{\sum }}\left( y_{i}-\overline{y}\right) =0.
\end{equation*}%
\bigskip 

\item Show that the variance of an empirical distribution is equal to the
average of the squared observations minus the square of the average
observation, that is:

\begin{equation*}
\frac{1}{N}\underset{i=1}{\overset{N}{\sum }}\left( y_{i}-\overline{y}%
\right) ^{2}=\frac{_{_{^{^{\underset{i=1}{\overset{N}{\sum }}\left(
y_{i}\right) ^{2}}}}}}{N}-\left( \overline{y}\right) ^{2}
\end{equation*}%
\bigskip 

\textit{Most of the remaining problems come from WMS, \textbf{7th edition}.
\ Note that the answers to odd-numbered, non-asterisked problems may be
found in the back of the book. \ (In contrast to those answers, please show
all your work.)}\bigskip 

\item WMS Exercise 2.77 

\item WMS Exercise 2.78.

\item WMS Exercise 2.83.

\item WMS Exercise 2.106.

\item WMS Exercise 2.117.

\item You are an R.A. for a Politics professor conducting a lab experiment
at NYU. \ The experiment is desperately seeking students who identify as
Republican. \ There are very few Republican undergraduates at NYU (only 10
percent of undergrads identify as Republicans) and so most are reluctant to
disclose their party affiliation. \ With just minutes to go before the lab
experiment, you run out onto Mercer\ Street to try to find subjects. \
Knowing that many students will refuse to tell you if they are Republicans,
you wonder if there is a less obtrusive question you can ask. \ \bigskip 

Because NYU's business school (Stern) is known to enroll many Republican
students, you first consider asking students if they are enrolled at Stern.
\ ($65\%$ of the Republicans at NYU are enrolled at Stern, but only $15\%$
of non-Republicans at NYU are enrolled at Stern.) \bigskip 

\begin{enumerate}
\item What is $P($Republican\TEXTsymbol{\vert}enrolled at Stern$)$ ?\bigskip 

\item What is $P($Republican $\cap $ enrolled at Stern$)$?\bigskip 

\item If you ask 100 random NYU students if they are enrolled at Stern and
bring those who say "yes" to the lab, how many\ Republicans should you
expect this to yield?\bigskip \bigskip 

Your answer to (c) seems like a low number, so you decide to ask a question
that will yield more "yeses": ask students their favorite color. \ (Blue is
the favorite color of $50\%$ of NYU undergrads, regardless of their party
affiliation.)\bigskip 

\item What is $P($Republican\TEXTsymbol{\vert}favorite color is blue$)$
?\bigskip 

\item What is $P($Republican $\cap $ favorite color is blue)?\bigskip 

\item If you ask 100 random NYU students their favorite color and bring
those who say "blue" to the lab, how many\ Republicans should you expect
this to yield?\bigskip 

\item OK, one more idea: ask 100 students whether they are left- or
right-handed and bring those saying "right-handed" to the lab. \ To
determine whether this method is expected to yield more students than either
of the two methods discussed so far, which of the following pieces of
information would you like to know, and why?

\begin{itemize}
\item the probability that a Republican is right-handed

\item the probability that a right-handed person is Republican
\end{itemize}
\end{enumerate}
\end{enumerate}

\end{document}

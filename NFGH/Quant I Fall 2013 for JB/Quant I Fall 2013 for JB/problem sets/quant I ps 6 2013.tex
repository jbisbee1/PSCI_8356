
\documentclass[11pt]{article}
%%%%%%%%%%%%%%%%%%%%%%%%%%%%%%%%%%%%%%%%%%%%%%%%%%%%%%%%%%%%%%%%%%%%%%%%%%%%%%%%%%%%%%%%%%%%%%%%%%%%%%%%%%%%%%%%%%%%%%%%%%%%%%%%%%%%%%%%%%%%%%%%%%%%%%%%%%%%%%%%%%%%%%%%%%%%%%%%%%%%%%%%%%%%%%%%%%%%%%%%%%%%%%%%%%%%%%%%%%%%%%%%%%%%%%%%%%%%%%%%%%%%%%%%%%%%
\usepackage[abbr]{harvard}
\usepackage{amssymb}
\usepackage{setspace,graphics,epsfig,amsmath,rotating,amsfonts,mathpazo}

\setcounter{MaxMatrixCols}{10}
%TCIDATA{OutputFilter=LATEX.DLL}
%TCIDATA{Version=5.50.0.2960}
%TCIDATA{<META NAME="SaveForMode" CONTENT="1">}
%TCIDATA{BibliographyScheme=BibTeX}
%TCIDATA{Created=Thursday, September 11, 2008 15:11:56}
%TCIDATA{LastRevised=Friday, November 01, 2013 18:03:45}
%TCIDATA{<META NAME="GraphicsSave" CONTENT="32">}
%TCIDATA{<META NAME="DocumentShell" CONTENT="Articles\SW\article.egan">}

\topmargin=0 in \headheight=0in \headsep=0in \topskip=0in \textheight=9in \oddsidemargin=0in \evensidemargin=0in \textwidth=6.5in
\input{tcilatex}
\begin{document}


New York University

Wilf Family Department of Politics

Fall 2013

\begin{center}
{\large \textbf{Quantitative Research in Political\ Science I}}

Professor Patrick Egan

\bigskip

\textbf{PROBLEM\ SET 6: Due Tuesday, November 6 at 10 a.m.}
\end{center}

\textit{A reminder: you may work with others in the class on this problem
set, and you are in fact encouraged to do so. \ However, the work you hand
in must be your own. \ Handwritten work is acceptable, but word-processed
work (e.g., using LaTeX) is preferred.}

\bigskip

\begin{enumerate}
\item You are studying the i.i.d. random variable, $Y.$ \ You are conducting
a hypothesis test with null and alternative $H_{0}:\mu _{Y}=24,$ $H_{A}:$ $%
\mu _{Y}>24.$ \ You draw a sample of $N=25$ and find that $\overline{Y}=25$
and $S_{U}=12.$

\begin{enumerate}
\item If $\alpha =.10,$ for what value of $\overline{Y}$ or greater do we
reject $H_{0}$ if we conduct a $z$-test? \ 

\item If $\alpha =.10,$ for what value of $\overline{Y}$ or greater do we
reject $H_{0}$ if we conduct a $t$-test? \ 

\item Now answer (a) and (b) again, but this time stipulate $\alpha =.001.$

\item What do the results in (a) through (c) suggest about how the
difference between $z$-tests and $t$-tests changes with $\alpha $? \ Given
what we know about the shape of the $t$ and the Normal distributions, why
does this make sense? \ (If you're stumped, it may help to look at Figure
7.3, page 360 of your text.) \ Provide a diagram (you can draw it by hand)
explaining your answer. \ 

\item What is the \textit{attained level of significance }(also known as the 
$p$-value) associated with the finding $N=25,\overline{Y}=25,$ $S_{U}=12$ if
we conduct a $t$-test?

\item What is the \textit{power} of this test if we stipulate $H_{A}:$ $\mu
_{Y}=27$ and $\alpha =.10$?

\item Given these parameters, what \textit{sample size }would we need for
our hypothesis test's power to equal or exceed $.80$?\medskip
\end{enumerate}

\item Consider two 95\% CIs drawn around some parameter $\theta $ that has
standard deviation $\sigma _{\theta }.$ \ The first CI is drawn with the
assumption that the sampling distribution of the standardized version of
estimator $\widehat{\theta }$ of the parameter $\left( \frac{\widehat{\theta 
}-\theta }{\sigma \widehat{_{\theta }}}\right) $ follows the $t$%
-distribution. \ The second CI is drawn with the assumption that the
sampling distribution follows the standard Normal. \ 

\begin{enumerate}
\item Show that the magnitude of the difference between the lower bounds of
these two confidence intervals is always equal to%
\begin{equation*}
\left( t_{\frac{a}{2},n-1}-z_{\frac{a}{2}}\right) \frac{\sigma _{\theta }}{%
\sqrt{n}}.
\end{equation*}

\item Express the magnitude of the difference in the lower bounds of these
two confidence intervals in terms of $\sigma _{\theta }$ for $n=1000;$ $%
n=100;$ and $n=25.$ \ Do these differences seem very large to you? In a few
sentences, say what you learn from this analysis.\medskip
\end{enumerate}

\item When we conduct a $t$-test, we are assuming that we are working with a
sample drawn from a population whose distribution follows the Normal. \ But
how well does the $t$-test work in small samples that are not Normal and are
in fact quite skewed? \ Using techniques learned in lab, do the following:

\begin{enumerate}
\item We'll be simulating a random variable that follows the \textit{gamma
distribution} with parameters $\alpha =1$ and $\beta =1.$ \ To get a sense
of what this distribution looks like, first draw a sample of $n=1,000$ from
the distribution using the Stata function \textbf{rgamma(1,1) }and graph the
distribution of these observations. \ Pretty skewed, right?

\item A random variable that follows the gamma distribution has mean $\alpha
\beta $ and variance $\alpha \beta ^{2}.$ \ So in our case, $\mu =1$ and $%
\sigma ^{2}=1.$ Show that therefore a hypothesis test with a .05 level of
significance should inaccurately reject the null $H_{0}:$ $\mu =1$ in favor
of the alternative $H_{A}:$ $\mu \neq 1$ only 5\% of the time. \ 

\item Now let's see if this is the case when we conduct a $t$-test on small
samples drawn from this distribution. \ Using Stata, do the following:

\begin{itemize}
\item Generate 1,000 samples of $n=15$ each from a gamma distribution with $%
\alpha =1$ and $\beta =1.$

\item In each of the samples, conduct a $t$-test pitting the null $H_{0}:$ $%
\mu =1$ against the alternative $H_{0}:$ $\mu \neq 1.$

\item Record the percentage of times the test (inaccurately) rejects the
null.

\item How close is this to 5\%? \ Does the $t$-test give us more or less
confidence that we are avoiding Type I error than it should? \ By how much?
\end{itemize}

\item Now run the same experiment with samples of $n=29.$ \ How do the
results change from what you found in (b)?

\item Finally, run the experiment again with samples of $n=29$, but now use
a $z$-test. \ How do the results change from what you found in (c)?
\end{enumerate}
\end{enumerate}

\end{document}

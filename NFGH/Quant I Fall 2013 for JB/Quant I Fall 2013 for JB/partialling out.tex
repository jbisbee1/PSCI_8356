
\documentclass[11pt]{article}
%%%%%%%%%%%%%%%%%%%%%%%%%%%%%%%%%%%%%%%%%%%%%%%%%%%%%%%%%%%%%%%%%%%%%%%%%%%%%%%%%%%%%%%%%%%%%%%%%%%%%%%%%%%%%%%%%%%%%%%%%%%%%%%%%%%%%%%%%%%%%%%%%%%%%%%%%%%%%%%%%%%%%%%%%%%%%%%%%%%%%%%%%%%%%%%%%%%%%%%%%%%%%%%%%%%%%%%%%%%%%%%%%%%%%%%%%%%%%%%%%%%%%%%%%%%%
\usepackage[abbr]{harvard}
\usepackage{amssymb}
\usepackage{setspace,graphics,epsfig,amsmath,rotating,amsfonts,mathpazo}

\setcounter{MaxMatrixCols}{10}
%TCIDATA{OutputFilter=LATEX.DLL}
%TCIDATA{Version=5.50.0.2960}
%TCIDATA{<META NAME="SaveForMode" CONTENT="1">}
%TCIDATA{BibliographyScheme=BibTeX}
%TCIDATA{Created=Thursday, September 11, 2008 15:11:56}
%TCIDATA{LastRevised=Monday, November 18, 2013 12:55:31}
%TCIDATA{<META NAME="GraphicsSave" CONTENT="32">}
%TCIDATA{<META NAME="DocumentShell" CONTENT="Articles\SW\article.egan">}

\topmargin=0 in \headheight=0in \headsep=0in \topskip=0in \textheight=9in \oddsidemargin=0in \evensidemargin=0in \textwidth=6.5in
\input{tcilatex}
\begin{document}


New York University

Wilf Family Department of Politics

\begin{center}
{\large \textbf{Quantitative Research in Political\ Science I}}

Professor Patrick Egan

\bigskip

\textbf{Partialling Out\footnote{%
borrowing heavily from\ Angrist and Pischke's \textit{Mostly Harmless
Econometrics} (pp. 35-36).}}
\end{center}

Another way to write the $k$'th element of the vector of OLS estimates $%
\widehat{\mathbf{\beta }}=\left( \mathbf{X}^{\prime }\mathbf{X}\right) ^{-1}%
\mathbf{X}^{\prime }\mathbf{y}$ is:%
\begin{equation*}
\widehat{\beta }_{k}=\frac{cov\left( y_{i},\widetilde{x}_{ki}\right) }{%
var\left( \widetilde{x}_{ki}\right) },
\end{equation*}

where $\widetilde{x}_{ki}$ is the residual of the regression of $x_{k}$ on
all the other $x$'s in the model, i.e.,%
\begin{eqnarray*}
x_{ki} &=&\widehat{\delta }_{0}+\sum\limits_{j\neq k}\widehat{\delta }%
_{j}x_{ji}+\widetilde{x}_{ki} \\
\widetilde{x}_{ki} &=&x_{ki}-\widehat{\delta }_{0}-\sum\limits_{j\neq k}%
\widehat{\delta }_{j}x_{ji} \\
\widetilde{x}_{ki} &=&x_{ki}-\widehat{x}_{ki}
\end{eqnarray*}

This shows us that each coefficient in a multivariate regression is the
bivariate slope coefficient for the corresponding regressor after \textbf{%
partialling out }the variation of all the other covariates with $y$ and
their covariation with $x_{k}$.\bigskip\ \ 

A [plain-language] proof: re-write $y_{i}$ as:%
\begin{equation*}
y_{i}=\widehat{\beta }_{0}+\widehat{\beta }_{1}x_{1i}+...+\widehat{\beta }%
_{k}x_{ki}+...+\widehat{\beta }_{K}x_{Ki}+\widehat{u}_{i}
\end{equation*}

Now note that:

\begin{itemize}
\item because $\widetilde{x}_{k}$ is a linear combination of all the $x$'s,
it is by construction uncorrelated with the $\widehat{u}$;

\item because $\widetilde{x}_{k}$ is a residual from a regression on all the
other $x$'s in the model, it is by construction uncorrelated with these $x$'s

\item and for the same reason, $cov\left( \widetilde{x}_{ki},x_{ki}\right) $
is just $var\left( \widetilde{x}_{ki}\right) $

\item thus $cov\left( y_{i},\widetilde{x}_{ki}\right) $ simplifies to 
\begin{eqnarray*}
cov\left( y_{i},\widetilde{x}_{ki}\right) &=&cov\left( \widehat{\beta }%
_{k}x_{ki},\widetilde{x}_{ki}\right) \\
&=&\widehat{\beta }_{k}var\left( \widetilde{x}_{ki}\right)
\end{eqnarray*}%
so%
\begin{equation*}
\widehat{\beta }_{k}=\frac{cov\left( y_{i},\widetilde{x}_{ki}\right) }{%
var\left( \widetilde{x}_{ki}\right) }.
\end{equation*}
\end{itemize}

Note that the magnitude of $\widehat{\beta }_{k}$ is larger to the extent
that $y$ varies with the variation in $x_{k}$ after accounting for $x_{k}$'s
covariation with the other $x$'s in the model. \ This nicely corresponds to
our notion of $\widehat{\beta }_{k}$ as an estimate of the \textit{ceteris
paribus}/all things being equal relationship between $y$ and $x_{k}$.\bigskip

It can also be shown that 
\begin{equation*}
\widehat{\beta }_{k}=\frac{cov\left( \widetilde{y}_{ki},\widetilde{x}%
_{ki}\right) }{var\left( \widetilde{x}_{ki}\right) },
\end{equation*}%
where $\widetilde{y}_{ki}$ is the residual from the regression of $y$ on all
the $x$'s \textit{except }$x_{k}$ in the model. \bigskip

One thing that's nice about both of these formulations is that in a
bivariate plot of either $y$ on $\widetilde{x}_{k}$ or $\widetilde{y}_{k}$
on $\widetilde{x}_{k}$, the slope of the (bivariate) best-fit line is, in
fact, $\widehat{\beta }_{k}.$ \ Here's an example from my book. \ I'm
interested in the relationship between the extent to which a political
party's \textquotedblleft owned\textquotedblright\ issues are salient in a
U.S. presidential election campaign and how well the party performs in the
election. \ The table displays a series of regression equations modeling $y$
(election performance) as a function of $x_{k}$ (the salience of the party's
owned issues) and other potential confounding $x$'s. \ The figures plot
(first) $y$ on $x$ and (second) $\widetilde{y}_{k}$ on $\widetilde{x}_{k}.$

\end{document}


\documentclass[11pt]{article}
%%%%%%%%%%%%%%%%%%%%%%%%%%%%%%%%%%%%%%%%%%%%%%%%%%%%%%%%%%%%%%%%%%%%%%%%%%%%%%%%%%%%%%%%%%%%%%%%%%%%%%%%%%%%%%%%%%%%%%%%%%%%%%%%%%%%%%%%%%%%%%%%%%%%%%%%%%%%%%%%%%%%%%%%%%%%%%%%%%%%%%%%%%%%%%%%%%%%%%%%%%%%%%%%%%%%%%%%%%%%%%%%%%%%%%%%%%%%%%%%%%%%%%%%%%%%
\usepackage[abbr]{harvard}
\usepackage{amssymb}
\usepackage{setspace,graphics,epsfig,amsmath,rotating,amsfonts,mathpazo}

\setcounter{MaxMatrixCols}{10}
%TCIDATA{OutputFilter=LATEX.DLL}
%TCIDATA{Version=5.50.0.2960}
%TCIDATA{<META NAME="SaveForMode" CONTENT="1">}
%TCIDATA{BibliographyScheme=BibTeX}
%TCIDATA{Created=Thursday, September 11, 2008 15:11:56}
%TCIDATA{LastRevised=Thursday, November 14, 2013 15:52:34}
%TCIDATA{<META NAME="GraphicsSave" CONTENT="32">}
%TCIDATA{<META NAME="DocumentShell" CONTENT="Articles\SW\article.egan">}

\topmargin=0 in \headheight=0in \headsep=0in \topskip=0in \textheight=9in \oddsidemargin=0in \evensidemargin=0in \textwidth=6.5in
\input{tcilatex}
\begin{document}


\begin{enumerate}
\item[3.] Indicate whether each of the following statements is true or
false. Explain each of your answers, using mathematics where necessary.

\begin{enumerate}
\item 

\item 

\item All things being equal, Type I errors are more likely with small
samples than with large samples.

False. \ The probability of a Type I error is specified by the researcher 
\textit{a priori }as the confidence coefficient $\alpha $, which remains
constant regardless of sample size.

\item All things being equal, Type II errors are more likely with large
samples than with small samples.

True, as shown by when conducting a $z$-test with a null $\theta _{0}$ and
alternative $\theta _{A}$, where $\theta _{0}<\theta _{A}$: 
\begin{eqnarray*}
P\left( \text{Type II error}\right)  &=&\beta =P\left( \text{Reject }%
H_{0}|H_{A}\text{ true}\right)  \\
&=&P\left( \widehat{\theta }<\theta _{0}+z_{\alpha }\sigma _{\widehat{\theta 
}}|\theta =\theta _{A}\right)  \\
&=&\Pr \left( \frac{\widehat{\theta }-\theta _{A}}{\sigma _{\widehat{\theta }%
}}<\frac{\theta _{0}+z_{\alpha }\sigma _{\widehat{\theta }}-\theta _{A}}{%
\sigma _{\widehat{\theta }}}|\theta =\theta _{A}\right)  \\
&=&\Phi \left( \frac{\theta _{0}+z_{\alpha }\sigma _{\widehat{\theta }%
}-\theta _{A}}{\sigma _{\widehat{\theta }}}\right)  \\
&=&\Phi \left( \frac{\theta _{0}-\theta _{A}}{\sigma _{\widehat{\theta }}}%
+z_{\alpha }\right) ,\text{ so} \\
\beta  &=&\Phi \left( \frac{\theta _{0}-\theta _{A}}{\frac{\sigma }{\sqrt{n}}%
}+z_{\alpha }\right) .
\end{eqnarray*}

Noting that $\theta _{0}-\theta _{A}<0,$we see that:%
\begin{equation*}
\frac{\partial \frac{\theta _{0}-\theta _{A}}{\frac{\sigma }{\sqrt{n}}}}{%
\partial n}<0,
\end{equation*}%
and so%
\begin{equation*}
\frac{\partial \beta }{\partial n}<0.
\end{equation*}%
Type II errors are thus \textit{less }likely with large samples than small
samples.

\item 
\begin{equation*}
\frac{\partial \sigma _{\overline{Y}}^{2}}{\partial \sigma _{Y}^{2}}<0.
\end{equation*}

FALSE. 
\begin{eqnarray*}
\sigma _{\overline{Y}}^{2} &=&\frac{\sigma _{Y}^{2}}{n},\text{ so} \\
\frac{\partial \sigma _{\overline{Y}}^{2}}{\partial \sigma _{Y}^{2}} &=&%
\frac{1}{n}>0.
\end{eqnarray*}

\item In an i.i.d. random sample of size $n$ drawn from the population $Y$,
the observation $Y_{1}$ is an unbiased estimate of $\mu _{Y}.$

True. \ Identicality ensures that $E\left( Y_{1}\right) =E\left( Y\right)
=\mu _{Y}.$
\end{enumerate}
\end{enumerate}

\end{document}

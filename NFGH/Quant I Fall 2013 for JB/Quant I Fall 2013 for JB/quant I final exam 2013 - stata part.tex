
\documentclass[11pt]{article}
%%%%%%%%%%%%%%%%%%%%%%%%%%%%%%%%%%%%%%%%%%%%%%%%%%%%%%%%%%%%%%%%%%%%%%%%%%%%%%%%%%%%%%%%%%%%%%%%%%%%%%%%%%%%%%%%%%%%%%%%%%%%%%%%%%%%%%%%%%%%%%%%%%%%%%%%%%%%%%%%%%%%%%%%%%%%%%%%%%%%%%%%%%%%%%%%%%%%%%%%%%%%%%%%%%%%%%%%%%%%%%%%%%%%%%%%%%%%%%%%%%%%%%%%%%%%
\usepackage[abbr]{harvard}
\usepackage{amssymb}
\usepackage{setspace,graphics,epsfig,amsmath,rotating,amsfonts,mathpazo}

\setcounter{MaxMatrixCols}{10}
%TCIDATA{OutputFilter=LATEX.DLL}
%TCIDATA{Version=5.50.0.2960}
%TCIDATA{<META NAME="SaveForMode" CONTENT="1">}
%TCIDATA{BibliographyScheme=BibTeX}
%TCIDATA{Created=Thursday, September 11, 2008 15:11:56}
%TCIDATA{LastRevised=Friday, December 13, 2013 09:54:55}
%TCIDATA{<META NAME="GraphicsSave" CONTENT="32">}
%TCIDATA{<META NAME="DocumentShell" CONTENT="Articles\SW\article.egan">}

\topmargin=0 in \headheight=0in \headsep=0in \topskip=0in \textheight=9in \oddsidemargin=0in \evensidemargin=0in \textwidth=6.5in
\input{tcilatex}
\begin{document}


New York University

Wilf Family Department of Politics

Fall 2013\bigskip

\begin{center}
{\large \textbf{Quantitative Research in Political\ Science I}}

Professor Patrick Egan

\bigskip
\end{center}

\bigskip

\begin{center}
\textbf{FINAL\ EXAMINATION: STATISTICAL COMPUTING PART}

(70 POINTS\ TOTAL)\textbf{\bigskip }

\textit{This exam is open-book, open-note.}
\end{center}

\bigskip

\begin{itemize}
\item To complete this part of the exam, download the dataset \texttt{%
annenberg2004.dta} from our class website (from the "Resources" section). \
If you have trouble doing this, see Andrew or call me (646-808-7271). \
\bigskip 

\item Your responses to the questions on this part of the exam should be
included in a \texttt{.log} file that you will print out and submit for your
exam grade. \ The file must include the commands you use in the analysis,
any output, and answers to questions (entered in the \texttt{.log} file as
comments).\bigskip

\item You must answer the substantive questions on this part of the exam in
words. \ \textbf{You will not earn full credit for Stata output submitted
without written interpretation.}
\end{itemize}

\newpage

\bigskip The dataset \texttt{annenberg2004.dta} is a sample of respondents
from the 2004 National\ Annenberg Election\ Survey (NAES), conducted during
the 2004 U.S. Presidential election campaign.\medskip\ \ \bigskip

\textbf{NOTES: \ }

\begin{itemize}
\item In many cases, proper analysis will require recoded versions of
variables in the dataset.

\item Unless specified otherwise, use an alpha of .05 when performing
statistical significance tests.\bigskip \bigskip
\end{itemize}

\begin{enumerate}
\item \textbf{(2 points) }Run the command that lists the names of variables,
their formats and labels.\bigskip

\item \textbf{(18 points) }\textit{bush\_favorability }is a measure of
respondents' favorability ratings of George W.\ Bush on a scale of zero
(least favorable) to ten (most favorable). \ 

\begin{enumerate}
\item What is the mode of \textit{bush\_favorability}? \ 

\item What is its mean?

\item What is the 95\% confidence interval about this mean?

\item What is the 90\% confidence interval about this mean?

\item Who rates Bush more favorably, men or women?

\begin{enumerate}
\item Answer this question with a $t$-test. \ Interpret your results.

\item Answer this question with a bivariate regression. \ Interpret your
results.
\end{enumerate}

\item Are higher income Americans more likely to assess Bush favorably than
lower income Americans? \ Answer this question under the assumption that 
\begin{equation*}
bush\_favorability=\beta _{0}+\beta _{1}income+u,
\end{equation*}%
and that the other Gauss-Markov assumptions are met. \ 

\begin{enumerate}
\item Before doing so, use diagnostics (as we did in class) to examine
carefully how \textit{income }should enter into the model as a predictor. \
Justify and document the decision you make regarding this. \ \textbf{%
Regardless of your decision, note that some recoding of \textit{income }is
necessary here}.

\item Interpret your results.\newpage
\end{enumerate}
\end{enumerate}

\item \textbf{(25 points) }\textit{kerry\_favorability }is a measure of
respondents' ratings of John Kerry on the same scale as \textit{%
bush\_favorability}. \ Create a new variable (as done in class) called 
\textit{rating\_diff }that is equal to \textit{bush\_favorability }minus 
\textit{kerry\_favorability.} \ \ 

\begin{enumerate}
\item Run a regression of \textit{rating\_diff }on the variables \textit{age}%
, \textit{female}, and \textit{income}. \ \ We'll call this Model I.

\item How well does Model I explain variation in the dependent variable? \
Cite two estimated statistics in your response.

\item Now run a regression of \textit{rating\_diff }on the same variables as
well as \textit{ideology--}a variable in which respondents rate themselves
very conservative (1) to very liberal (5) on a five-point scale\textit{. \ }%
(For the moment, treat \textit{ideology} as an interval-level variable.) \
We'll call this Model II.

\begin{enumerate}
\item In a few sentences, explain what happens to the coefficient on \textit{%
female }between Model I and Model II and your substantive interpretion of
this change. \ 

\item How well does Model\ II explain variation in the dependent variable
compared to Model I? \ 

\item What does Model II predict is the difference in \textit{rating\_diff }%
between those who are very conservative and those who are very liberal,
holding the other factors constant? \ Approximately what percentage of the
range of the dependent variable is this difference equal to?

\item What do your answers to (ii) and (iii) suggest about the importance of 
\textit{ideology }in explaining the dependent variable compared to the other
variables in Model II?
\end{enumerate}

\item Finally, run the estimation in a way that does not require that we
assume \textit{ideology }is an interval-level variable. \ Call this equation
Model\ III.

\begin{enumerate}
\item Now think carefully: what would we need to see in Model\ III that
would give us confidence in treating \textit{ideology} as an interval-level
variable in the present context? \ Do we see this here?

\item In a few sentences, say what Model III tells us about the \textit{%
ceteris paribus} relationship between \textit{ideology} and \textit{%
rating\_diff}.
\end{enumerate}

\item Using the proper Stata commands, create a table displaying the
estimates from Models\ I, II and III. \ The table should display regression
coefficients, their standard errors, and markers of statistical significance
in three columns, along with four statistics from each of these regressions (%
$N$, root MSE, $R^{2}$ and adjusted $R^{2}$). \ Variable names should be
displayed on the rows .\bigskip \newpage
\end{enumerate}

\item \textbf{(25 points) }Explore the following question using OLS. \ 

\begin{itemize}
\item We expect that voters who rated their personal economic situation as
poor in 2004 would be likely to blame the incumbent (Bush), and thus have
lower values of \textit{rating\_diff }than those in better economic
circumstances. \ However, we might theorize that the \textit{ceteris paribus 
}association between a voter's personal economic situation and \textit{%
rating\_diff }is even stronger for Independent voters (who are not
affiliated with either the Democratic or Republican parties), as these
Independents do not have the cue of party identification to rely upon when
rating the two candidates.

\item To explore this question, use the variables \textit{pid} (a
categorical variable that is a measure of voters' party identification) and 
\textit{economic\_situation }(a variable measuring how voters rate their
personal economic situation on a scale of 1 (excellent) to 4 (poor)). \ For
purposes of this question, you may treat this variable as interval-level.

\item From here, you're on your own. \ You will need to create new
variables, estimate the appropriate model, present results both in tabular
and graphical form, and interpret the results. \ Do your estimates confirm
the theory? \ Make sure to describe your results in plain English.\bigskip
\end{itemize}

\item Close your \texttt{.log} file email it to Andrew by 3 p.m.. \ Then
print it out a hard copy and put it in Andrew's mailbox. \ 
\end{enumerate}

\end{document}

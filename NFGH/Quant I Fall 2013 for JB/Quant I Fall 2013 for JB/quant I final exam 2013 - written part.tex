
\documentclass[11pt]{article}
%%%%%%%%%%%%%%%%%%%%%%%%%%%%%%%%%%%%%%%%%%%%%%%%%%%%%%%%%%%%%%%%%%%%%%%%%%%%%%%%%%%%%%%%%%%%%%%%%%%%%%%%%%%%%%%%%%%%%%%%%%%%%%%%%%%%%%%%%%%%%%%%%%%%%%%%%%%%%%%%%%%%%%%%%%%%%%%%%%%%%%%%%%%%%%%%%%%%%%%%%%%%%%%%%%%%%%%%%%%%%%%%%%%%%%%%%%%%%%%%%%%%%%%%%%%%
\usepackage[abbr]{harvard}
\usepackage{amssymb}
\usepackage{setspace,graphics,epsfig,amsmath,rotating,amsfonts,mathpazo}

\setcounter{MaxMatrixCols}{10}
%TCIDATA{OutputFilter=LATEX.DLL}
%TCIDATA{Version=5.50.0.2960}
%TCIDATA{<META NAME="SaveForMode" CONTENT="1">}
%TCIDATA{BibliographyScheme=BibTeX}
%TCIDATA{Created=Thursday, September 11, 2008 15:11:56}
%TCIDATA{LastRevised=Thursday, December 12, 2013 17:50:19}
%TCIDATA{<META NAME="GraphicsSave" CONTENT="32">}
%TCIDATA{<META NAME="DocumentShell" CONTENT="Articles\SW\article.egan">}

\topmargin=0 in \headheight=0in \headsep=0in \topskip=0in \textheight=9in \oddsidemargin=0in \evensidemargin=0in \textwidth=6.5in
\input{tcilatex}
\begin{document}


New York University

Wilf Family Department of Politics

Fall 2013\bigskip 

\begin{center}
{\large \textbf{Quantitative Research in Political\ Science I}}

Professor Patrick Egan

\bigskip
\end{center}

\bigskip

\begin{center}
\textbf{FINAL\ EXAMINATION: WRITTEN PART}

(70 POINTS\ TOTAL)\textbf{\bigskip }

\textit{This exam is open-book, open-note.}

\bigskip

\bigskip

\newpage

\bigskip
\end{center}

\begin{enumerate}
\item \textbf{(15 points)} \ \ Consider the following (admittedly simple)
example. \ A DGP defined by the population model $y=\beta _{0}+\beta _{1}x+u$
gives rise to the following dataset of 4 observations:%
\begin{equation*}
\mathbf{y}=%
\begin{bmatrix}
4 \\ 
2 \\ 
-6 \\ 
1%
\end{bmatrix}%
\qquad \text{and \ \ \ \ }\mathbf{X}=%
\begin{bmatrix}
1 & 2 \\ 
1 & 3 \\ 
1 & -1 \\ 
1 & 4%
\end{bmatrix}%
,
\end{equation*}

where the second column of $\mathbf{X}$ is composed of observations of the
variable $x$. \ In answering the following questions, be sure to show all
your work. \ \bigskip

\begin{enumerate}
\item Show that the OLS estimates $\widehat{\beta }_{0}\approx -2.89$ and $%
\widehat{\beta }_{1}\approx 1.57.$ \ You will be glad to know that 
\begin{equation*}
\left( \mathbf{X}^{\prime }\mathbf{X}\right) ^{-1}\approx 
\begin{bmatrix}
.536 & -.143 \\ 
-.143 & .071%
\end{bmatrix}%
.
\end{equation*}

\item Show that $\widehat{\sigma }\equiv SEE\approx 3.33$.\bigskip

\item Show that $R^{2}\approx .61$ .\newpage
\end{enumerate}

\item \textbf{(15 points)} \ Consider four random variables $W,$ $X,$ $Y$
and $Z$, where%
\begin{gather*}
cov(W,Y)>0;\quad cov(W,X)=0; \\
cov(Z,Y)=0;\quad cov(Z,X)<0\text{, } \\
\text{and }cov(X,Y)\text{ is unknown.}
\end{gather*}%
\smallskip

Say whether the following statements are TRUE or FALSE, and explain why. \
Assume we have a large number of observations of the joint distribution of
all four variables from an i.i.d. random sample.

\begin{enumerate}
\item If we estimate the equation%
\begin{equation*}
\widehat{y_{i}}=\widehat{\beta }_{0}+\widehat{\beta }_{1}x_{i},
\end{equation*}%
$\widehat{\beta }_{1}$ is a \textit{biased} estimate of the parameter $\beta
_{1}$ due to the omission of $w$ and $z$. \bigskip \bigskip

\item The estimate of the parameter $\beta _{1}$ we obtain from the
estimated equation 
\begin{equation*}
\widehat{y_{i}}=\widehat{\beta }_{0}+\widehat{\beta }_{1}x_{i}
\end{equation*}%
will be \textit{more efficient} than the estimate of the parameter $\beta
_{1}$ obtained from the equation%
\begin{equation*}
\widehat{y_{i}}=\widehat{\beta }_{0}+\widehat{\beta }_{1}x_{i}+\widehat{%
\beta }_{2}z_{i}.
\end{equation*}%
\bigskip

\item The estimate of the parameter $\beta _{1}$ we obtain from the
estimated equation%
\begin{equation*}
\widehat{y_{i}}=\widehat{\beta }_{0}+\widehat{\beta }_{1}x_{i}
\end{equation*}%
will be \textit{more efficient} than the estimate of $\beta _{1}$ obtained
from the equation%
\begin{equation*}
\widehat{y_{i}}=\widehat{\beta }_{0}+\widehat{\beta }_{1}x_{i}+\widehat{%
\beta }_{2}w_{i}.
\end{equation*}%
\newpage
\end{enumerate}

\item \textbf{(15 points) \ }Consider three variables $X,$ $Y$ and $Z$,
where in the population

\begin{itemize}
\item $X$ takes on the value zero 50 percent of the time and the value one
50 percent of the time, while

\item $Z$ takes on the value zero 3 percent of the time and the value one 97
percent of the time.
\end{itemize}

\bigskip

You are interested in estimating the \textit{ceteris paribus }association of 
$X$ with $Y$ as well as the \textit{ceteris paribus }association of $Z$ with 
$Y.$ \ To do so, you use the model 
\begin{equation*}
y_{i}=\beta _{0}+\beta _{1}x_{i}+\beta _{2}z_{i}+u_{i}.
\end{equation*}

Assume that this model is properly specified, and the Gauss-Markov
assumptions hold.\bigskip

\begin{enumerate}
\item One or more of the following four statements is true. \ In a few
sentences, identify the correct statement(s) and explain:%
\begin{eqnarray*}
var\left( \beta _{1}\right) &=&\frac{\sigma ^{2}}{n\cdot var\left( x\right)
\cdot \left( 1-R_{x}^{2}\right) }\text{ \ \ \ \ \ \ \ \ \ \ \ \ \ \ \ \ }%
\widehat{var\left( \widehat{\beta _{1}}\right) }=\frac{\widehat{\sigma }^{2}%
}{n\cdot var\left( x\right) \cdot \left( 1-R_{x}^{2}\right) } \\
var\left( \widehat{\beta _{1}}\right) &=&\frac{\sigma ^{2}}{n\cdot var\left(
x\right) \cdot \left( 1-R_{x}^{2}\right) }\text{ \ \ \ \ \ \ \ \ \ \ \ \ \ \
\ \ \ }\widehat{var\left( \widehat{\beta _{1}}\right) }=\frac{\sigma ^{2}}{%
n\cdot var\left( x\right) \cdot \left( 1-R_{x}^{2}\right) }
\end{eqnarray*}%
\bigskip

\item It is the case that $R_{x}^{2}=R_{z}^{2}.$ \ Why can we say for sure
that 
\begin{equation*}
var\left( \widehat{\beta _{1}}\right) <var\left( \widehat{\beta _{2}}\right) 
\text{ \ ?}
\end{equation*}%
\bigskip

\item All things being equal, with which of the two findings should you be
more comfortable? \ Why?

\begin{itemize}
\item A failure to reject the null that the \textit{ceteris paribus }%
association between $Z$ and $Y$ is zero.

\item A failure to reject the null that the \textit{ceteris paribus }%
association between $X$ and $Y$ is zero. \newpage
\end{itemize}
\end{enumerate}

\item \textbf{(25 points) \ }Consider the Stata output on the following
page. \ It is an OLS analysis of "feeling thermometer" ratings given to
Barack Obama (on a zero to 100 scale) in the 2012 American National\
Election\ Studies by a nationally representative sample of American adults.
\ Be sure to explain your answers and show your work.\bigskip 

\begin{enumerate}
\item What proportion of the respondents in the sample own guns?\bigskip 

\item What is the rating predicted to be given to\ Obama by a
(non-Hispanic)\ African American man born in the U.S. whose education and
age are equal to the American average, whose household income is \$60,000,
and who is a military veteran and a union member but who is not a gun
owner?\bigskip 

\item How many standard deviations away from $y$ is the typical prediction $%
\widehat{y}$ ?\bigskip 

\item The constant term in the regression $\approx 72.$ \ Describe the
hypothetical American whose predicted rating of Obama is indicated by this
term (however nonsensical the prediction may be).\bigskip 

\item What is the approximate predicted difference in ratings given to Obama
between someone with a household income of \$30,000 and someone with an
income of \$45,000, holding all other covariates constant?\bigskip 

\item What is $\frac{\partial ObamaFT}{\partial AGE}$? \ Your response
should include both a mathematical expression and a few sentences of
explanation.\bigskip 

\item What is $\frac{\partial ObamaFT}{\partial EDUC}$? \ Your response
should include both a mathematical expression and a few sentences of
explanation.
\end{enumerate}
\end{enumerate}

\end{document}

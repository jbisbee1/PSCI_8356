
\documentclass[11pt]{article}
%%%%%%%%%%%%%%%%%%%%%%%%%%%%%%%%%%%%%%%%%%%%%%%%%%%%%%%%%%%%%%%%%%%%%%%%%%%%%%%%%%%%%%%%%%%%%%%%%%%%%%%%%%%%%%%%%%%%%%%%%%%%%%%%%%%%%%%%%%%%%%%%%%%%%%%%%%%%%%%%%%%%%%%%%%%%%%%%%%%%%%%%%%%%%%%%%%%%%%%%%%%%%%%%%%%%%%%%%%%%%%%%%%%%%%%%%%%%%%%%%%%%%%%%%%%%
\usepackage[abbr]{harvard}
\usepackage{amssymb}
\usepackage{setspace,graphics,epsfig,amsmath,rotating,amsfonts,mathpazo}

\setcounter{MaxMatrixCols}{10}
%TCIDATA{OutputFilter=LATEX.DLL}
%TCIDATA{Version=5.50.0.2960}
%TCIDATA{<META NAME="SaveForMode" CONTENT="1">}
%TCIDATA{BibliographyScheme=BibTeX}
%TCIDATA{Created=Sunday, September 21, 2008 09:09:16}
%TCIDATA{LastRevised=Tuesday, October 01, 2013 18:45:14}
%TCIDATA{<META NAME="GraphicsSave" CONTENT="32">}
%TCIDATA{<META NAME="DocumentShell" CONTENT="Articles\SW\article.egan">}

\parskip=0pt
\topmargin=0 in \headheight=0in \headsep=0in \topskip=0in \textheight=9in \oddsidemargin=0in \evensidemargin=0in \textwidth=6.5in
\begin{document}


\singlespacing

\textbf{NYU Department of Politics - Quant I}

\textbf{Fall 2013 - Prof.\ Patrick Egan}

\doublespacing

\section{Lecture 1}

\begin{itemize}
\item Housekeeping.

\begin{itemize}
\item Syllabus:

\begin{itemize}
\item Deliverables.

\begin{itemize}
\item Attendance.

\item Class and lab.

\item Homeworks.

\item Exams.
\end{itemize}

\item Grades.

\item Help.

\begin{itemize}
\item Introduce Andrew.

\item Propose Andrew's office hours; check to see if works.
\end{itemize}

\item Books.

\item Stata.

\item NYU\ Classes. [check.]

\item Hand out questionnaires.
\end{itemize}
\end{itemize}
\end{itemize}

\begin{itemize}
\item Why are you here? \ (i.e., why are you getting a Ph.D. in political
science?)

\begin{itemize}
\item You are here because you enjoy asking and answering questions about
politics.

\begin{itemize}
\item (If not, you're in the wrong line of work.)
\end{itemize}

\item So what kinds of questions do you find interesting? \ List on board.
\end{itemize}

\item This is a course about quantitative analysis in political science.

\begin{itemize}
\item Quantiative analysis takes its place alongside other methodologies in
empirical political science. \ It is particularly helpful and appropriate
for particular kinds of social scientific tasks. \ 

\begin{itemize}
\item Tends to be based on: \textit{numerical }measurements of specific
aspects of social and political phenomena; \ [contrast to: non-numerical]

\item Is more interested in developing and testing generalizable theories
about \textit{multiple }phenomena; [particular cases]

\item Seeks measurements and analyses that are easily \textit{replicable }by
other researchers. [difficult to replicate in its entirety.]
\end{itemize}

\item Contrast this to \textit{qualitative }research [on all three aspects].

\item In addition, appropriateness of \textit{qualitative }research depends
on...

\begin{itemize}
\item Hypothesis testing vs. hypothesis generation

\item Agreed-upon measures of concepts vs. those still up for debate

\item Analyst's willingness to apply less or more \textit{structure} to the
study of the phenomenon
\end{itemize}
\end{itemize}

\item When political scientists work with data, we generally do so for three
reasons:

\begin{itemize}
\item What can we say about the data we have?

\item What can we say about the data we don't have (but we know is out there
in the world) based on the data we do have?

\item What can we say about the data we'd expect to see under a hypothetical
scenario, based on the data we do have?
\end{itemize}

\item These activities generally require three kinds of statistics:

\begin{itemize}
\item Descriptive

\item Inferential (from samples to populations)

\item Prediction (from models to hypothetical scenarios)

\begin{itemize}
\item And lingering behind all of this is a kind of statistic we'll use all
the time: the \textit{test }statistic.
\end{itemize}
\end{itemize}

\item What's a \textit{statistic}? \ 

\begin{itemize}
\item It's a number that summarizes data.
\end{itemize}
\end{itemize}

\subsection{Variables 101}

\begin{itemize}
\item 
\begin{itemize}
\item We study \textbf{units}. \ They are the level at which we wish to make
statements about a social process. \ Units are often also known as \textbf{%
cases}.

\begin{itemize}
\item People, Counties, Nations, Dyads.
\end{itemize}

\item Units have \textbf{attributes}\textit{. \ }An attribute is any
characteristic of a unit that in theory might distinguish it from other
units. \ 

\item \textbf{Variables} are \textit{logical }groupings of \textit{mutually
exclusive }attributes. \ The analyst assigns each attribute a \textbf{value}%
\textit{. \ }We then say that a variable takes on a value for any particular
unit. \ 

\begin{itemize}
\item The variable \textquotedblleft hair color\textquotedblright\ takes on
the value \textquotedblleft red\textquotedblright\ for the unit Me. \ 

\item The variable \textquotedblleft party affiliation\textquotedblright\
takes on the value \textquotedblleft Republican\textquotedblright\ for the
unit Mitt Romney.
\end{itemize}

\item For ease of data manipulation in quantitative analysis, we typically
assign \textbf{scores}\textit{\ }to each potential value a variable can take
on. \ The scores are simply numbers associated with each value. \ Sometimes
the scores are meaningful in their own right; often they are not. \ \ But
it's generally a lot easier to enter and manipulate data via numbers than
via the words we use to describe values.

\item Note that all of these--which units, which attributes, how to group
into variables, how to score the variables--are \textit{choices }that must
be considered and justified by the analyst. \ Typically there are
conventions within subfields of political science that either must be
adhered to (which is what we usually do), or if we depart from them we need
to justify this departure. \ Sometimes the departure itself is a noteworthy
innovation. \ Going to leave aside here a whole field of theory on how we
move from concepts to \textbf{measures}. \ In this class, in most cases,
we'll take the measures as given.
\end{itemize}
\end{itemize}

\subsection{Levels of measurement}

\begin{itemize}
\item Variables can be measured at various levels. \ We'll consider four
such levels in our class:

\begin{itemize}
\item Nominal

\item Ordinal

\item Interval

\item Ratio
\end{itemize}

\item Variables measured at the \textbf{nominal }level take on values that
cannot be ordered in any logical way.

\begin{itemize}
\item Egs.
\end{itemize}

\item Variables measured at the \textbf{ordinal} level taken on values that
can be rank ordered, but that's it. \ We know nothing about how much more or
less one value is than another.

\begin{itemize}
\item Egs.
\end{itemize}

\item Variables measured at the \textbf{interval} level take on values whose
differences can be meaningfully compared. \ I.e., the interval between two
values is meaningful.

\begin{itemize}
\item Eg: Fahrenheit. \ The Year.
\end{itemize}

\item Variables measured at the \textbf{ratio} level take on values such
that the value of zero is meaningful in a specific sense: it means \textit{%
nothing} of the quantity being measured.. \ 

\begin{itemize}
\item Eg. \ income. \ height. \ age...number of wars...miles. \ 
\end{itemize}

\item Mathematical operations:

\begin{itemize}
\item Nominal: equality.

\item Ordinal. \ equality. \ greater than or less than.

\item Interval: addition and subtraction; averages

\item Ratio: \ multiplication and division (i.e. ratios) - "twice as tall,"
"six times as wealthy"

\item Note that we generally have more information as we move up the ladder.
\ Your first instinct should be to use measures that retain as much
information about your units as possible.
\end{itemize}

\item Three trickier cases:

\begin{itemize}
\item Richter (ratio, although slightly tricky: $y_{Richter}=$ $\log _{10}[$%
shaking amplitude$]$, so each unit represents a \textit{doubling} of
amplitude.) \ 

\item Celsius has a meaningful zero (the freezing point of water), but it is 
\textit{not} ratio level. \ Why? \ Because "zero degrees Celsius" $\neq $
zero of the quantity "temperature." \ [E.g., is a 30-degree Celsius day
(86F) thirty times as hot as a one-degree Celsius day (32F)? \ Not
meaningful.] \ Contrast to "zero miles" or "zero years old." \ Celsius is
interval level. \ Contrast to Kelvin, measured at the absolute zero. \ Zero
degrees K = -273.15 C; 1 degree K = 0 C, and so on.

\item Latitiude, longitude: for all intents and purposes, ratio. \ Of
course, zero on the latitiude scale doesn't mean "zero distance." \ It means
we are at the Equator. \ But as a unit of measure, to travel two degrees
latitude is to travel twice as far as one degree latitude. If you travel
from the Equator to Vermont, at 45 degrees latitude North, you've covered 
\textit{half the distance} to the North\ Pole (at 90 degrees latitude). \
Therefore ratio.
\end{itemize}

\item A special case: the dichotomous variable. \ Two ways to consider it:
as a nominal variable with two categories, or as a ratio variable with two
values, zero and one. \ E.g. "gender" versus "female."

\item Choice of level of measurement is often up to the analyst. \ Many
underlying concepts yield several choices for levels of measurement. Again,
your first instinct should be to use measures that retain as much
information about your units as possible. \ E.g.:

\begin{itemize}
\item Location of residence: \ region of country [nominal], county
[nominal], zip code [nominal!], latitude and longitude [ratio].

\item Hair color: common usage [nominal], amount of pheomelanin and
eumelanin (hair pigments) [ratio]

\item Income: poor, middle class, rich [ordinal]; dollars per year [ratio]
\end{itemize}
\end{itemize}

\subsection{Data structures}

\begin{itemize}
\item The typical way that we store data (and really, the way that we think
about data structures) is via a \textbf{data table}. \ Excel, any
statistical software program. \ Each unit is given a row. \ Each variable is
given a column. \ [You could of course do it the other way, but as in many
things you'll find that you save time and headaches by being consistent with
convention.] \ The scores (and/or in some cases, the values) the variables
take on for each case are entered in the cells. \ [Draw on board.]
\end{itemize}

\subsection{Summarizing data: displays}

\begin{itemize}
\item So we've got a group of units, and we know the values taken on by a
set of variables in this set of units. \ Our next move is typically to say
something meaningful about the group with the data at hand. [Gesture to
table.] \ Why not just present this? \ 

\item You're not going to get any more detailed than this. \ But
descriptions of data usually involve tradeoffs between detail and parsimony.

\item So let's move to a higher level of abstraction. \ A frequency table.

\begin{itemize}
\item Displays the frequency with which a variable takes on values in a
group of units.

\item Columns with value, number of units, proportion of units

\item One row for each value.

\item And for ordinal, interval, and ratio-level data: cumulative percentile.
\end{itemize}

\item When displaying frequency distributions, sometimes it makes more sense
to do so graphically. \ Typically graphs move a bit toward parisomony at the
expense of some detail. \ [scale on board] \ What information is missing
from the graph that isn't in the table?

\begin{itemize}
\item Another choice: \textbf{recode} the data into categories and then
produce either a frequency table or a bar chart of the categories. [e.g. on
board]
\end{itemize}

\item When we move to interval- or ratio-level data with lots of categories,
a histogram is almost always preferred to a frequency table.
\end{itemize}

\subsection{Summarizing data: measures of central tendency and dispersion}

\begin{itemize}
\item An even higher level of abtraction: measures of central tendency and
measures of dispersion.

\begin{itemize}
\item Measures of \textbf{central tendency }tell us about the \textit{%
typical value }of the variable in a group of units.

\begin{itemize}
\item mode: the value of the variable most frequently observed in the group
of units (all LOMs; wait to say this): ;

\item median: the value of the smallest observation for which the cumulative
percentage is 50 or greater (ordinal and up);

\item mean - the average: $\overline{y}=\frac{1}{N}\overset{N}{\underset{i=1}%
{\sum }}y_{i}$ \ \ (interval and up). \ 

\item Note that all of these measures summarize all the observations of a
variable with just one number.
\end{itemize}

\item Measures of \textbf{dispersion} typically accompany measures of
central tendency. \ They provide a sense of the \textquotedblleft
spread\textquotedblright\ of a variable's distribution -- a.k.a. the amount
of \textit{variation }in its distribution.

\begin{itemize}
\item \textit{range }(the difference between largest and smallest values);

\item \textit{IQR} (the difference between the values at the 75\%ile and
25\%ile,

\item \textit{variance} $s^{2}=\frac{1}{N}\overset{N}{\underset{i=1}{\sum }}%
(y_{i}-\overline{y})^{2}.$ \ Note different from book, because more
intuitive. \ The standard deviation (s.d.) is simply $\sqrt{s^{2}}.$ \ The
s.d. is a more informative measure of dispersion: it is the average distance
of an observation from the mean. \ It is measured in the same units as the
variable itself. \ What sign must the variance and thus s.d. always be? \
(LOM: interval or higher):
\end{itemize}

\item Furthermore, social scientists often speak \textit{qualitatively}
about the frequency distribution of a variable:

\begin{itemize}
\item It may be symmetric [draw] or skewed [draw], also number of children
on handout.

\begin{itemize}
\item Typically the median is a better measure of central tendency for
variables with skewed distributions than the mean, because it is resistant
to outliers. \ Use age at married example.
\end{itemize}

\item It may be unimodal or bimodal [number of children]

\item There are \textit{quantitative }ways to speak about the distribution
of a variable. \ For example the \textbf{skewness} of a distribution is
typically calculated as $g_{1}=\frac{1}{N\cdot s^{3}}\overset{N}{\underset{%
i=1}{\sum }}(y_{i}-\overline{y})^{3}\ $This statistic will be zero in a
perfectly symmetric distribution. \ It will be \textit{negative }if
observations below the mean tend to be farther way from it than observations
above the mean/data are skewed left/long left "tail"). \ If will be \textit{%
positive }if data are skewed right (vice versa/long right "tail"). \ You
rarely see this statistic used in political science, but it will pop out
[handout].\newpage
\end{itemize}
\end{itemize}
\end{itemize}

\section{Lecture 2}

[retrieve questionnaires.]

\subsection{The Normal distribution}

\begin{itemize}
\item 
\begin{itemize}
\item One more thing: the Normal distribution. \ 

\begin{itemize}
\item Bell shaped.

\item Unimodal, symmetric.

\item Many variables, empirically, are distributed in a way that
approximates the Normal distribution. \ E.g.: \ height. \ 

\item When this is the case, we can use a rule of thumb called Tchebysheff's
Theorem to describe the distribution:

\begin{itemize}
\item the interval $\overline{y}\pm s$ contains about 68\% of the
observations

\item the interval $\overline{y}\pm 2s$ contains about 95\% of the
observations; and

\item the interval $\overline{y}\pm 3s$ contains about all of the
observations.
\end{itemize}

\item Example: the distribution of the height of American females
approximates the Normal distribution. \ Its mean, $\overline{y}$, is 63.5
and its s.d., $s$ = 3. \ Your cousin's best friend's new roommate is a woman
who is 69.5 inches tall. \ She is taller than what proportion of females in
the American population? \ \{draw on board\}

\item Because the distribution of the mean of independently drawn
observations from any population tends toward the Normal when the sample
size gets large, we think a lot about the Normal. \ This fact also creates
lots of empirical Normal distributions. \ More to come.

\item However, do NOT make the mistake of assuming all distributions are
Normal. \ Not e.g.: internet usage. \ Not e.g.:\ number of children.
\end{itemize}

\item Inevitably, we find ourselves talking about how means and dispersion
change over time and/or comparing groups. \ This begins to invite
multivariate statistics.

\item But for now, that's it for descriptive statistics with regard to one
variable. Pretty simple stuff, really. \ But social scientists spend a lot
of time discussing and displaying descriptive statistics. \ Pay attention to
talks in the department; you'll see. \ 

\item Note what we haven't done: talked about samples versus populations. \
We've taken the data as given, and have resisted making inferences to a
population.

\item That's our next big step. \ But to get there, we need to ascend
through the basics of probability theory.
\end{itemize}
\end{itemize}

\subsection{Probability theory: the logic}

\begin{itemize}
\item We are all familiar with the process of moving from populations to
samples. \ I have a 52-card deck that contains 13 spades. \ What's the
probability of drawing a (sample) spade at random from a perfectly shuffled
deck?

\begin{itemize}
\item Obviously, it's 1/4.

\item We know this because we know the distribution of spades in the
\textquotedblleft population,\textquotedblright\ and we have been precise
about the sampling process.
\end{itemize}

\item Well, we're now going to make the trip in reverse. \ 

\begin{itemize}
\item Now, we know the sample. \ We know its central tendency and its
dispersion.

\item We make precise (and justified) assumptions about the sampling process.

\item These allow us to make very good guesses about the population's
central tendency and dispersion using the sample's central tendency and
dispersion.

\item We need the tools of the population-to-sample process to understand
the sample-to-population process.

\item So let's learn them!
\end{itemize}
\end{itemize}

\subsection{A probabilistic model for an experiment}

\begin{itemize}
\item In probability theory, we use the term \textbf{experiment }to refer to 
\textit{the process by which an observation is made}. \ An \textbf{%
observation }is a quantity of interest:

\begin{itemize}
\item the price of a stock

\item the number of experimental subjects who choose "A" instead of "B"

\item the proportion of Pew survey respondents who approve of Obama's job as
president

\item the proportion of Gallup ""

\item in this usage, experiments don't just happen in labs, but the term is
helpful because it gets us thinking about social processes in the
experimental context.
\end{itemize}

\item Experiments have one or more outcomes called \textbf{events}.

\begin{itemize}
\item Experiment: gubernatorial election in an imaginary country that has
101 voters; majority rule, everyone votes for Candidate A or B.

\item Possible events (list): a. Candidate A wins. \ b. Candidate B wins. \
c. Candidate A wins with 76 votes. \ d. Candidate A wins with 56 votes. \ e.
Candidate A wins in a landslide ($\equiv $ 67 voters or greater). \ How
about some others? \ (In this example, we do not care who voted for whom; we
care only about the aggregate result.)

\item Note that the first event can be decomposed into [how many] other
events?

\item How about the last event?

\item a, b and e are \textbf{compound }events. \ By contrast, c. and d. are 
\textbf{simple }events. \ Let's formalize this. \ Because certain concepts
from set theory will be helpful for expressing relationships amoung events,
we will also associate a distinct point--a \textbf{sample point--}with each
simple event. \ We use $E_{i}$ to refer to the simple event or sample point $%
i$. \ Draw on board: circle $S$ with dots denoting $E$s.

\item The \textbf{sample space }$S$ associated with an experiment is the set
consisting of all possible sample points. \ How many simple events are in $S$
for the hypothetical election ? (102; list on board with a table)

\item In set notation, we write $S=\{E_{1},E_{2},...E_{102}\},$ where $E_{i}$
denotes candidate A's number of votes plus one.

\item This sample space consists of a countable (finite) number of sample
points. \ By definition, it is a \textbf{discrete sample space.}

\item Simple events / sample points are mutually exclusive. \ 

\item By contrast, compound events are sets of sample points. \ The compound
event $A$, \textquotedblleft A wins by landslide\textquotedblright\ occurs
if and only if one of the events $E_{68}...E_{102}$ occurs. \ Thus $%
A=\{E_{68}...E_{102}\}.$

\begin{itemize}
\item A simple event $E_{i}$ is included in compound event $A$ if and only
if $A$ occurs whenever $E_{i}$ occurs.

\item Now we can be more specific about what an \textbf{event }is: it is a
collection of sample points (a subset of $S$).

\item We can also consider $S$ an event in itself. \ It occurs whenever the
experiment occurs.
\end{itemize}
\end{itemize}

\item We're now ready to construct a probabilistic model for an experiment
with a discrete sample space. \ We do so by \textit{assigning a number, }$%
\mathit{P(A),}$\textit{\ to each event }$\mathit{A}$\textit{\ }in $S$. \ Of
course, we can't do this willy-nilly. \ We do so that \textit{three axioms
of probability} hold:

\begin{itemize}
\item Axiom 1: \ $P(A)\geq 0.$ \ 

\item Axiom 2: $P(S)=1.$ \ $S$ occurs every time the experiment is performed.

\item Axiom 3: For any sequence of pairwise mutually exclusive events $%
A_{1},A_{2},A_{3}$,...$A_{n},$ it must be the case that P($A_{1}\cup
A_{2}\cup A_{3}...\cup A_{n})=\overset{n}{\underset{i=1}{\sum }}P(A_{i}).$
That is, the relevative frequency of the union of these mutually exclusive
events (that is, an event made up of these events) equals the sum of their
relative frequences. \ 

\item When we assign numbers $P(A)$ to events in this way, the numbers are
defined as the \textbf{probabilities of }$\mathbf{A}$\textbf{.}

\item There are deep-thought ideas about what probability actually means. \
But the most intuitive way to think about probability of an event $A$ is the
proportion of the time event $A$ would occur if we were to repeat the
experiment, in the exact same way, infinitely many times.
\end{itemize}
\end{itemize}

\subsection{Calculating the Probability of an\ Event of Interest: overview}

\begin{itemize}
\item The three axioms of probability put bounds on the probabilities we
assign to events of interest. \ Now we need to talk about how to assign
probabilities in a rigorous way. \ \ Your text discusses two such methods;
often either can be used to assign probabilities in a given situation. \ The
first is the \textit{sample point }method. \ We won't be discussing this
here in class; but it incorporates many of the techniques you learned
regarding probability in Math Camp. \ The second is the \textit{event
composition} method. \ Because this method illustrates many of the laws of
probability that you will use throughout your quantitative analysis classes,
we will spend a bit of time reviewing this method. \ 
\end{itemize}

\subsection{The event-composition method}

\begin{itemize}
\item In short, the \textit{event-composition} method proceeds by
decomposing and composing event $A$ into \textit{unions} and \textit{%
intersections} of events with conveniently calculated probabilities. \ Four
tools help you do this:

\begin{itemize}
\item 
\begin{itemize}
\item the definitions of conditional probability and independence

\item multiplicative and additive laws

\item the probability of an event and its complement

\item the law of total probability and Bayes'\ Rule
\end{itemize}
\end{itemize}

\item Before proceeding, recall that

\begin{itemize}
\item $A\cap B$ \ "$A$ intersection $B$" is the compound event where \textit{%
both} $A$ and $B$ happen. \ (AND)

\item $A\cup B$ "$A$ union $B$" is the compound event where either $A$ or $B$
happens, or both. \ (OR)

\item \lbrack Diagram on board]
\end{itemize}

\item We require definitions of [put these on left-hand board for easy
referral]:

\begin{itemize}
\item \textbf{conditional probability}: $P(A|B)=\frac{P(A\cap B)}{P(B)}.$

\begin{itemize}
\item Note that this is a \textit{definition }that corresponds with notions
of probability. \ As such it cannot be proven, but it can been shown that
the definition corresponds with commonsense notions of probability. \ It has
the intuitive meaning that $P(A|B)$ is the probability that both $A$ and $B$
occur given as a proportion of the probability that $B$ occurs. \ The same
is true for the following definition...
\end{itemize}

\item \textbf{independence}: \ For two events $A$ and\ $B$, if \textit{any
one} of the following conditions holds$:$

\begin{equation*}
\begin{array}{l}
P(A|B)=P(A) \\ 
P(B|A)=P(B) \\ 
P(A\cap B)=P(A)P(B)%
\end{array}%
\end{equation*}%
, then $A$ and\ $B$ are said to be \textit{independent }events. \ If \textit{%
none} of these three conditions holds, then the events are said to be 
\textit{dependent}.
\end{itemize}

\item We also require two \textit{laws. \ }The first is about the \textit{%
intersection} of two or more events; the second is about the \textit{union}
of two or more events.

\begin{itemize}
\item \textbf{multiplicative law} (\textit{intersection}):

\begin{itemize}
\item $P(A\cap B)=P(A)P(B|A)=P(B)P(A|B)$ \ OR \ 

\item $P(A\cap B)=$ $P(A)P(B)$ if $A,\ B$ are independent events.

\item Note this law follows directly from our definition of conditional
probability.

\item can be extended to $3$ events: \ 

\item 
\begin{eqnarray}
P(A\cap B\cap C) &=&P(\left( A\cap B\right) \cap C)\text{ \ [associative
property]}  \notag \\
&=&P(A\cap B)P(C|A\cap B).\text{[applying the multiplicative law once]} \\
&=&P(A)P(B|A)P(C|A\cap B).\text{[applying the law again]}  \notag
\end{eqnarray}

\item can be extended to $k$ events: \ $P(A_{1}\cap A_{2}\cap ...\cap
A_{k})=P(A_{1})P(A_{2}|A_{1})...P(A_{k}|A_{1}\cap A_{2}\cap ...\cap
A_{k-1}). $
\end{itemize}

\item \textbf{additive law }(\textit{union}):

\begin{itemize}
\item $P(A\cup B)=P(A)+P(B)-P(A\cap B)$, \ 

\item where if $A,B$ mutually exclusive, $P(A\cap B)=0.$

\item why? \ Look at Venn Diagram. \ We're avoiding "double-counting" the
intersection of events $A$ and $B$.:\FRAME{ftbpF}{3.8873in}{2.2286in}{0pt}{}{%
}{Figure}{\special{language "Scientific Word";type "GRAPHIC";display
"USEDEF";valid_file "T";width 3.8873in;height 2.2286in;depth
0pt;original-width 3.5118in;original-height 2.1004in;cropleft "0";croptop
"1";cropright "1";cropbottom "0";tempfilename
'KR1MEP01.wmf';tempfile-properties "XPR";}}
\end{itemize}

\item A proof:%
\begin{eqnarray*}
A\cup B &=&A\cup \left( \symbol{126}A\cap B\right) ,\text{ and }A,\left( 
\symbol{126}A\cap B\right) \text{ are mutually exclusive events. \ Also,} \\
B &=&(\symbol{126}A\cap B)\cup \left( A\cap B\right) ,\text{and }(\symbol{126%
}A\cap B),\left( A\cap B\right) \text{ are mutually exclusive events. Then}
\\
P\left( A\cup B\right) &=&P\left( A\right) +P\left( \symbol{126}A\cap
B\right) \text{ and }P\left( B\right) =P(\symbol{126}A\cap B)+P\left( A\cap
B\right) \text{ [by Axiom 3] and thus} \\
P\left( A\cup B\right) &=&P\left( A\right) +P\left( B\right) -P\left( A\cap
B\right) .\blacksquare
\end{eqnarray*}
\end{itemize}

\item it's helpful to remember that the probability of \textbf{complementary
events }is such that $P(A)=1-P(\sim A).$ \ Proof is easy:%
\begin{eqnarray*}
S &=&A\cup \symbol{126}A\text{ (since }A\text{ is an event in sample space }S%
\text{)} \\
P(S) &=&P(A)+P(\symbol{126}A)\text{ \ (by Axiom 3, since }A,\symbol{126}A%
\text{ mutually exclusive)} \\
1 &=&P(A)+P(\symbol{126}A)\text{ \ (by Axiom 2)} \\
P(A) &=&1-P(\sim A).\blacksquare
\end{eqnarray*}

\item So let's run through an example using the event composition method.

\begin{itemize}
\item I randomly assign a group of 16 students into 3 teams of 6, 5 and 5
students. \ 11 of the students are male.

\item What is $P\left( A\right) =$the probability that the team of six
students (call this "Team 1") is entirely male?

\item First, note that the event \textquotedblleft all members of Team 1 are
male\textquotedblright\ is equivalent to the event \textquotedblleft the
first six assigned students are all male.\textquotedblright

\begin{itemize}
\item Let's decompose this event into simpler events A through F:

\item A: the first student picked is male.

\item B: the second student picked is male.

\item F: the sixth student picked is male.
\end{itemize}

\item The event of interest occurs if and only if all events A through F
occur. \ It is thus the intersection of all these events: $A\cap B\cap C\cap
D\cap E\cap F.$ \ We therefore want to find $P(A\cap B\cap C\cap D\cap E\cap
F).$

\item What should we do?

\begin{itemize}
\item Re-write using multiplicative law:%
\begin{eqnarray*}
P(A_{1}\cap A_{2}\cap ...\cap A_{k})
&=&P(A_{1})P(A_{2}|A_{1})...P(A_{k}|A_{1}\cap A_{2}\cap ...\cap A_{k-1})%
\text{, so} \\
P(A_{1}\cap A_{2}\cap ...\cap A_{6})
&=&P(A_{1})P(A_{2}|A_{1})...P(A_{k}|A_{1}\cap A_{2}\cap ...\cap A_{5})
\end{eqnarray*}%
\ 
\end{itemize}

\item What is the probability that the first student picked is male? \ $%
\frac{11}{16}.$%
\begin{equation*}
P(A_{1}\cap A_{2}\cap ...\cap A_{6})=\frac{11}{16}%
P(A_{2}|A_{1})...P(A_{k}|A_{1}\cap A_{2}\cap ...\cap A_{5})
\end{equation*}

\item Given A1, what's the probability that the second student picked is
male? \ $\frac{10}{15}.$%
\begin{equation*}
P(A_{1}\cap A_{2}\cap ...\cap A_{6})=\frac{11}{16}\cdot \frac{10}{15}\cdot
...P(A_{6}|A_{1}\cap A_{2}\cap ...\cap A_{5})
\end{equation*}

\item And so forth: 
\begin{equation*}
P(A_{1}\cap A_{2}\cap ...\cap A_{6})=\frac{11}{16}\cdot \frac{10}{15}\cdot 
\frac{9}{14}\cdot \frac{8}{13}\cdot \frac{7}{12}\cdot \frac{6}{11}=\frac{3}{%
52}
\end{equation*}

\item In many instances (including exams), we are happy to write this in
factor notation. \ How might we do this? \ 

\begin{itemize}
\item Note that the product of the numerators is $11\cdot 10\cdot ...\cdot
2\cdot 1$ with the final five multiplicands "shaved off." $\ $

\item This is just$\frac{11\cdot 10\cdot ...\cdot 2\cdot 1}{5\cdot 4\cdot
3\cdot 2\cdot 1},$ or$\frac{11!}{5!}.$

\item Similarly, the denominator can be written$\frac{10!}{16!}.$

\item So another way to write this is:%
\begin{equation*}
P(A_{1}\cap A_{2}\cap ...\cap A_{6})=\frac{11!}{5!}\cdot \frac{10!}{16!}
\end{equation*}
\end{itemize}
\end{itemize}
\end{itemize}

\subsection{The law of total probability and Bayes' Rule}

\begin{itemize}
\item finally, the \textit{law of total probability} and \textit{Bayes'\ Rule%
} can be helpful in the event-composition approach to assigning
probabilities to events.

\item recall that we are working with a discrete sample space $S$, composed
of simple events.

\begin{itemize}
\item we can of course also view $S$ as a union of $k$ mutually exclusive
subsets:

\begin{itemize}
\item $S=B_{1}\cup B_{2}\cup ...\cup B_{k};$

\item $B_{i}\cap B_{j}=\varnothing ,\forall i\neq j.$

\item A collection of sets $\{B_{1},B_{2}...B_{k}\}$ such that (1) their
union is equivalent to $S$ and (2) are themselves mutually exclusive is said
to be a \textbf{partition }of $S$.

\item If $A$ is a subset of $S$, it may be \textbf{decomposed }as the union
of its intersections with each of the partitions of $S$ as follows: \ $%
A=(A\cap B_{1})\cup (A\cap B_{2})\cup ...(A\cap B_{k}).$ \ Draw Figure 2.12
\{p. 71\} on board as illustration.

\item \FRAME{ftbpF}{4.782in}{3.3931in}{0pt}{}{}{intersection.tif}{\special%
{language "Scientific Word";type "GRAPHIC";maintain-aspect-ratio
TRUE;display "USEDEF";valid_file "F";width 4.782in;height 3.3931in;depth
0pt;original-width 2.5754in;original-height 1.8196in;cropleft "0";croptop
"1";cropright "1";cropbottom "0";filename 'intersection.tif';file-properties
"XNPEU";}}
\end{itemize}

\item The partitioning and decomposing notions yield the following:

\begin{itemize}
\item If the collection $\{B_{1},B_{2}...B_{k}\}$ is a partition of $S$ such
that $P(B_{i})>0$ for $i=1,2,...k$, then the \textbf{Law of\ Total
Probability }states:%
\begin{equation*}
P(A)=\sum_{i=1}^{k}P(A|B_{i})P(B_{i})
\end{equation*}

\item Proof \{brief\}:

\begin{itemize}
\item If $i\neq j,$ the intersections $A\cap B_{i}$ and $A\cap B_{j}$ do not
overlap. \ Put formally, their intersection is the empty set, because

\begin{eqnarray*}
(A\cap B_{i})\cap (A\cap B_{j}) &=&A\cap (B_{i}\cap B_{j})\text{ \ \
[distributive law]} \\
&=&A\cap \oslash \\
&=&\oslash .
\end{eqnarray*}%
The events $A\cap B_{i}$ and $A\cap B_{j}$ are thus mutually exclusive.

\item Because $A$ is the union of all these mutually exclusive events, we
can write 
\begin{eqnarray*}
P(A) &=&P(A\cap B_{1})+P(A\cap B_{2})+...+P(A\cap B_{k})\text{ \ \ \ \ \
[additive law]} \\
&=&P(A|B_{1})P(B_{1})+P(A|B_{2})P(B_{2})+...P(A|B_{k})P(B_{k})\text{
[muliplicative law]} \\
&=&\sum_{i=1}^{k}P(A|B_{i})P(B_{i})\quad \blacksquare .
\end{eqnarray*}

\item Why do we care? \ Because there are lots of instances where it's
easier to calculate $P(A|B_{i})$ than $P(A)$. \ In other cases, we
intrinsically care about $P(A|B_{i})$. I'll show you this in a minute, but
first let's derive one additional important result.
\end{itemize}

\item If the collection $\{B_{1},B_{2}...B_{k}\}$ is a partition of $S$ such
that $P(B_{i})>0$ for $i=1,2,...k$, then

\begin{eqnarray*}
P(B_{j}|A) &=&\frac{P(A\cap B_{j})}{P(A)}\text{ \ [definition of conditional
probability]} \\
&=&\frac{P(A|B_{j})P(B_{j})}{\sum_{i=1}^{k}P(A|B_{i})P(B_{i})}\text{ \
[definition again (num), law of total probability (denom)]}
\end{eqnarray*}

\item This is \textbf{Bayes'\ Rule}.

\item Let's write this in a simple case, where there are only two partitions
in the sample space:

\FRAME{itbphF}{4.0938in}{2.4641in}{0in}{}{}{Figure}{\special{language
"Scientific Word";type "GRAPHIC";maintain-aspect-ratio TRUE;display
"USEDEF";valid_file "T";width 4.0938in;height 2.4641in;depth
0in;original-width 5.76in;original-height 3.4562in;cropleft "0";croptop
"1";cropright "1";cropbottom "0";tempfilename
'KQN80Y00.wmf';tempfile-properties "XPR";}}

\item We then have:%
\begin{equation*}
P(B_{1}|A)=\frac{P(A|B_{1})P(B_{1})}{P(A|B_{1})P(B_{1})+P(A|B_{2})P(B_{2})}
\end{equation*}

\item or more simply (writing $B_{1}$ as simply $B$):%
\begin{equation*}
P(B|A)=\frac{P(A|B)P(B)}{P(A|B)P(B)+P(A|\symbol{126}B)P(\symbol{126}B)}
\end{equation*}

\item So classic example (2.124, page 73):%
\begin{eqnarray*}
P(Dem|Favor) &=&\frac{P(Favor|Dem)P(Dem)}{%
P(Favor|Dem)P(Dem)+P(Favor|REP)P(REP)} \\
&=&\frac{\left( .7\right) \left( .6\right) }{\left( .7\right) \left(
.6\right) +\left( .3\right) \left( .4\right) } \\
&=&7/9 \\
&=&.\overline{7}
\end{eqnarray*}

\item Note our process of determining the probability $P(Dem|Favor).$ \ We
used the definition of conditional probability (which undergirds Bayes'
Rule) to decompose the set (DEM $\cap $ FAVOR).
\end{itemize}

\item Another classic example (check if there's time) \ Ex. 2.125, p. 73.

\begin{itemize}
\item Begin by having students write $P\left( Disease|Pos\right) $ according
to the definition of Bayes'\ Rule.

\item What information do we need to know?

\begin{itemize}
\item $P\left( Pos|Disease\right) =.90$

\item $P\left( \symbol{126}Pos|\symbol{126}Disease\right) =.90$

\item $P\left( Disease\right) =.01$
\end{itemize}

\item And so:

\begin{eqnarray*}
P\left( Disease|Pos\right) &=&\frac{P(Pos|Disease)P(Disease)}{%
P(Pos|Disease)P(Disease)+P(Pos|\symbol{126}Disease)P(\symbol{126}Disease)} \\
&=&\frac{(.9)(.01)}{(.9)(.01)+(.10)(.99)} \\
&=&\frac{.009}{.009+.099}=\frac{1}{12}
\end{eqnarray*}

\item Put this in \ perspective:

\begin{itemize}
\item What is the probability of a false positive? $P\left( Pos|\symbol{126}%
Disease\right) =.10=\frac{1}{10}$

\item What is the probability of a false negative? \ $P\left( \symbol{126}%
Pos|Disease\right) =.10=\frac{1}{10}$
\end{itemize}

\item What affects the reliability of this test?\newpage
\end{itemize}
\end{itemize}
\end{itemize}

\section{Lecture 3}

\begin{itemize}
\item We've now provided an intellectual foundation for our understanding of
probability. \ We've conceived of probabalistic events in the context of an 
\textbf{experiment }whose results are--in their most basic form--\textbf{%
simple events}. \ The set of all possible events for a given experiment is
its \textbf{sample space. \ }

\begin{itemize}
\item We then specified 3 axioms governing the assignment of probabilities
to these events;

\item And then discussed four tools by decompose and compose events of
interest into events with conveniently calculated probabilities.
\end{itemize}

\item Before moving on, I want to formalize one additional tool that we use
without thinking about it--which is \textit{assigning probabilities in a
sample space consisting of equiprobable events}:

\begin{itemize}
\item Last time we had the example of assigning 16 students into 3 teams of
6, 5 and 5 students, with 11 of the students male.

\begin{itemize}
\item As you may recall, we began by considering the probability that the
first student picked is male. \ What is it? \ $\frac{11}{16}.$ \ Why?
\end{itemize}

\item Here's how to formalize our intuition. \ 

\item Consider a sample space $S$ consisting of $n$ \textit{equiprobable}
simple events $E$ such that $P\left( E_{1}\right) =P\left( E_{2}\right)
=...P\left( E_{n}\right) .$

\begin{itemize}
\item Recall that we define an $A$ as a subset of $S$ -- that is, some
subset of sample points$\ $that make up $S$.\ 

\item Then $P(A)=\frac{|A|}{n}$,.where the notation $|A|$ stands for the
number of elements in the set $A$ -- which in this context is the number of
sample points in $A.$
\end{itemize}

\item Can you see how we get this from the axioms of probability?

\begin{itemize}
\item First let's show that for all events $E$ to be equiprobable, it must
be that%
\begin{equation*}
P\left( E_{i}\right) =\frac{1}{n}\text{ for }i=\{1,2...n\}
\end{equation*}

\item This is because

\item $P(S)=P\left( E_{1}\cup E_{2}\cup ...\cup E_{n}\right) =P\left(
E_{1}\right) +P\left( E_{2}\right) +...+P\left( E_{n}\right) =1$ \ [by Axiom
3 and then Axiom 2]

\item $P\left( E_{1}\right) +P\left( E_{1}\right) +...+P\left( E_{1}\right)
=nP\left( E_{1}\right) =1$ \ [definition of equiprobability]

\item $P\left( E_{1}\right) =\frac{1}{n}$; and similarly $P\left(
E_{i}\right) =\frac{1}{n}$ for $i=\{1,2...n\}$

\item Finally, since%
\begin{eqnarray*}
P\left( A\right) &=&\sum_{E_{i}\in A}P\left( E_{i}\right) \ \text{[Axiom 3
again]} \\
P\left( A\right) &=&\sum_{E_{i}\in A}\frac{1}{n}=\frac{|A|}{n}\blacksquare .
\end{eqnarray*}

\item \lbrack For reference:]

\begin{itemize}
\item Axiom 2: $P(S)=1.$ \ $S$ occurs every time the experiment is performed.

\item Axiom 3: For any sequence of pairwise mutually exclusive events $%
A_{1},A_{2},A_{3}$,...$A_{n},$ it must be the case that P($A_{1}\cup
A_{2}\cup A_{3}...\cup A_{n})=\overset{n}{\underset{i=1}{\sum }}P(A_{i}).$
\end{itemize}
\end{itemize}

\item The point here is that even our intuitive basic notion of assigning
probabilities to events is derived from the axioms of probability in logical
ways.
\end{itemize}
\end{itemize}

\subsection{The notion of a random variable}

\begin{itemize}
\item Now let's discuss a particular kind of experiment: one in which the
events of interest are numerical--that is, they are identfied in a
meaningful way by numbers. \ 

\begin{itemize}
\item Numerical events of interest to political scientists might be the
number of seats the Republican Party will hold in the House of\
Representatives after a midterm election, or the number of signatories to a
treaty.

\item In the language of an experiment, we assign a \textit{real number} to
each point in the sample space.

\item Call this number the variable $Y$.

\begin{itemize}
\item Think back: what's a \textit{variable}? \ [Make them look this up]: \
Recall that a variable is a logical grouping of attributes. \ Variables take
on values that are exhaustive and mutually exclusive. \ 
\end{itemize}

\item Each sample point can only take on one value of $Y$. \ But the same
value may be assigned to multiple sample points. \ 

\begin{itemize}
\item \lbrack Draw a circle with $S$, the sample space. \ Draw another
circle, $Y$, the variable. \ Draw lines from several points in $S$ to one
point in $Y$.]
\end{itemize}

\item What do we call such a mapping? \ A \textit{function}. Thus the
variable $Y$ is a \textit{function} of the sample points in $S$.

\item Recall that a function is a mathematical relation assigning each
element of one set (the source) to one and only one element of another set
(the target). \ 

\item In this case, the function's source is $S$ and its target is $Y$. \ \
We can write $f:S\rightarrow Y.$ \ This function (and by extension, $Y)$ is
defined as a \textbf{random variable}. \ Keep this in mind: whenever we talk
about a random variable, \textit{we are really talking about a function}
that maps each simple event in a sample space $S$ to a (meaningful) number.

\item We write a random variable with a capital letter: $Y$.

\begin{itemize}
\item Consequently, we write \textquotedblleft the probability that $Y=0$%
\textquotedblright\ as $P(Y=0)$, and so forth. \ 

\item Typically, we refer to any observed or hypothetical value of $Y$ with
a lowercase letter denoting the value. \ So we write $P(Y=y)$, for example.

\item We will still talk about the event of interest $A$, but we will give
it a number, $a$. The event $A$ thus $\equiv $\{all sample points such that $%
Y$ $=a$\}.
\end{itemize}

\item More about random variables very shortly. \ But first:
\end{itemize}
\end{itemize}

\subsection{Quick detour: The notion of a random \textit{sample}}

\begin{itemize}
\item In our subsequent discussions of probability, we will often invoke the
notion of a \textbf{random sample}. \ Let's take a moment to place this idea
in the language of our probabilistic model of an experiment.

\item In this framework:

\begin{itemize}
\item our experiment is the drawing of a \textbf{sample }(the units selected
for analysis) from a \textbf{population} (the group of units about which we
wish to make inferences)

\item the \textbf{design }of our experiment is the method of sampling.

\begin{itemize}
\item one big decision we make, for example, is whether to sample \textit{%
with replacement }(units selected for the sample are \textquotedblleft
placed back into the population\textquotedblright\ and may therefore again
be sampled) or \textit{without replacement} (units selected are set aside
and cannot be selected again). \ Keep these terms in mind; we'll revisit
them later.
\end{itemize}

\item the most common way to draw a sample is called \textit{random
sampling. \ }Let $N$ and $n$ represent the numbers of elements in the
population and sample, respectively. \ In this context, a total of [how
many?] $\binom{N}{n}$ [the binomial coefficient] different samples without
replacement (how many is that? $\frac{N!}{n!\left( N-n\right) !}$) may be
drawn. \ If sampling is conducted in such a way that each of these samples
has an equal probability of being selected, then we have engaged in \textit{%
random sampling}, and the result is said to be a \textit{random sample}.
\end{itemize}
\end{itemize}

\subsection{Back to random variables: probability distributions}

\begin{itemize}
\item We'll begin our discussion of random variables by focusing on those
that are \textbf{discrete}. \ A random variable $Y$ is discrete if it can
take on only a finite or countably infinite (a one-to-one correspondence
with the integers) number of distinct values.

\item In making inferences from samples to populations,

\begin{itemize}
\item we need to know the probability of having observed a particular event.
\ 

\item events of interest are frequently numerical events that correspond to
values $y$ of discrete random variables $Y$.

\begin{itemize}
\item An example of such an event might be that 45 out of 100 people we
surveyed said they intended to vote for Mitt Romney.
\end{itemize}

\item In other words, we need to know $P(Y=y)$ for all the values the RV $Y$
can take on.

\item This collection of probabilities is called the \textbf{probability
distribution }of the RV $Y$. \ 

\item As an example, consider the experiment: roll a pair of six-sided dice
and record the sum of the pips on their faces.

\item The sample space $S$ consists of 36 simple events. \ 

\begin{itemize}
\item Draw a square, label it S, divide into grid of 6 x 6.
\end{itemize}

\item Define the random variable $Y$ = the sum of the pips appearing on the
faces of a random roll of a pair of six-sided dice.

\begin{itemize}
\item Draw another square, label it Y, and map a few of the sample points in
S to Y.

\item Formally, $P(Y=y)=\sum_{E_{i}:Y(E_{i})=y}P(E_{i}).$ \ The probability $%
Y=y$ is defined as the sum of the probabilities of the sample points in $S$
assigned the value $y$. \ Sometimes we write this $p(y)$.

\item It its most primitive form, $Y$'s probability distribution is a 
\textit{table }in which each $y$ is listed alongside $P(Y=y)$ \ \ But it may
also be a \textit{formula} mapping each $y$ to its $P(Y=y),$ or a \textit{%
graph} doing the same thing.

\item Table: (entitled "The Probability Distribution of the RV $Y$")
\end{itemize}
\end{itemize}
\end{itemize}

\begin{center}
\begin{tabular}{ccl}
\hline\hline
$y$ & \# of sample points & $P(Y=y)$ \\ \hline
2 & 1 & \multicolumn{1}{c}{$\frac{1}{36}$} \\ 
3 & 2 & \multicolumn{1}{c}{$\frac{2}{36}$} \\ 
4 & 3 & \multicolumn{1}{c}{$\frac{3}{36}$} \\ 
5 & 4 & \multicolumn{1}{c}{$\frac{4}{36}$} \\ 
6 & 5 & \multicolumn{1}{c}{$\frac{5}{36}$} \\ 
7 & 6 & \multicolumn{1}{c}{$\frac{6}{36}$} \\ 
8 & 5 & \multicolumn{1}{c}{$\frac{5}{36}$} \\ 
9 & 4 & \multicolumn{1}{c}{$\frac{4}{36}$} \\ 
10 & 3 & \multicolumn{1}{c}{$\frac{3}{36}$} \\ 
11 & 2 & \multicolumn{1}{c}{$\frac{2}{36}$} \\ 
12 & 1 & \multicolumn{1}{c}{$\frac{1}{36}$} \\ \hline
\multicolumn{1}{l}{\textbf{totals}} & \textbf{36} & \multicolumn{1}{c}{%
\textbf{1}} \\ \hline\hline
\end{tabular}
\end{center}

\begin{itemize}
\item 
\begin{itemize}
\item 
\begin{itemize}
\item Graph: [draw on board]

\item A function (here's one way, there may be others/better). \ Here, $p(y)$
is known as a \textbf{probability function}. 
\begin{equation*}
P(Y=y)=p(y)=\ \frac{6-|7-y|}{36},y=\{1,2,...6\}.
\end{equation*}
\end{itemize}

\item The probability function of a discrete random variable is also called
its \textbf{probability mass function, or PMF. \ }The "mass of a random
variable at $y$" is the PMF evaluated at $y$, or $p\left( y\right) .$

\item The probabilities associated with distinct values of $y$ sum to 1, and
in fact they always do, as the second axiom of probability requires.
\end{itemize}
\end{itemize}

\subsection{Random variables: expected values}

\begin{itemize}
\item The probability distribution of an RV is a \textit{theoretical model }%
for the empirical distribution of data associated with a real population. \
If we were to roll a pair of dice over and over again, recording the sum of
the faces each time, the empirical distribution would look very much like
the theoretical probability distribution of $Y$ we just specified.

\item As we do with empirical data, we can describe a random variable by
talking about its central tendency and dispersion. \ (At what level of
measurement are we implicitly working here? \ Interval or higher.) \ So we
will be concerned about the mean and variance of a random variable.

\item We specify and manipulate formulas describing random variables using
the \textit{expectations operator}, which is defined as follows:

\begin{itemize}
\item Let $Y$ be a discrete RV with the probability function $p(y)$. \ The 
\textbf{expected value }of $Y$ is written $E(Y),$ and defined to be:%
\begin{equation*}
E(Y)\equiv \sum_{y}yp(y).
\end{equation*}

\item Note that this is each possible value of $Y$ times the probability
that $Y\ $takes on the value $y,$ i.e., $p(y),$ summed up over all $y$.
[Calculate this value for the two dice example.]

\item So the expectations operator tells us to consider all the possible
values of a RV, multiply each of these values by their probability of
occurrence, and to sum up these products.

\item The expected value is the way we talk about the central tendency of a
random variable with a theoretical probability distribution. \ It is
equivalent to the idea of the mean of an empirical frequency distribution.

\item Now we take what is a very powerful step as we move into inferential
statistics. \ Recall that the probability distribution of an RV is a \textit{%
theoretical model }for the empirical distribution of data associated with a
real population. \ If this theoretical model is accurate, then $E(Y)=\mu $,
the \textit{population mean}.%
\begin{equation*}
E(Y)\equiv \sum_{y}yp(y)=\mu .
\end{equation*}

\item Here, $\mu $ is a \textit{parameter}: a characteristic of the
distribution of $Y$ in the population that we never actually observe but
about which we are often keenly interested. \ It is the first of many such
parameters we will encounter in this class.
\end{itemize}

\item We are often interested in the expected value of \textit{functions of
random variables. \ }(You'll see why in a minute.) \ Consider as usual the
discrete RV, $Y$ with probability function $p(y)$. \ Now consider any
real-valued function of $Y$, $g(Y).$ Then the expected value of $g(Y)$is
given by:%
\begin{equation*}
E[g(Y)]=\sum_{y}g(y)p(y).
\end{equation*}

\item That is, the expected value of a function of a RV is given by:

\begin{itemize}
\item evaluating the function for each value of $Y$

\item multiplying it by the probability that $Y=y$

\item and summing up over all possible values of $Y.$
\end{itemize}

\item Now this is not a definition, but rather can be proven based upon the
definition of $E\left( Y\right) .$Proof: \ 

\begin{itemize}
\item Consider a discrete RV $Y$ taking on a finite number, $n$, of values $%
y_{1},y_{2},...y_{n}.$

\item Suppose $g(y)$ takes on $m$ different values $g_{1},g_{2}...g_{m},m%
\leq n.$

\begin{itemize}
\item We're being this picky because it's of course possible that $g\left(
{}\right) $ is not one-to-one; that is, it may take on the same value for
multiple $y$'s.
\end{itemize}

\item Note that $g(Y)$ is itself a random variable (by definition of RV). \
It is a function mapping the sample space\ $Y$ to the reals.

\item So we can define a new probability function, $p$-star, for $g.$ \ 

\item Write the probability that $g$ takes on a value $g_{i}$ as%
\begin{eqnarray*}
p^{\ast }(g_{i}) &=&P[g(Y)=g_{i}] \\
&=&\sum_{y_{j}:g(y_{j})=g_{i}}p(y_{j})\text{, for }i=1,2,...m\text{ \ }
\end{eqnarray*}

\item where $y_{j}:g(y_{j})=g_{i}$ means "all $y_{j}$ such that $g$ equals $%
g_{i}$ when evaluated at $y_{j}.$"

\item Now the definition of expected value tells us that:%
\begin{equation*}
E[g(Y)]=\sum_{i=1}^{m}g_{i}\,p^{\ast }(g_{i})
\end{equation*}

\item (Note that we're doing the same thing as before:

\begin{itemize}
\item Taking each possible value of $g,$ that is all the $g_{i}$'s

\item Multiplying it by its associated probability (which we've defined in
this case as $\,p^{\ast }(g_{i})$)

\item And summing up these products.)
\end{itemize}

\item Moving on with proof:%
\begin{eqnarray*}
&=&\sum_{i=1}^{m}g_{i}\left\{ \sum_{y_{j}:g(y_{j})=g_{i}}p(y_{j})\right\} 
\text{ [substituting]} \\
&=&\sum_{i=1}^{m}\left\{ \sum_{y_{j}:g(y_{j})=g_{i}}g_{i}\,p(y_{j})\right\} 
\text{ [can move }g_{i}\text{ inside because of nested summations] } \\
&=&\sum_{j=1}^{n}g(y_{j})p(y_{j}).\text{[since }g_{i}=g(y_{j})\text{ for any 
}y_{j};\text{ we change index to signify we're} \\
&&\text{now interested in all }n\text{ values that }Y\text{ can take on]} \\
&=&\sum_{y}g(y)p(y).\text{ \ [writing more simply]}
\end{eqnarray*}

\item Again, the intuition behind $E[g(Y)]=\sum_{y}g(y)p(y)$ is
straightforward. \ We proceed by:

\begin{itemize}
\item evaluating the function for each value of $Y$

\item multiplying it by the probability that $Y=y$

\item and summing up over all possible values of $Y.$
\end{itemize}
\end{itemize}

\item We can now define the variance of a random variable $Y$. \ Recall
before that we defined the variance of an empirical variable as $s^{2}=\frac{%
1}{N}\overset{N}{\underset{i=1}{\sum }}(y_{i}-\overline{y})^{2}.$ \ 

\begin{itemize}
\item Well, the variance of a random variable is how it varies about its
mean:%
\begin{equation*}
VAR(Y)\equiv E[(Y-E\left( Y\right) )^{2}]
\end{equation*}

\item And if the RV $Y$ accurately describes the population distribution then%
\begin{equation*}
VAR(Y)=E[(Y-\mu )^{2}].
\end{equation*}

\item The standard deviation of $Y$ is $+\sqrt{VAR(Y)}.$

\item And (again assuming that our RV variable is an accurate theoretical
model of the world) then $VAR(Y)=\sigma ^{2}$, the population variance, with 
$\sqrt[+]{\sigma ^{2}}=\sigma $ the population standard deviation.

\item Do Example 3.2 (p. 94) on the board.
\end{itemize}
\end{itemize}

\subsection{Some helpful results}

\begin{itemize}
\item Go over \textquotedblleft Some helpful results about the math of
expectations for discrete RVs\textquotedblright\ handout. \ Proofs:

\item To show $E(c)=c$:

\begin{itemize}
\item Consider the function $g(Y)\equiv c.$ By the expectation of a function
of a random variable theorem, $E(c)=\sum_{y}cp(y)=c\sum_{y}p(y).$ \ But by
Axiom 2, $\sum_{y}p(y)=1.$ \ Hence $E(c)=c(1)=c.$ \ $\blacksquare .$
\end{itemize}

\item To show $E[cg(Y)]=cE[g(Y)]$:

\begin{itemize}
\item By the expectation of a function of a random variable theorem,%
\begin{eqnarray*}
E[cg(Y)] &=&\sum_{y}cg(y)p(y) \\
&=&c\sum_{y}g(y)p(y) \\
&=&cE[g(Y)]\quad \blacksquare .
\end{eqnarray*}
\end{itemize}

\item To show that we can distribute expectations, consider the case where $%
k=2$:%
\begin{eqnarray*}
E[g_{1}(Y)+g_{2}(Y)] &=&\sum_{y}[g_{1}(y)+g_{2}(y)]p(y)\text{ \ \ [since }%
g_{1}(Y)+g_{2}(Y)\text{ is a function of }Y\text{]} \\
&=&\sum_{y}[g_{1}(y)p(y)]+\sum_{y}[g_{2}(y)p(y)]\text{ \ \ \ [distributing
summations]} \\
&=&E[g_{1}(Y)]+E[g_{2}(Y)]\text{ \ \ \ [by definition of the expectations
operator]} \\
&&\blacksquare .
\end{eqnarray*}

\item All of these results help us re-write the formula for the population
variance in a very helpful way (proof is on handout).

\item \lbrack Emphasize: we always treat parameters as constants when
applying the expectations operator. \ They are invariant.]
\end{itemize}

\subsection{Three theoretical probability models}

\begin{itemize}
\item As illustrations, we will discuss three of the standard discrete
probability distributions:

\begin{itemize}
\item the \textbf{Bernoulli}

\item the \textbf{binomial}

\item the \textbf{Poisson}
\end{itemize}
\end{itemize}

\subsection{The Bernoulli}

\begin{itemize}
\item A Bernoulli experiment is the observation of an experiment consisting
of one trial with two outcomes: zero or one.

\item Thus the Bernoulli random variable $Y$ takes on the values $\left\{
0,1\right\} .$

\item E.g.'s:

\begin{itemize}
\item a coin toss

\item whether an individual approves or disapproves of Barack Obama's
performance as president

\item whether a country signs a treaty
\end{itemize}

\item A Bernoulli random variable is characterized by one parameter: $\pi $,
the probability of success.

\item Thus its probability distribution is%
\begin{eqnarray*}
p\left( y=1\right) &=&\pi \\
p\left( y=0\right) &=&1-\pi
\end{eqnarray*}

\item Sometimes it's convenient to write the probability (mass) function%
\begin{equation*}
p\left( y\right) =\pi ^{y}\left( 1-\pi \right) ^{\left( 1-y\right) }
\end{equation*}

\begin{itemize}
\item Think of the exponents as "switches" which turn on and off (i.e. equal
to one) depending on the value of $y$ at which the function is being
evaluated.
\end{itemize}

\item For a little practice, let's show that%
\begin{eqnarray*}
E\left( Y\right) &=&\pi . \\
E\left( Y\right) &=&\sum_{y}p\left( y\right) y \\
&=&p\left( y=0\right) \left( 0\right) +p\left( y=1\right) \left( 1\right) \\
&=&\pi .
\end{eqnarray*}

\item How about 
\begin{eqnarray*}
VAR\left( Y\right) &=&\pi \left( 1-\pi \right) \\
VAR\left( Y\right) &\equiv &E[(Y-)^{2}] \\
&=&E(Y^{2})-\left[ E\left( Y\right) \right] ^{2}\text{ [from handout; see
how it's already helpful?]}
\end{eqnarray*}

\item Note that 
\begin{eqnarray*}
E(Y^{2}) &=&\sum_{y}y^{2}p(y=Y) \\
&=&0^{2}\left( 1-\pi \right) +1^{2}\pi \\
&=&\pi
\end{eqnarray*}

\item Now substitute%
\begin{eqnarray*}
VAR\left( Y\right) &=&E(Y^{2})-\left[ E\left( Y\right) \right] ^{2} \\
&=&\pi -\pi ^{2}=\pi \left( 1-\pi \right) .
\end{eqnarray*}

\item Let's talk a bit more about \textbf{parameters}. \ Together with the
probability function, parameters determine the shape, location and spread of
the distribution. \ A \textbf{location parameter }specifies the location in
the center of the distribution. As we change a location parameter\ we shift
the PMF to the left or right. \ Its empirical referent is, of course, the
mean. \ A \textbf{scale parameter }specifies the spread (or scale) of the
distribution around its central location. \ Its empirical referent is the
standard deviation. \ What's notable about the Bernoulli (and, as you will
see, the Binomial and the Poisson) is that they only have location
parameters; their spread and their location are tied together.
\end{itemize}

\subsection{The Binomial \ }

\begin{itemize}
\item A binomial experiment is the observation of a sequence of identical
and independent Bernoulli trials. \ Specifically:

\begin{itemize}
\item A fixed number, $n,$ of trials with one of two outcomes: success $S$
or failure $F$.

\item As with the Bernoulli, the probability of success on any single trial
is $\pi $. \ The probability of failure is thus $1-\pi .$

\item The trials are independent.

\item The random variable of interest is $Y$, the \# of successes observed
during the $n$ trials.
\end{itemize}

\item E.g.'s:

\begin{itemize}
\item \# of heads observed in a certain \# of coin tosses

\item \# of people approving of Barack Obama out of a certain \# of people

\item \# of countries signing a treaty out of $n$ eligible countries
\end{itemize}

\item Let's find the binomial probability distribution. \ Recall that a
probability distribution (table, graph, formula) tells us with what
probability our RV of interest, $Y$, takes on all possible values $y$, i.e. $%
P(Y=y)$, or simply $p(y).$

\item In the context of a probability model, we consider the event "the
number of successes equals $y,"$ or $Y=y,$ to be our \textquotedblleft event
of interest.\textquotedblright\ 

\item Let's assign a probability $p(y)$ to each of these events.

\item One such event might be \ \ $SSFSFFFSFS...FS$, where there are $n$
such positions.

\item If we were to count the $S$'s and $F$'s, we might see%
\begin{equation*}
S_{1}SSSS...SSS_{y}\qquad F_{1}FF....FF_{n-y}
\end{equation*}

\item In this context, the number of $S$'s equals what? \ $y$. \ And so the
number of $F$'s equals what? \ $n-y.$

\item Think about our laws of probability. \ The event of interest we have
witnessed is the \textbf{intersection }of $n$ simple events: $S_{1}\cap
S_{2}\cap ...\cap S_{y}\cap F_{1}\cap F_{2}...\cap F_{n-y}.$ \ These are 
\textbf{independent }events (by definition of the binomial experiment). \ So 
$P(S_{1}\cap S_{2}\cap ...\cap S_{y}\cap F_{1}\cap F_{2}...\cap F_{n-y})$
simply equals $P(S_{1})P(S_{2})...P(S_{y})P(F_{1})P(F_{2})P(F_{n-y}).$

\item So what's that? \ It's $\pi ^{y}\left( 1-\pi \right) ^{n-y}.$

\item But this is \textit{not }the probability of seeing the event $Y=y.$ \
Why? \ Because the event $Y=y$ could happen in many more ways than the order
in which we saw it. \ How many different ways are there to order $y$ $S$'s
and $n-y\quad F$'s? \ It's the number of ways of choosing $y$ elements from
a total of $n$ elements. Or "$n$, choose $y$." \ From math camp, you'll
remember that this is equal to (ha!) the binomial coefficient $\binom{n}{y},$%
or $\frac{n!}{y!(n-y)!}.$

\item Thus the probability of observing $Y=y$ is 
\begin{eqnarray*}
p(y) &=&\frac{n!}{y!(n-y)!}\pi ^{y}(1-\pi )^{n-y}. \\
&& \\
&&\text{This can be written as }Binomial(\pi ;n)
\end{eqnarray*}

\item In the parlance of probability distributions, the binomial has two 
\textbf{parameters}: $\pi $ and $n$.

\item So (very quickly). \ 9 students in the class, 5 male. \ I pick six
students at random with replacement. \ What's the chance that I pick the
same number of males and females?

\begin{itemize}
\item Call 'success' a female. \ $\pi =\frac{4}{9}.$

\item We have \# of trials $n=6$. \ We have \# of successes as $y=3.$

\item We thus evaluate the probability distribution for $Bi(\frac{4}{9};6)$
at $y=3:$%
\begin{eqnarray*}
p(Y &=&3)=\frac{6!}{3!(6-3)!}\left( \frac{4}{9}\right) ^{3}\left( 1-\frac{4}{%
9}\right) ^{6-3} \\
&\approx &.30
\end{eqnarray*}
\end{itemize}

\item Let's say I turned this question around a bit. \ I draw six students
at random (with replacement). \ On average, how many females will I pick? \
And how much will this number vary over repeated draws of six?

\item Answering questions like this requires a formula for the expected
value and the variance of the Bernoulli random variable $Y$.

\item And in fact, we can prove that if $Y$ is a binomial random variable, 
\begin{equation*}
E(Y)=n\pi =\mu \text{ \ and \ \ }VAR(Y)=n\pi (1-\pi )=\mu (1-\pi )=\sigma
^{2}.
\end{equation*}

\item We'll omit the proofs in class because we need to move on. \ But the
proofs (pp. 107-108) are informative.

\item Thus $E(Y|p=\frac{4}{9};n=6)=6\times \frac{4}{9}=2.\overline{6}.$ \ $%
VAR(Y)\approx 2.7\left( 1-\frac{4}{9}\right) =1.5.$ \ The s.d. is $\sqrt{1.5}%
\approx 1.2$. \ The typical \# of females I will draw in a sample of 6 is
2.7, but the typical value will fall 1.2 females away from 2.7. \ 

\item In practice, Binomials are not conveniently calculated. \ We typically
use statistical software to calculate the probability that the Binomial
random variable $Y$ takes on some value $y.$ \ In Stata, we type \texttt{.di
binomialp}$(n,y,\pi )$ to get $P\left( Y=y\right) .$\newpage 
\end{itemize}

\section{Lecture 4}

\subsection{The Poisson}

\begin{itemize}
\item A final discrete probability distribution we will examine is the
Poisson.

\item A Poisson experiment is the observation of a \textit{count of events}
that occur in an \textit{interval, broadly defined} -- a given space, time
period, or any other dimension. \ It is particularly helpful when modifying
relatively \textit{infrequent }events (as more frequent events can be
modeled with more generic distributions).

\begin{itemize}
\item Environmental laws per Congressional session.

\item Errors per page.

\item Government shutdowns per decade.

\item Homeless shelters per census tract.
\end{itemize}

\item We discuss the Poisson after the Binomial because it can actually be
conceived of the Binomial experiment as the number of trials approaches
infinity. \ (What? \ Let's consider:)

\begin{itemize}
\item Split the interval into $n$ subintervals, each so small that at most
one event could occur in it. \ That is, in each subinterval a Bernoulli
trial takes place:%
\begin{eqnarray*}
P\left( y=1\right) &=&\pi \\
P\left( y=0\right) &=&1-\pi \\
\text{AND }P\left( y>1\right) &=&0.
\end{eqnarray*}

\item In this subinterval, 
\begin{equation*}
p\left( y\right) =\pi ^{y}\left( 1-\pi \right) ^{\left( 1-y\right) }\text{
[Bernoulli]}
\end{equation*}

\item And if we have $n$ subintervals, then%
\begin{equation*}
p(y)=\frac{n!}{y!(n-y)!}\pi ^{y}(1-\pi )^{n-y}\text{ \ [Binomial]}
\end{equation*}

\item How many such subintervals are needed? \ Who knows. \ But we can
handle this problem by making the subintervals infinitely small by taking
the limit of the Binomial probability function as $n$ goes to infinity. \
Our parameter of interest--the number of successes over the interval--is $%
n\pi .$ \ We call this parameter $\lambda =n\pi .$ \ So:%
\begin{eqnarray*}
&&\lim_{n\longrightarrow \infty }\frac{n!}{y!(n-y)!}\pi ^{y}(1-\pi )^{n-y} \\
&=&\lim_{n\longrightarrow \infty }\frac{n!}{y!(n-y)!}\left( \frac{\lambda }{n%
}\right) ^{y}\left( 1-\frac{\lambda }{n}\right) ^{n-y} \\
&=&\lim_{n\longrightarrow \infty }\frac{n!}{y!(n-y)!}\frac{\lambda ^{y}}{%
n^{y}}\left( 1-\frac{\lambda }{n}\right) ^{n}\left( 1-\frac{\lambda }{n}%
\right) ^{-y}
\end{eqnarray*}

\item Noting that by definition of $e,$%
\begin{equation*}
\lim_{n\longrightarrow \infty }\left( 1-\frac{\lambda }{n}\right)
^{n}=e^{-\lambda }
\end{equation*}

\item And so%
\begin{eqnarray*}
&=&\lim_{n\longrightarrow \infty }\frac{n!}{y!(n-y)!}\frac{\lambda ^{y}}{%
n^{y}}e^{-\lambda }\left( 1\right) \\
&=&\frac{e^{-\lambda }\lambda ^{y}}{y!}\lim_{n\longrightarrow \infty }\frac{%
n!}{(n-y)!n^{y}}\text{ \ [pulling out everything not related to }n\text{]} \\
&=&\frac{\lambda ^{y}}{y!}e^{-\lambda }\lim_{n\longrightarrow \infty }\frac{%
n\left( n-1\right) \left( n-2\right) ...(n-y+1)}{n^{y}}\text{ \ [shaving off 
}n-y\text{ terms from numerator]} \\
&=&\frac{\lambda ^{y}}{y!}e^{-\lambda }\lim_{n\longrightarrow \infty }\frac{n%
}{n}\times \frac{\left( n-1\right) }{n}\times \frac{\left( n-2\right) }{n}%
\times ...\times \frac{(n-y+1)}{n} \\
&=&\frac{\lambda ^{y}}{y!}e^{-\lambda }\lim_{n\longrightarrow \infty
}1\times \left( 1-\frac{1}{n}\right) \times \left( 1-\frac{2}{n}\right)
\times ...\times \left( 1-\frac{y+1}{n}\right) \\
&=&\frac{\lambda ^{y}}{y!}e^{-\lambda }\left( 1\right)
\end{eqnarray*}

\item And so%
\begin{equation*}
p\left( y\right) =\frac{\lambda ^{y}}{y!}e^{-\lambda }.
\end{equation*}
\end{itemize}

\item In practice, Poissons are not conveniently calculated. \ We typically
use statistical software to calculate the probability that the Binomial
random variable $Y$ takes on some value $y.$ \ In Stata, we type \texttt{.di
poissonp}$(\lambda ,y)$ to get $P\left( Y=y\right) .$\ 

\item It is the case [proofs are in your book] that for a Poisson RV, 
\begin{eqnarray*}
\mu &=&E\left( Y\right) =\lambda \\
\sigma ^{2} &=&VAR\left( Y\right) =\lambda .
\end{eqnarray*}
\end{itemize}

The take-home point is that there are many theoretical models of specific
experiments. \ Each of them has a probability distribution $p(y)$, and each
has a mean ($E(Y)$) and a variance ($VAR(Y)$). \ Get to know these a bit by
browsing the remainder of Chapter 3. \ See also the inside back cover of
your text for a quick summary.

\subsection{Continuous random variables}

\begin{itemize}
\item We often deal with random variables that can take on an uncountably
infinite number of values. \ These RVs are known as \textit{continuous }%
random variables.

\item The reason we care: \ it's impossible to assign nonzero probabilities
to all the uncountably infinite points on an interval while satisfying that
they all sum to 1. \ Thus the notion of $p(y)$ from discrete world is
irrelevant here. \ We must develop a different method to describe the
probability distribution of a continuous RV.

\item Let's begin by defining the cumulative distribution function (or cdf,
or \textquotedblleft distribution function\textquotedblright ) of the RV $Y$
as $F(y),$ where%
\begin{equation*}
F(y)\equiv P(Y\leq y)\text{ for -}\infty <y<\infty .
\end{equation*}

\FRAME{ftbpF}{4.5637in}{3.2055in}{0pt}{}{}{Figure}{\special{language
"Scientific Word";type "GRAPHIC";maintain-aspect-ratio TRUE;display
"USEDEF";valid_file "T";width 4.5637in;height 3.2055in;depth
0pt;original-width 49.3331in;original-height 34.6001in;cropleft "0";croptop
"1";cropright "1";cropbottom "0";tempfilename
'MU0BBU00.bmp';tempfile-properties "XPR";}}

\item A cdf has the following properties:%
\begin{eqnarray*}
F(-\infty ) &\equiv &\underset{y\rightarrow \text{-}\infty }{\lim }F(y)=0 \\
F(\infty ) &\equiv &\underset{y\rightarrow \infty }{\lim }F(y)=1 \\
&&F(y)\text{ is a nondescreasing function of }y\text{, which means that} \\
y_{1} &<&y_{2}\Longrightarrow F(y_{1})\leq F(y_{2}).
\end{eqnarray*}

\item Both discrete and continous RVs have cdfs. \ See your text [page 158]
for an example of how to develop a cdf of an RV with a binomial distribution
. \ A RV $Y$ with cdf $F(y)$ is said to be \textbf{continuous} if $F(y)$ is
continuous for -$\infty <y<\infty .$ \ By contrast, the cdf's of discrete
RVs are always \textbf{step }functions: they have discontinuities separating
the possible values of $y$ that they can take on.

\item Remember that we talked earlier about the difficulty of assigning
probabilities to points? \ Well with continuous RVs, we don't. \ In fact if $%
Y$ is a continuous RV,%
\begin{equation*}
P(Y=y)=0\forall \text{ real numbers }y\text{.}
\end{equation*}

\item Sounds weird, but isn't. \ When we move to the world of continuous
RVs, the differences between values of $y$ become infinitely small, making
the chance that we see any one particular value of $y$ zero.

\begin{itemize}
\item For example, what's the probability of observing a temperature of
50.73093764 degrees Fahrenheit on October 2 in New York\ City? \ Now add 10
additional random digits to this number. \ Do it again. \ It doesn't take
long to get to values for which it would be quite likely we would never see
even if we were to measure the temperature on all\ October 2s that ever
exist. \ And even \textit{those} values can be made more precise.
\end{itemize}

\item Instead, we get at this notion with the idea of \textbf{density}. \
Define the function $f(y)$ as the derivative of $f$:%
\begin{equation*}
f(y)\equiv \frac{dF(y)}{dy}=F^{\prime }(y).
\end{equation*}

\item Wherever the derivative exists, $f(y)$ is the \textit{probability
density function} (pdf, `density function') for the RV $Y$. \ [NOTE:\ The
following diagram should be labeled "The density function"

\FRAME{ftbpF}{4.181in}{2.6501in}{0pt}{}{}{Figure}{\special{language
"Scientific Word";type "GRAPHIC";maintain-aspect-ratio TRUE;display
"USEDEF";valid_file "T";width 4.181in;height 2.6501in;depth
0pt;original-width 10.6666in;original-height 6.7463in;cropleft "0";croptop
"1";cropright "1";cropbottom "0";tempfilename
'MU0BD801.bmp';tempfile-properties "XPR";}}

\item How to think about this intuitively:

\begin{itemize}
\item Note that where the cdf is changing rapidly (has a steep slope), the
density is larger. \ Where it is changing slowly (has a flatter slope), the
density is smaller.
\end{itemize}

\item Having defined $f(y)\equiv \frac{dF(y)}{dy},$ we therefore can write%
\begin{equation*}
F(y)=\int\limits_{-\infty }^{y}f(t)dt,
\end{equation*}

where $f(\cdot )$ is the pdf and $t$ is a placeholder, the variable of
integration. \ The pdf $f(\cdot )$ has the following properties:%
\begin{eqnarray*}
f(y) &\geq &0\forall y,\text{-}\infty <y<\infty . \\
\int_{-\infty }^{\infty }f(y)dy &=&1.
\end{eqnarray*}

\item Be sure to work through the examples in your book to get a sense of
the math of cdf's and pdf's.

\item Now what if we want to find the probability that the random variable $Y
$ falls in a certain interval, e.g. $P(a\leq Y\leq b)$? \ It is the area
under the density function in this interval [draw]%
\begin{eqnarray}
P(a &<&Y\leq b)=P(Y\leq b)-P(Y\leq a)  \notag \\
&=&F(b)-F(a)  \notag \\
&=&\int_{a}^{b}f(y)dy.
\end{eqnarray}

\item In the continuous RV case, note that $P(a<Y<b)=P(a<Y\leq b)=P(a\leq
Y<b)=P(a\leq Y\leq b).$ \ Why?

\item Note that this is not necessarily the case for discrete RVs, because
for those kinds of RVs, $P(y\leq a)$ is not necessarily equal to $P(y<a).$
\end{itemize}

\subsection{Expected values for continuous random variables}

\begin{itemize}
\item Remember that in the case of discrete RV, we wrote%
\begin{equation*}
E(Y)\equiv \sum_{y}yp(y)\text{ \ and }E[g(Y)]=\sum_{y}g(y)p(y)
\end{equation*}%
The math of expectations for continuous RVs is very similar:%
\begin{eqnarray*}
E(Y) &\equiv &\int_{-\infty }^{\infty }yf(y)dy \\
E[g(Y)] &=&\int_{-\infty }^{\infty }g(y)f(y)dy
\end{eqnarray*}

\item Furthermore:%
\begin{eqnarray*}
VAR(Y) &\equiv &\int_{-\infty }^{\infty }(y-\mu )^{2}f(y)dy \\
&=&E(Y^{2})-\mu ^{2}.\text{ \ [Exercise.]}
\end{eqnarray*}

\item Reassuringly, this is the same as in the discrete case, as we showed
last time in class.
\end{itemize}

\subsection{Theoretical models of continuous probability distributions}

\begin{itemize}
\item We'll look at two continuous RVs in detail:

\begin{itemize}
\item The \textbf{Uniform}

\item The \textbf{Normal}
\end{itemize}

\item And three distributions related to the Normal that we will use
constantly in statistical tests:

\begin{itemize}
\item 
\begin{itemize}
\item the \textbf{Chi-squared} $\left( \chi ^{2}\right) $ distribution 

\item the\textbf{\ t-distribution} 

\item the \textbf{F idstribution}
\end{itemize}
\end{itemize}
\end{itemize}

\subsection{The Uniform distribution}

\begin{itemize}
\item Consider a random variable that can take on any value in an interval
between two values, and these values are equiprobable.

\begin{itemize}
\item e.g., the date of the election called by a prime minister who must
call an election at some point in her five-year term.

\item the length of a student essay (in words) that is required to be
between 1,000 and 2,000 words.

\item the length of a Tweet, which must be between 1 and 148 characters.
\end{itemize}

\item (Hold off for a moment on scrutinizing whether these values are really
all equiprobable; we'll get to that in a minute.)

\item (Let's also hold off for a moment at balking at the fact that these
are actually examples of discrete random variables, rather than continuous
RVs.)

\item We can represent the density function of such an RV like this (change $%
a$ and $b$ to the thetas):\FRAME{ftbpF}{3.3831in}{1.9709in}{0pt}{}{}{Figure}{%
\special{language "Scientific Word";type "GRAPHIC";maintain-aspect-ratio
TRUE;display "USEDEF";valid_file "T";width 3.3831in;height 1.9709in;depth
0pt;original-width 7.2536in;original-height 4.2133in;cropleft "0";croptop
"1";cropright "1";cropbottom "0";tempfilename
'MU0HSZ03.bmp';tempfile-properties "XPR";}}

\item This flat density function represents the fact that all values in the
interval are equiprobable.

\item This leads to a pdf that looks like this:%
\begin{equation*}
f\left( y\right) =\left\{ 
\begin{array}{cc}
\frac{1}{\theta _{2}-\theta _{1}}, & \theta _{1}\leq y\leq \theta _{2} \\ 
0, & \text{elsewhere.}%
\end{array}%
\right. 
\end{equation*}

\item How do we get to the pdf? \ Simple geometry. \ 

\begin{itemize}
\item We know the rectangular area under the distribution function must
equal 1.

\item We know the length of the base of the rectangle: it's $\theta
_{2}-\theta _{1}.$

\item Therefore its height $h$ must solve  $\left( \theta _{2}-\theta
_{1}\right) h=1$ and thus $h=\frac{1}{\theta _{2}-\theta _{1}}.$
\end{itemize}

\item Note that when we "stretch out" the interval of a Uniform RV, its
density gets smaller and smaller. \ Can you explain the intuition for this?

\item Let's derive the cdf of the Uniform:%
\begin{eqnarray*}
F\left( y\right)  &=&\int\limits_{-\infty }^{y}f(t)dt \\
&=&\int\limits_{\theta _{1}}^{y}\frac{1}{\theta _{2}-\theta _{1}}dt \\
&=&\left. \frac{t}{\theta _{2}-\theta _{1}}\right\vert _{\theta _{1}}^{y}=%
\frac{y-\theta _{1}}{\theta _{2}-\theta _{1}}
\end{eqnarray*}

\item Now let's show that $\mu =E\left( Y\right) =\frac{\theta _{1}-\theta
_{2}}{2}.$%
\begin{eqnarray*}
E\left( Y\right)  &\equiv &\int yf\left( y\right) dy \\
&=&\int_{\theta _{1}}^{\theta _{2}}y\frac{1}{\theta _{2}-\theta _{1}}dy \\
&=&\frac{1}{\theta _{2}-\theta _{1}}\left. \frac{y^{2}}{2}\right\vert
_{\theta _{1}}^{\theta _{2}}=\frac{\left( \theta _{2}\right) ^{2}-\left(
\theta _{1}\right) ^{2}}{2\left( \theta _{2}-\theta _{1}\right) } \\
&=&\frac{\theta _{1}+\theta _{2}}{2}.
\end{eqnarray*}

\item Of course, this is quite intutive. \ The expected value of a Uniform
RV should be at the point that divides the interval in half!

\item A final result [proof will be on your HW]: 
\begin{equation*}
\sigma ^{2}=VAR(Y)=\frac{\left( \theta _{2}-\theta _{1}\right) ^{2}}{12}
\end{equation*}

\item OK, now let's deal with the quibbles. \ When we think about the
examples I mentioned earlier, we know enough about them that they are not
properly modeled as having all their values equiprobable. \ [Go through
examples.]

\item However, what if we began examining a social process knowing nothing
about it?

\begin{itemize}
\item Or more to the point, we wanted to be completely \textit{agnostic }%
about how the values might be distributed? \ 

\item Assuming that the process follows a Uniform distribution is a good way
to do this. \ And in fact, that's exactly what we do in a lot of statistical
modeling. \ We also do this in a lot of formal modeling when we want to be
agnostic about the beliefs an actor might have about the value of something.
\ In both cases, we call this "Uniform prior beliefs," or just a "Uniform
prior."
\end{itemize}
\end{itemize}

\subsection{The Normal distribution}

\begin{itemize}
\item Many empirical distributions are closely approximated by a
distribution that is:

\begin{itemize}
\item symmetric

\item has positive (non-zero) probability for all possible values of $y$

\item and is "bell shaped": specifically it has inflection points at one
standard deviation away from its mean.
\end{itemize}

\item In fact, it can be shown that:

\begin{itemize}
\item in repeated random samples from a population

\item the means of these samples

\item are distributed around the population mean, $\mu ,$ as described by a
density function that we can actually write down.

\item The proof of this phenomenon is called the Central Limit Theorem, and
this density function is defined as the Normal density function.
\end{itemize}

\item We won't be proving the Central Limit Theorem. \ In a few lectures, I
will instead be giving you empirical evidence for its existence. \ 

\item So unlike the other PMFs and PDFs we've looked at so far, we won't
derive this function. \ It requires learning a lot of mathematics that I
don't think you'll find particularly relevant.

\item Instead, you will take it on faith that a RV $Y$ has a Normal
probability distribution iff, for $\sigma >0$ and $-\infty <y<\infty ,$ the
density function of $Y$ is%
\begin{equation*}
f(y)=\frac{1}{\sigma \sqrt{2\pi }}e^{\frac{-(y-\mu )^{2}}{2\sigma ^{2}}},%
\text{ for }-\infty <y<\infty .
\end{equation*}

\item Draw on board, noting $\mu $ on x-axis (axis is labeled `$y$'), y-axis
labeled `$f(y)$', height of curve labeled $f(y)=\frac{1}{\sigma \sqrt{2\pi }}%
e^{\frac{-(y-\mu )^{2}}{2\sigma ^{2}}}.$

\item Note that the Normal has two parameters: $\mu $ and $\sigma .$

\item As usual, if empirical values in population are distributed Normal,
then%
\begin{equation*}
E(Y)=\mu \text{ and }VAR(Y)=\sigma ^{2}.
\end{equation*}

\item So based on what we learned earlier, what is $P(a\leq Y\leq b)$ if $Y$
is distributed Normal?%
\begin{equation*}
P(a\leq Y\leq b)=\int_{a}^{b}f(y)dy=\int_{a}^{b}\frac{1}{\sigma \sqrt{2\pi }}%
e^{\frac{-(y-\mu )^{2}}{2\sigma ^{2}}}dy.
\end{equation*}

\item \lbrack Shade in area of graph.]

\item A closed-form expression for this interval does not exist. \
Evaluating requires numerical methods. \ Commands are available in Stata
(you'll learn them).

\item Typically (and you'll recall this from any statistics class!), we 
\textbf{standardize }a Normally distributed variable so that our units are
measured in terms of standard deviations. \ That is we transform that normal
RV $Y$ into the standard normal RV $Z:$%
\begin{equation*}
Z\equiv \frac{Y-\mu }{\sigma }.
\end{equation*}

\item The RV $Z$ is itself distributed Normal with mean zero and standard
deviation one. \ Generally wecan standardize any Normal variable without
losing any relevant information. \ The original distribution is now
expressed in units of the standard deviation of the original Normal RV.

\item If we standardize a Normal, its pdf becomes simpler:%
\begin{equation*}
f\left( z\right) =\frac{1}{\sqrt{2\pi }}e^{-\frac{z^{2}}{2}}
\end{equation*}

\item We use the pdf and cdf of the standard Normal so much that we have
special symbols for them:

\begin{itemize}
\item We write $\phi \left( z\right) $ ("or little phi of z") to denote the
pdf of the standard Normal evaluated at $Z=z$

\item and $\Phi \left( z\right) =\Pr \left( Z\leq z\right) $, (or "big phi
of z") to denote the cdf of the standard Normal evaluated at $Z=z.$
\end{itemize}
\end{itemize}

\subsection{Three distributions related to the standard Normal}

\begin{itemize}
\item 
\begin{itemize}
\item There are three distributions related to the Normal that we will use
constantly in statistical tests. \ It's not much use to go into detail about
them here; I just want you to be aware that they all derived from the
Normal. \ We will encounter them again soon.

\begin{itemize}
\item the \textbf{Chi-squared} $\left( \chi ^{2}\right) $ distribution
[where $Y$ is the sum of the squares of a series of standard Normal RVs]

\item the\textbf{\ t-distribution} [where $Y$ is the ratio of a standard
Normal RV / the square root of a chi-squared RV]

\item the \textbf{F idstribution} [where $Y$ is the ratio of two chi-squared
RVs]
\end{itemize}
\end{itemize}
\end{itemize}

\end{document}

% Sections won't have numbers
% Make the page area as large as possible
% Set the first level numbering to arabic and the second level
% to alphabetic. The remaining two levels are arabic
%Change to \pagestyle{empty} to suppress numbers on following pages
%\pagestyle{empty}
% Do not reset outermost enumerate counter


\documentclass[11pt]{article}
%%%%%%%%%%%%%%%%%%%%%%%%%%%%%%%%%%%%%%%%%%%%%%%%%%%%%%%%%%%%%%%%%%%%%%%%%%%%%%%%%%%%%%%%%%%%%%%%%%%%%%%%%%%%%%%%%%%%%%%%%%%%%%%%%%%%%%%%%%%%%%%%%%%%%%%%%%%%%%%%%%%%%%%%%%%%%%%%%%%%%%%%%%%%%%%%%%%%%%%%%%%%%%%%%%%%%%%%%%%%%%%%%%%%%%%%%%%%%%%%%%%%%%%%%%%%
\usepackage{amsmath}
\usepackage{sw20exm1}
\usepackage{mathpazo}

\setcounter{MaxMatrixCols}{10}
%TCIDATA{OutputFilter=LATEX.DLL}
%TCIDATA{Version=5.50.0.2960}
%TCIDATA{<META NAME="SaveForMode" CONTENT="1">}
%TCIDATA{BibliographyScheme=Manual}
%TCIDATA{Created=Tuesday, August 26, 2008 15:10:21}
%TCIDATA{LastRevised=Tuesday, August 06, 2013 15:20:20}
%TCIDATA{<META NAME="ViewSettings" CONTENT="0">}
%TCIDATA{<META NAME="GraphicsSave" CONTENT="32">}
%TCIDATA{<META NAME="DocumentShell" CONTENT="Exams and Syllabi\SW\SW Exam #1 - 8.5x11 Page">}
%TCIDATA{CSTFile=Exam.cst}

\setlength{\topmargin}{-0.75in}
\setlength{\textheight}{9.25in}
\setlength{\oddsidemargin}{0.0in}
\setlength{\evensidemargin}{0.0in}
\setlength{\textwidth}{6.5in}
\def\labelenumi{\arabic{enumi}.}
\def\theenumi{\arabic{enumi}}
\def\labelenumii{(\alph{enumii})}
\def\theenumii{\alph{enumii}}
\def\p@enumii{\theenumi.}
\def\labelenumiii{\arabic{enumiii}.}
\def\theenumiii{\arabic{enumiii}}
\def\p@enumiii{(\theenumi)(\theenumii)}
\def\labelenumiv{\arabic{enumiv}.}
\def\theenumiv{\arabic{enumiv}}
\def\p@enumiv{\p@enumiii.\theenumiii}
\pagestyle{plain}
\setcounter{secnumdepth}{0}
\newtheorem{theorem}{Theorem}
\newtheorem{acknowledgement}[theorem]{Acknowledgement}
\newtheorem{algorithm}[theorem]{Algorithm}
\newtheorem{axiom}[theorem]{Axiom}
\newtheorem{case}[theorem]{Case}
\newtheorem{claim}[theorem]{Claim}
\newtheorem{conclusion}[theorem]{Conclusion}
\newtheorem{condition}[theorem]{Condition}
\newtheorem{conjecture}[theorem]{Conjecture}
\newtheorem{corollary}[theorem]{Corollary}
\newtheorem{criterion}[theorem]{Criterion}
\newtheorem{definition}[theorem]{Definition}
\newtheorem{example}[theorem]{Example}
\newtheorem{exercise}[theorem]{Exercise}
\newtheorem{lemma}[theorem]{Lemma}
\newtheorem{notation}[theorem]{Notation}
\newtheorem{problem}[theorem]{Problem}
\newtheorem{proposition}[theorem]{Proposition}
\newtheorem{remark}[theorem]{Remark}
\newtheorem{solution}[theorem]{Solution}
\newtheorem{summary}[theorem]{Summary}
\newenvironment{proof}[1][Proof]{\noindent\textbf{#1.} }{\ \rule{0.5em}{0.5em}}
\input{tcilatex}
\begin{document}


\begin{center}
\textbf{Wilf Family\ Department of Politics}

\textbf{Quantitative Research in Political Science I, Professor Patrick\ Egan%
}

\textbf{Diagnostic Exam}
\end{center}

\bigskip

\noindent \textbf{Instructions.} This exam should take you up to three hours
to complete. \ You are welcome to use notes and books from previous classes
as you need them. \ Show your work for all problems on separate sheets of
paper. \ \ 

\bigskip

\begin{enumerate}
\item The random variable $Y$ has a density function

\begin{equation*}
f(y)=\left\{ 
\begin{array}{cc}
cy, & 0\leq y\leq 4 \\ 
0, & \text{elsewhere}%
\end{array}%
\right.
\end{equation*}

\begin{enumerate}
\item What value of $c$ would make $f(y)$ a probability density function?

\item What is $F(y)$?

\item What is the probability that $Y$ takes on a value in the interval $%
[1,3]$?
\end{enumerate}

\item 100 residents polled in Country $A$ have a mean income of \$42,000
with a standard deviation of \$1,000. \ 100 residents polled in Country $B$
have a mean income of \$44,000 with a standard deviation of \$2,400. \ \ 

\begin{enumerate}
\item Would you accept the claim that Country $B$ is more affluent than
Country $A$ at the 95\% confidence level? \ Be sure to explain fully your
choice of a test statistic and rejection region, and provide a diagram
showing the rejection region, critical value, and test statistic.

\item Provide a \textbf{90\%} confidence interval for the difference in mean
income between the two countries.
\end{enumerate}

\item The cdf of a random variable $X$ is given by%
\begin{equation*}
F(X)=\frac{e^{x}}{1+e^{x}}.
\end{equation*}

\begin{enumerate}
\item Is it possible to compute the probability density function of $X$? \
If so, what is it?
\end{enumerate}

\item You believe that increases in education are associated with increases
in GDP. \ You also believe that this association is stronger for democracies
than non-democracies. \ In addition to a measure of GDP, you have continuous
measures of education (\textit{education}) and democracy (\textit{democracy}%
), where \textit{democracy} is measured on a scale of 1 to 10 (10 is
highest). \ Assume the relationship between education and GDP is linear. \ 

\begin{enumerate}
\item Specify an appropriate model.

\item Provide an equation for the for the marginal impact of education on
GDP.

\item You estimate the model with OLS. \ Provide an expression for your
estimate of the marginal impact of education on GDP.

\item Provide an expression for the standard error for your answer in (c).

Note: Your answers to (e) and (f) should be in terms of quantities from your
OLS estimates.

\item For a country with a democracy level of 6, by how much would you
expect GDP to increase if education rose from 12 to 16?

\item Provide a 95\% confidence interval for your answer in (e).

\item You think it likely that there are diminishing returns to education in
this relationship. \ How might you transform your data to account for this?

\item Does this analysis demonstrate that education leads to increases in
GDP? \ Why or why not?
\end{enumerate}

\item You believe that the true model determining the dropout rate in school
districts is given by

\begin{equation*}
dropout_{i}=\beta _{0}+\beta _{1}perpupil_{i}+\beta
_{2}povertyrate_{i}+\beta _{3}classsize_{i}+u_{i},
\end{equation*}

\qquad where

\begin{itemize}
\item $dropout_{i}$ is the dropout rate in the $i$th district as a
percentage of students enrolled;

\item $perpupil_{i}$ is the per-pupil expenditure in the $i$th district (in
thousands of dollars);

\item $povertyrate_{i}$ is the percentage of students in the $i$th district
with family incomes below the poverty line; and

\item $classsize_{i}$ is the average class size (in number of students) in
the $i$th district.
\end{itemize}

You run OLS and find that

\begin{equation*}
\begin{tabular}{ll}
$\widehat{\beta }_{0}=5.78$ & $\sigma _{\widehat{\beta }_{0}}=0.96$ \\ 
$\widehat{\beta }_{1}=-3.00$ & $\sigma _{\widehat{\beta }_{1}}=0.48$ \\ 
$\widehat{\beta }_{2}=1.00$ & $\sigma _{\widehat{\beta }_{2}}=0.70$ \\ 
$\widehat{\beta }_{3}=0.40$ & $\sigma _{\widehat{\beta }_{3}}=0.84$ \\ 
\multicolumn{1}{c}{} & \multicolumn{1}{c}{} \\ 
\multicolumn{2}{c}{$R^{2}=0.35$} \\ 
\multicolumn{2}{c}{$N=1000$}%
\end{tabular}%
\end{equation*}

\begin{enumerate}
\item What is a theoretically justified hypothesis about the relationship
between the percentage of a district's students below the poverty line and
the dropout rate? \ Use this hypothesis in a test determining whether you
are 95\% certain that increases in the poverty rate lead to increases in the
attrition rate. \ Include equations for $H_{0}$ and $H_{1}$.

\item You find that there is a high correlation between a district's poverty
rate and average class size. \ How might this affect your estimates of $%
\beta _{2}$ and $\beta _{3}$? \ 

\item You decide that you would be content to show that the effect of either
poverty or class size is significant at the 95\% level. \ Is either of these
two variables significant at the 95\% level (note this is a question about
joint significance)? \ Write down $H_{0}$ and $H_{1}$, as well as equations
for any test statistic you use. \ Big hint: \ You obtain an $R^{2}$ of .27
from the bivariate regression%
\begin{equation*}
dropout_{i}=\delta _{0}+\delta _{1}perpupil_{i}+\nu _{i}.
\end{equation*}

\item Suppose per-pupil expenditure were measured as hundred dollars spent
per student (rather than thousand dollars spent). \ Can you say what $%
\widehat{\beta }_{1}$ would be? Can you say what the $t$-statistic
associated with $\widehat{\beta }_{1}$ would be?

\item Assuming that the multivariate model specified above is the true
model, is $\beta _{2}$ the total effect of poverty on the dropout rate? \
Why or why not?
\end{enumerate}
\end{enumerate}

\end{document}

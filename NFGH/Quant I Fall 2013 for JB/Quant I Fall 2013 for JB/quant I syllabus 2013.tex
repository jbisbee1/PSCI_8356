
\documentclass[11pt]{article}
%%%%%%%%%%%%%%%%%%%%%%%%%%%%%%%%%%%%%%%%%%%%%%%%%%%%%%%%%%%%%%%%%%%%%%%%%%%%%%%%%%%%%%%%%%%%%%%%%%%%%%%%%%%%%%%%%%%%%%%%%%%%%%%%%%%%%%%%%%%%%%%%%%%%%%%%%%%%%%%%%%%%%%%%%%%%%%%%%%%%%%%%%%%%%%%%%%%%%%%%%%%%%%%%%%%%%%%%%%%%%%%%%%%%%%%%%%%%%%%%%%%%%%%%%%%%
\usepackage[abbr]{harvard}
\usepackage{amssymb}
\usepackage{setspace,graphics,epsfig,amsmath,rotating,amsfonts,mathpazo}

\setcounter{MaxMatrixCols}{10}
%TCIDATA{OutputFilter=LATEX.DLL}
%TCIDATA{Version=5.50.0.2960}
%TCIDATA{<META NAME="SaveForMode" CONTENT="1">}
%TCIDATA{BibliographyScheme=BibTeX}
%TCIDATA{Created=Thursday, September 11, 2008 15:11:56}
%TCIDATA{LastRevised=Sunday, September 22, 2013 19:41:49}
%TCIDATA{<META NAME="GraphicsSave" CONTENT="32">}
%TCIDATA{<META NAME="DocumentShell" CONTENT="Articles\SW\article.egan">}

\topmargin=0 in \headheight=0in \headsep=0in \topskip=0in \textheight=9in \oddsidemargin=0in \evensidemargin=0in \textwidth=6.5in
\input{tcilatex}
\begin{document}


New York University

Wilf Family Department of Politics

Fall 2013

\bigskip

\begin{center}
{\large \textbf{Quantitative Research in Political\ Science I}}

(POL-GA.1250)

Professor Patrick Egan
\end{center}

This class is the first step in your training for a lifetime of research and
scholarship using quantitative analysis. \ The goal of this class is to
provide you with a thorough grounding in the basic tools of quantitative
analysis as used by political scientists so that you can begin to (1)
properly conduct your own research; (2) read and evaluate research performed
by other political scientists; and (3) prepare for the additional training
you will need in order to answer questions about politics you find
interesting and worthwhile.

\bigskip

\textbf{Logistics}

\bigskip

\textit{Class meetings.} \ We will meet on Mondays and\ Wednesdays from 10
a.m. to noon in Room 217. \ Attendance is mandatory. \ There is also a
mandatory computer lab session held on Fridays from at 10 a.m. in the
3rd-floor computer lab (Room 335).

\bigskip

\textit{Contact information. \ }Email is best way to reach me: \ \texttt{%
patrick.egan@nyu.edu}. \ \ Phone sometimes works, too: \ (212) 992-8078. \ \
Office hours are Tuesdays from 4 to 6 p.m., and by appointment. \ I'm in
Room 327.

\bigskip

\textit{Teaching assistant. \ }Andrew Peterson (\texttt{ajp502@nyu.edu}) is
our teaching assistant.\ \ He will handle problem sets and is your resource
for understanding what we cover together in class. \ He will also conduct
the computer lab sessions held on Fridays and is your resource for learning
Stata, the statistical software program on which we will be training you
this semester. \ He will be holding weekly office hours at a time T.B.A.

\bigskip 

\textit{Course website. \ }Assignments, resources and datasets will be
posted on the Quant I site on NYU\ Classes, accessible through NYU\ Home at 
\texttt{https://home.nyu.edu/academics}.

\bigskip 

\textbf{Course requirements and grading}

\bigskip

Your grade is determined as follows:

\qquad 
\begin{tabular}{ll}
Problem sets & 40\% \\ 
Midterm exam & 25\% \\ 
Final exam & 35\%%
\end{tabular}

\bigskip

You are encouraged to work on problem sets together, but you must write up
each problem set on your own and perform any analyses yourself. \
Handwritten assignments will be accepted through the midterm examination. \
After that, problem sets must be submitted in some word-processed format
(preferably\ LaTeX, but Microsoft Word using\ Equation\ Editor is
acceptable, too). \ Problem sets will typically be distributed on Thursdays
and due Mondays. The first will be due on Monday, September 30. \ Late
problem sets will not be accepted. \ They will be assigned a score of zero.

\bigskip

\newpage

\textbf{Books}

\bigskip

All required and optional texts are available at the NYU\ Bookstore, 726
Broadway.

\bigskip

\textit{Required texts}

\begin{itemize}
\item Wackerly, Mendenhall, and Scheaffer. \textit{Mathematical Statistics
with Applications}, 7th edition. Thompson. [WMS]

\item Wooldridge. \ \textit{Introductory Econometrics}, 5th edition. Cengage.
\end{itemize}

\textit{Previous editions }of these texts are nearly identical, much less
expensive, and therefore acceptable; note however that problem sets will be
keyed to the latest editions. \ If you purchase a previous version, it is
your responsibility to ensure that you are working on the correct problems.

\bigskip

\textit{Optional texts}

\begin{itemize}
\item Cameron \&\ Trivedi, \textit{Microeconomics Using\ Stata. \ }Revised
Edition.\textit{\ }(Stata Press, 2010). \ An excellent resource whose only
drawback is that it can at times presume too much about your knowledge of
econometrics. \ The original edition is also fine. \ If you find this book
rough going, try: (1) any of the various editions of Acock, \textit{A Gentle
Introduction to Stata}; (2) any edition of Kohler and Kreuter, \textit{Data
Analysis Using Stata }or (3) any recent edition of Hamilton, \textit{%
Statistics with Stata. \ }\ 

\item Mitchell. \textit{A Visual Guide to Stata Graphics}. 3rd edition.
Stata Press. A beautiful, incredibly useful book on the graphics
capabilities of Stata. The 2nd edition, available used at a lower price, is
almost as good.

\item Lamport. \textit{LaTeX: A Document Preparation System. }2nd edition.
Addison-Wesley. Although we will point you to lots of good LaTeX resources
available on the Web, it can be helpful to have an actual reference book at
your fingertips when working through a LaTeX document. This one's pretty
good and relatively affordable. (More thorough---and more expensive---is 
\textit{Guide to LaTeX, }4th ed. by Kopka and Daly, Addison-Wesley.)

\bigskip
\end{itemize}

\textbf{Schedule of topics and readings }(subject to change)

\bigskip

\textit{How to think about the readings: }consider the readings
supplementary, not primary. \ Your primary source of information will be
class lectures, which will not always adhere to either the order or
presentation of the texts. \ It is often helpful to read more than one
presentation of an idea before you truly absorb it, so be ecumenical. \ If
the texts are not working for you, I can recommend others at lower or higher
levels that may be more appropriate. \ \ \ 

\bigskip

\begin{itemize}
\item \textbf{Part I. Description and Inference Regarding \textit{One}
Variable}

\begin{itemize}
\item Week 1 (Sept. 23-Sept. 27). \textit{\ Introduction to Data Analysis. \
Summarizing and Displaying Univariate Data. \ Probability Basics.}

\begin{itemize}
\item WMS, Chapters 1 and 2. \ 

\item Lab: LaTeX basics. \bigskip\ 
\end{itemize}

\item Week 2 (Sept. 30-Oct. 4).\textit{\ \ The Math of Expectations. \
Discrete and Continuous Probability\ Distributions.}

\begin{itemize}
\item WMS, Chapters 3 and 4.

\item Lab: Stata basics. \ Getting started, logs, good practice. \ (See
Nagler, \textquotedblleft Coding Style and Good Computing
Practices\textquotedblright ).\bigskip
\end{itemize}

\item Week 3 (Oct. 7-11). \textit{The Notion of Independence. \ Linear
Combinations of Random\ Variables. \ The Central Limit Theorem.}

\begin{itemize}
\item WMS, Chapter 5 (part), Chapter 7.

\item Lab: Generating and visualizing random variables. \bigskip
\end{itemize}

\item Week 4 (Oct. 16-18). \textit{Estimation. \ }

\begin{itemize}
\item No class Oct. 14 (University holiday).

\item WMS, Chapter 8.

\item Lab: The Central Limit Theorem. \ Loops.\bigskip
\end{itemize}

\item Week 5 (Oct. 21-25). \textit{Properties of Point Estimators. \
Hypothesis Testing.}

\begin{itemize}
\item WMS, Chapters 9 and 10.

\item Lab: Hypothesis testing.\bigskip
\end{itemize}
\end{itemize}
\end{itemize}

\begin{center}
MIDTERM EXAMINATION COVERING PART I OF COURSE: NOVEMBER 1\bigskip
\end{center}

\begin{itemize}
\item \textbf{Part II. Description and Inference Regarding Relationships 
\textit{Among} Variables}

\begin{itemize}
\item Week 6 (Sept. 28-Nov. 1). \textit{\ Covariance and Correlation. \
Visualizing\ Relationships. \ The Regression Line. \ The Normal Equations.}

\begin{itemize}
\item No lab this week (Midterm examination).\bigskip
\end{itemize}

\item Week 7 (Nov. 4-8).\textit{\ \ The Linear Model: Bivariate Case. \
Properties of Least Squares. \ Statistical Inference. \ The\ Gauss-Markov
Assumptions.}

\begin{itemize}
\item Wooldrige, Ch. 2.

\item Lab: Bivariate regression.\bigskip
\end{itemize}

\item Week 8 (Nov. 11-15). \textit{The Linear Model: Multivariate Case. \
OLS in Matrix Form. \ }

\begin{itemize}
\item Wooldridge, Chs. 3 and 4.

\item Lab: Multivariate regression I.\bigskip
\end{itemize}

\item Week 9 (Nov. 18-22). \ \textit{The Linear Model: Further Topics I:
Goodness of fit. Variable transformations, quadratics.}

\begin{itemize}
\item Wooldridge, Chs. 5 and 6.

\item Lab: Multivariate regression II.\bigskip
\end{itemize}

\item Week 10 (Nov. 25-27). \textit{The Linear Model: Further Topics II:
Interaction terms. Indicator variables. \ Heteroskedasticity.}

\begin{itemize}
\item Wooldridge, Chs 7 and 8 (selections).

\item No lab this week (Thanksgiving).\bigskip
\end{itemize}

\item Week 11 (Dec. 2-6). \textit{The Linear Model: Further Topics III: The
Linear Probability Model. Measurement error. \ The threats of selection bias
and endogeneity. \ Outliers.}

\begin{itemize}
\item Wooldridge, Ch 9 (selections).

\item Lab: Multivariate regression III.\bigskip
\end{itemize}

\item Week 12 (Dec. 9-13). \textit{Finale.}

\begin{itemize}
\item What's next (12/9): Wooldridge Ch. 17 (selections) \ 

\item Review (12/11)

\item 5-hour final\ exam (12/13, 10 am - 3 pm). \ 
\end{itemize}
\end{itemize}
\end{itemize}

\end{document}

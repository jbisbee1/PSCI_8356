
\documentclass[11pt]{article}
%%%%%%%%%%%%%%%%%%%%%%%%%%%%%%%%%%%%%%%%%%%%%%%%%%%%%%%%%%%%%%%%%%%%%%%%%%%%%%%%%%%%%%%%%%%%%%%%%%%%%%%%%%%%%%%%%%%%%%%%%%%%%%%%%%%%%%%%%%%%%%%%%%%%%%%%%%%%%%%%%%%%%%%%%%%%%%%%%%%%%%%%%%%%%%%%%%%%%%%%%%%%%%%%%%%%%%%%%%%%%%%%%%%%%%%%%%%%%%%%%%%%%%%%%%%%
\usepackage[abbr]{harvard}
\usepackage{amssymb}
\usepackage{setspace,graphics,epsfig,amsmath,rotating,amsfonts,mathpazo}

\setcounter{MaxMatrixCols}{10}
%TCIDATA{OutputFilter=LATEX.DLL}
%TCIDATA{Version=5.50.0.2953}
%TCIDATA{<META NAME="SaveForMode" CONTENT="1">}
%TCIDATA{BibliographyScheme=BibTeX}
%TCIDATA{Created=Thursday, September 11, 2008 15:11:56}
%TCIDATA{LastRevised=Wednesday, December 11, 2013 09:45:34}
%TCIDATA{<META NAME="GraphicsSave" CONTENT="32">}
%TCIDATA{<META NAME="DocumentShell" CONTENT="Articles\SW\article.egan">}

\topmargin=0 in \headheight=0in \headsep=0in \topskip=0in \textheight=9in \oddsidemargin=0in \evensidemargin=0in \textwidth=6.5in
\input{tcilatex}
\begin{document}


New York University

Wilf Family Department of Politics

\begin{center}
{\large \textbf{Quantitative Research in Political\ Science I}}

Professor Patrick Egan

\bigskip

\textbf{Interactions Among Predictors of }$\mathbf{Y}$\bigskip
\end{center}

It can be the case that a data generating process (DGP) is characterized by 
\textit{ceteris paribus} effect on $y$ of one predictor variable, say $x$,
changing depending on the value of another predictor variable, e.g. $z$. \ 

This can happen for two (very distinct) reasons. \ It may be that (1) $x$'s
effect on $y$ is \textit{moderated }by $z$, or that (2) $x$ affects $y,$
which in turn affects $z$, in which case $x$'s effect on $y$ is said to be 
\textit{mediated }by $z$. \ Perhaps surprisingly, we use the \textit{same }%
specification in OLS to account for either moderation or mediation, which is
to create a term that is the product of $x$ and $z$ and include it--as well
as $x$ and $z$ themselves---as predictors of $y$, like this:%
\begin{equation*}
y=\beta _{0}+\beta _{1}x+\beta _{2}z+\beta _{3}xz+u.
\end{equation*}

Here, $xz$ is what is known as an \textbf{interaction term }that models the
effect of $x$ on $y$ as depending on the level of $z$. \ Specifically, the
partial effect of $x$ on $y$ is estimated as%
\begin{equation*}
\widehat{\frac{\partial y}{\partial x}}=\widehat{\beta }_{1}+\widehat{\beta }%
_{3}z.
\end{equation*}

Thus $x$'s effect on $y$ is no longer linear, but rather it changes
depending on the level of $z$:

\begin{itemize}
\item If $\widehat{\beta }_{1}$ and $\widehat{\beta }_{3}$ are of the 
\textit{same} sign, then the effect of $x$ becomes greater in magnitude to
the extent $z$ is large. \ 

\item If $\widehat{\beta }_{1}$ and $\widehat{\beta }_{3}$ are of \textit{%
opposite} signs, then the magnitude of $x$'s effect diminishes (and perhaps
even changes sign) as $z$ becomes large.\bigskip
\end{itemize}

Note that we are \textit{simultaneously} estimating the partial effect of $z$
on $y$, as%
\begin{equation*}
\widehat{\frac{\partial y}{\partial z}}=\widehat{\beta }_{2}+\widehat{\beta }%
_{3}x.
\end{equation*}

\bigskip An additional important point is that if $z$ is an indicator
variable, estimating the regression of $y$ on $x$ separately over the two
values of $z$ is equivalent to estimating the interaction model. \ That is,
estimates derived from the models 
\begin{equation*}
y=\delta _{0}+\delta _{1}x+u,z=1\text{ \ and \ }y=\gamma _{0}+\gamma
_{1}x+u,z=0\text{ }
\end{equation*}
will return estimates $\widehat{\delta }_{1}$ and $\widehat{\gamma }_{1}$
such that 
\begin{eqnarray*}
\widehat{\beta }_{0} &=&\widehat{\delta }_{0} \\
\widehat{\beta }_{1} &=&\widehat{\delta }_{1} \\
\widehat{\beta }_{2} &=&\widehat{\delta }_{0}-\widehat{\gamma }_{0}\text{,
and} \\
\widehat{\beta }_{3} &=&\widehat{\delta }_{1}-\widehat{\gamma }_{1}\text{. }
\end{eqnarray*}

Let's now look at the cases of moderation and mediation in turn.\bigskip

\pagebreak \textbf{1. }$x$\textbf{'s effect on }$y$\textbf{\ is MODERATED by 
}$z$

\FRAME{dhF}{6.7729in}{3.4637in}{0pt}{}{}{moderator.bmp}{\special{language
"Scientific Word";type "GRAPHIC";maintain-aspect-ratio TRUE;display
"USEDEF";valid_file "F";width 6.7729in;height 3.4637in;depth
0pt;original-width 6.44in;original-height 3.2802in;cropleft "0";croptop
"1";cropright "1";cropbottom "0";filename 'C:/Users/Patrick
Egan/Desktop/moderator.bmp';file-properties "XNPEU";}}

\begin{itemize}
\item In this setup, it is assumed (sometimes implausibly) that $z$ enters
the DGP in some way that is independent of $x$. \ The varable $z$ is called
a \textit{moderator}. \ 

\item In survey research, it is often the case that $z$ is some (mostly)
indelible characteristic of an individual (such as race, religion, gender,
etc.) while $x$ is something that is modeled as exogenously
\textquotedblleft happening\textquotedblright\ to the individual (such as
being exposed to a political campaign, experiencing hot or cold weather,
having a low draft lottery number, being assigned to treatment instead of
control, etc.). \ 

\item The point here is that the impact of $x$ on $y$ varies for individuals
with different levels of $z$. \ We describe this process in many ways. \ We
may say that \textquotedblleft $x$'s effect on $y$ is heterogeneous with
respect to $z$,\textquotedblright\ or that \textquotedblleft $z$ augments
(or depresses) $x$'s effect on $y.$\textquotedblright\ \ 

\item Note that $z$ is assumed not only to moderate the effect of $x$ on $y$%
; the specification also allows for the possibility that $z$ itself directly
affects $y$.

\item Here, when we instead estimate the model $y=\beta _{0}+\beta _{1}x$,
the estimate $\widehat{\beta }_{1}$ is what is called the average treatment
effect (ATE) of $x$ on $y$, because by ignoring the heterogeneity due to $z$
the model simply generates an estimate of $x$'s average effect on $y$ over
all values of $z$. \ \bigskip
\end{itemize}

\pagebreak \textbf{2. }$x$\textbf{'s effect on }$y$\textbf{\ is MEDIATED by }%
$z$ \FRAME{fhF}{6.6054in}{2.5313in}{0pt}{}{}{mediator.bmp}{\special{language
"Scientific Word";type "GRAPHIC";maintain-aspect-ratio TRUE;display
"USEDEF";valid_file "F";width 6.6054in;height 2.5313in;depth
0pt;original-width 6.5414in;original-height 2.4898in;cropleft "0";croptop
"1";cropright "1";cropbottom "0";filename 'C:/Users/Patrick
Egan/Desktop/mediator.bmp';file-properties "XNPEU";}}\ 

\begin{itemize}
\item In this setup, it is assumed that the presence of $x$ affects the
level of $z$, which in turn affects $y$. The variable $z$ thus mediates the
effect of $x$ on $y$.\ 

\item It is often the case that the roles of $x$ and $z$ are reversed from
how we think about them with regard to moderation. \ For example, in survey
research, $x$ is often some indelible characteristic of an individual while $%
z$ is something that the individual (purposefully or circumstantially)
\textquotedblleft selects\textquotedblright\ into. \ If $y$ is income, some
examples could be: ($x$: gender; $z$: union membership); ($x$: race; $z$:
educational attainment); ($x$: year of birth; $z$: technological skills),
etc. \ The point here is that the impact of $x$ on $y$ occurs
\textquotedblleft through\textquotedblright\ $z$. \ 

\item In the language of mediation, we say that the \textit{total }effect of 
$x$ on $y$ can be decomposed into the \textit{mediated }effect that occurs
through $z$ and the \textit{direct }effect of $x$ on $y$ itself.

\item Here, when we instead estimate the model $y=\beta _{0}+\beta _{1}x$,
the estimate $\widehat{\beta }_{1}$ is what is the total effect of $x$ on $y$%
, because by ignoring any mediation due to $z$ the model simply generates an
estimate of $x$'s total effect on $y$ over all values of $z$.
\end{itemize}

\end{document}

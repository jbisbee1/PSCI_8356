
\documentclass[11pt]{article}
%%%%%%%%%%%%%%%%%%%%%%%%%%%%%%%%%%%%%%%%%%%%%%%%%%%%%%%%%%%%%%%%%%%%%%%%%%%%%%%%%%%%%%%%%%%%%%%%%%%%%%%%%%%%%%%%%%%%%%%%%%%%%%%%%%%%%%%%%%%%%%%%%%%%%%%%%%%%%%%%%%%%%%%%%%%%%%%%%%%%%%%%%%%%%%%%%%%%%%%%%%%%%%%%%%%%%%%%%%%%%%%%%%%%%%%%%%%%%%%%%%%%%%%%%%%%
\usepackage[abbr]{harvard}
\usepackage{amssymb}
\usepackage{setspace,graphics,epsfig,amsmath,rotating,amsfonts,mathpazo}

\setcounter{MaxMatrixCols}{10}
%TCIDATA{OutputFilter=LATEX.DLL}
%TCIDATA{Version=5.50.0.2960}
%TCIDATA{<META NAME="SaveForMode" CONTENT="1">}
%TCIDATA{BibliographyScheme=BibTeX}
%TCIDATA{Created=Thursday, September 11, 2008 15:11:56}
%TCIDATA{LastRevised=Friday, November 01, 2013 18:02:45}
%TCIDATA{<META NAME="GraphicsSave" CONTENT="32">}
%TCIDATA{<META NAME="DocumentShell" CONTENT="Articles\SW\article.egan">}

\topmargin=0 in \headheight=0in \headsep=0in \topskip=0in \textheight=9in \oddsidemargin=0in \evensidemargin=0in \textwidth=6.5in
\input{tcilatex}
\begin{document}


New York University

Wilf Family Department of Politics

Fall 2010

\begin{center}
{\large \textbf{Quantitative Research in Political\ Science I}}

(G53.1250)

Professor Patrick Egan

\bigskip

\textbf{PROBLEM\ SET 5: Due Monday, November 1 at beginning of class.}
\end{center}

\textit{A reminder: you may work with others in the class on this problem
set, and you are in fact encouraged to do so. \ However, the work you hand
in must be your own. \ Handwritten work is acceptable, but word-processed
work (e.g., using LaTeX) is preferred.}

\bigskip

\begin{enumerate}
\item Show that%
\begin{equation*}
S_{U}^{2}\equiv \frac{\sum\nolimits_{i}\left( Y_{i}-\overline{Y}\right) ^{2}%
}{n-1}
\end{equation*}

is an unbiased estimator for the population variance, $\sigma ^{2}.$\medskip

\item WMS Exercise 8.1.

\item WMS Exercise 8.2.

\item WMS\ Exercise 8.3.

\item WMS Exercise 8.4.

\item WMS Exercise 8.5.

\item WMS Exercise 8.6.

\item WMS Exercise 8.8.

\item As your text does on page 445, define the \textit{relative efficiency }%
of two unbiased estimators $\widehat{\theta }_{1}$ and $\widehat{\theta }_{2}
$ of a parameter $\theta $ as $\frac{VAR\left( \widehat{\theta }_{2}\right) 
}{VAR\left( \widehat{\theta }_{1}\right) }.$ \ Now consider the following
proposed estimators for $\mu $ constructed from a random sample $%
Y_{1},Y_{2}...Y_{n}$ from a population with mean $\mu $ and variance $\sigma
^{2}:$%
\begin{equation*}
\widehat{\mu }_{1}\equiv \frac{1}{2}\left( Y_{1}+Y_{2}\right) \qquad 
\widehat{\mu }_{2}\equiv \frac{1}{4}Y_{1}+\frac{1}{2}\left[ \frac{%
Y_{2}+Y_{3}+...+Y_{n-11}}{\left( n-2\right) }\right] +\frac{1}{4}Y_{n}.
\end{equation*}

\begin{enumerate}
\item Show that $\widehat{\mu }_{1}$ and $\widehat{\mu }_{2}$ are unbiased
for $\mu .$

\item Calculate $VAR\left( \widehat{\mu }_{1}\right) $ and $VAR\left( 
\widehat{\mu }_{2}\right) .$

\item What is their relative efficiency?

\item What is the relative efficiency of any two sample means whose sample
sizes differ by one unit (i.e., sample 1 has $n$ units, while sample 2 has $%
n+1$ units) when both samples are drawn from the same population?
\end{enumerate}
\end{enumerate}

\begin{center}
[MORE\ ON\ NEXT\ PAGE]
\end{center}

\begin{enumerate}
\item[10.] You have a client who wishes to estimate the mean age of the
population of a certain neighborhood by drawing a sample of people and
asking them their age. \ The population has mean age $\mu $ and variance $%
\sigma ^{2}.$ \ Assume that a cost of \$5 is incurred per each person
sampled.

\begin{enumerate}
\item Your client suggests a budget of \$1,000. \ With that budget, how far
on average will the sample estimate $\overline{Y}$ fall from the true
population mean $\mu $ ?

\item If you know that $\sigma =16,\ $can you give your client a more
precise answer to (a)?

\item If your client wants to be 90 percent confident that the estimate of
the population's mean age is no more than one year away from the true mean,
does she need to spend more or less money? \ How much more or less?
\end{enumerate}
\end{enumerate}

\end{document}

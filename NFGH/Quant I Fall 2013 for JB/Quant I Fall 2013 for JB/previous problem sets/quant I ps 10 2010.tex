
\documentclass[11pt]{article}
%%%%%%%%%%%%%%%%%%%%%%%%%%%%%%%%%%%%%%%%%%%%%%%%%%%%%%%%%%%%%%%%%%%%%%%%%%%%%%%%%%%%%%%%%%%%%%%%%%%%%%%%%%%%%%%%%%%%%%%%%%%%%%%%%%%%%%%%%%%%%%%%%%%%%%%%%%%%%%%%%%%%%%%%%%%%%%%%%%%%%%%%%%%%%%%%%%%%%%%%%%%%%%%%%%%%%%%%%%%%%%%%%%%%%%%%%%%%%%%%%%%%%%%%%%%%
\usepackage[abbr]{harvard}
\usepackage{amssymb}
\usepackage{setspace,graphics,epsfig,amsmath,rotating,amsfonts,mathpazo}

\setcounter{MaxMatrixCols}{10}
%TCIDATA{OutputFilter=LATEX.DLL}
%TCIDATA{Version=5.50.0.2953}
%TCIDATA{<META NAME="SaveForMode" CONTENT="1">}
%TCIDATA{BibliographyScheme=BibTeX}
%TCIDATA{Created=Thursday, September 11, 2008 15:11:56}
%TCIDATA{LastRevised=Saturday, December 11, 2010 14:06:20}
%TCIDATA{<META NAME="GraphicsSave" CONTENT="32">}
%TCIDATA{<META NAME="DocumentShell" CONTENT="Articles\SW\article.egan">}

\topmargin=0 in \headheight=0in \headsep=0in \topskip=0in \textheight=9in \oddsidemargin=0in \evensidemargin=0in \textwidth=6.5in
\input{tcilatex}
\begin{document}


New York University

Wilf Family Department of Politics

Fall 2010

\begin{center}
{\large \textbf{Quantitative Research in Political\ Science I}}

(G53.1250)

Professor Patrick Egan

\bigskip

"\textbf{PROBLEM\ SET" 10: To be reviewed (as necesary) on Tuesday, December
14.}
\end{center}

\textit{You are not required to turn in this problem set, but you may find
it helpful to complete it as you prepare for the final examination to be
held December 15. \ James will answer questions about this problem set in
the review session scheduled for 10 a.m. on December 14.}

\begin{enumerate}
\item Show (using matrix notation) that by construction in OLS (where $%
\mathbf{i}$ is a vector of ones as in Matrix Algebra Handout II, page 4), 
\begin{equation*}
\mathbf{i}^{\prime }\widehat{\mathbf{u}}=0.
\end{equation*}

\item Show (using matrix notation) that by construction in OLS for any $N$ x
1 vector of observations of the $k$'th regressor $\mathbf{x}_{k}$, it is the
case that%
\begin{equation*}
cov\left( \mathbf{x}_{k},\widehat{\mathbf{u}}\right) =0.
\end{equation*}%
\bigskip

For the rest of this assignment, use (once again) the \textit{\
countycrime.dta} dataset found on our class Blackboard site that you used
for Problem Set \#9.

\item Analyze the following questions using Stata, but answer them with a
few sentences in plain English. \ Provide appropriate Stata output as an
attachment to your assignment.

\begin{enumerate}
\item Controlling for household income, does the percentage of a county's
residents who have college degrees have any impact on crime rates? \ What is
it?

\item Construct a plot displaying the relationship between $\widehat{\mathit{%
crimerate}}$ and \textit{college}. \ To do this, write a loop that generates
the value of $\widehat{\mathit{crimerate}}$ \ when \textit{college }is equal
to 5, 7,...25. \ (In doing so, you may find it helpful to use the
coefficients saved by the \texttt{regress} command in the vector \texttt{e(b)%
}.)\ Then plot these connected points.

\item You are interested in how well this model applies to counties in New
York State. \ Construct a plot showing at a glance that Tompkins County, has
a crime rate much lower than predicted by a regression of \textit{crimerate}
on \textit{college }among counties in New York State.

\item Look up Tompkins County on the Internet. \ Why might it be an outlier?
\ What would we like to control for to see?\bigskip
\end{enumerate}

The following two questions require that you use two techniques not yet
covered in class but discussed already in lab: OLS with indicator variables
and with interaction terms. \ Many of you are already familiar with these
techniques. \ I encourage you to consult the cited sections of Wooldridge if
not. \ We'll cover them in detail next week in class.\bigskip 

\item Is unemployment higher in counties located in the South than in those
located outside the South?

\begin{enumerate}
\item Answer this question first with a $t$-test to compare group means.

\item Now answer this question by running a bivariate specification
regressing \textit{unemprate} on the indicator variable \textit{south. }If
you are unsure about how to interpret the coefficient on \textit{south }in
this context, consult Wooldridge pp. 225-231.

\item Identify two important similarities found between the two analyses. \ 

\item We required a lot more assumptions to perform the second analysis than
the first. \ In a few sentences, reflect upon why they are unnecessary to
answer the question posed at (4) above.\bigskip
\end{enumerate}

\item Is the association between poverty and crime by county stronger in the
South or outside the South?

\begin{enumerate}
\item Answer this question first by running two separate regressions.

\item Now create a term \textit{povxsouth} that incorporates the interaction
between \textit{south} and \textit{povrate}. \ Run the proper analysis that
includes the interaction term. \ If needed, consult Wooldridge pp. 238-243.

\item Identify two important similarities found between the estimates
generated by the two analyses.

\item Construct a plot with two lines: one showing $\widehat{\mathit{%
crimerate}}$\textit{\ }by \textit{povrate }among counties in the South, and
the other showing $\widehat{\mathit{crimerate}}$\textit{\ \ }by\textit{\
povrate }among non-Southern counties.\bigskip
\end{enumerate}

\item Pretend that you are a Marxist-Socialist who wants to show that crime
is caused by poverty. \ To do so, you are hell-bent on finding the
combination of regressors (including \textit{povrate}) that results in a
very large coefficient on \textit{povrate} when \textit{crimerate }is
regressed upon these regressors\textit{.}

\begin{enumerate}
\item Using the proper Stata commands, choose half of your observations at
random and set them aside. \ Do not include these observations in the
analysis.

\item Find the combination of variables that yields the highest possible
coefficient on \textit{povrate }that you can obtain. \ If you're feeling
creative, you should be able to write a Stata loop that accomplishes this,
but that is not necessary. \ Just have fun with it.

\item Once you've arrived at a specification, note the results. \ Now, take
the other half of your dataset out of "cold storage." \ How well does your
model do with these other data? \ What is the lesson learned? \ 
\end{enumerate}
\end{enumerate}

\end{document}


\documentclass[11pt]{article}
%%%%%%%%%%%%%%%%%%%%%%%%%%%%%%%%%%%%%%%%%%%%%%%%%%%%%%%%%%%%%%%%%%%%%%%%%%%%%%%%%%%%%%%%%%%%%%%%%%%%%%%%%%%%%%%%%%%%%%%%%%%%%%%%%%%%%%%%%%%%%%%%%%%%%%%%%%%%%%%%%%%%%%%%%%%%%%%%%%%%%%%%%%%%%%%%%%%%%%%%%%%%%%%%%%%%%%%%%%%%%%%%%%%%%%%%%%%%%%%%%%%%%%%%%%%%
\usepackage[abbr]{harvard}
\usepackage{amssymb}
\usepackage{setspace,graphics,epsfig,amsmath,rotating,amsfonts,mathpazo}

\setcounter{MaxMatrixCols}{10}
%TCIDATA{OutputFilter=LATEX.DLL}
%TCIDATA{Version=5.50.0.2953}
%TCIDATA{<META NAME="SaveForMode" CONTENT="1">}
%TCIDATA{BibliographyScheme=BibTeX}
%TCIDATA{Created=Thursday, September 11, 2008 15:11:56}
%TCIDATA{LastRevised=Wednesday, November 17, 2010 16:55:02}
%TCIDATA{<META NAME="GraphicsSave" CONTENT="32">}
%TCIDATA{<META NAME="DocumentShell" CONTENT="Articles\SW\article.egan">}

\topmargin=0 in \headheight=0in \headsep=0in \topskip=0in \textheight=9in \oddsidemargin=0in \evensidemargin=0in \textwidth=6.5in
\input{tcilatex}
\begin{document}


New York University

Wilf Family Department of Politics

Fall 2010

\begin{center}
{\large \textbf{Quantitative Research in Political\ Science I}}

(G53.1250)

Professor Patrick Egan

\bigskip

\textbf{PROBLEM\ SET 7: Due Monday, November 22 at beginning of class.}
\end{center}

\textit{A reminder: you may work with others in the class on this problem
set, and you are in fact encouraged to do so. \ However, the work you hand
in must be your own. \ Your work must be word-processed in order for you to
receive credit for the assignment.}

\bigskip

For this assigment, you are to use the "\textit{cps sample.dta}"\textit{\ }%
dataset found on our class Blackboard site. \ As mentioned in class, this is
a sample of $N=5,000$ from the Current Population Survey, conducted every
month by the U.S.\ Census Bureau with a representative sample of U.S.
households. \ The \textit{cps sample.dta }datafile is from the CPS survey
conducted in November 2004.\bigskip\ \ 

\textit{Note: }the questions require that you analyze the CPS's household
income variable (\texttt{hufaminc}). \ It is coded at the ordinal level, but
the analyses require that it be at an interval level. \ To do this, create a
recoded version\ of \texttt{hufaminc }in which each case is assigned a
household income equal to the midpoint of their interval of \texttt{%
hufaminc. }Pick some reasonable value for the highest interval of \texttt{%
hufaminc. }\bigskip

\begin{enumerate}
\item The following scatterplot was produced by an analyst interested in
exploring the relationship between turnout (measured with variable \texttt{%
pes1}) and household income (\texttt{hufaminc}). \ The Stata command
generating the scatterplot was \texttt{twoway (scatter hufaminc pes1 ).}%
\bigskip

\FRAME{dtbphF}{3.6322in}{2.6628in}{0pt}{}{}{Figure}{\special{language
"Scientific Word";type "GRAPHIC";maintain-aspect-ratio TRUE;display
"USEDEF";valid_file "T";width 3.6322in;height 2.6628in;depth
0pt;original-width 5.5945in;original-height 4.0949in;cropleft "0";croptop
"1";cropright "1";cropbottom "0";tempfilename
'LC1TUA00.wmf';tempfile-properties "XPR";}}

\begin{enumerate}
\item There are many, many things wrong with this figure with regard to both
accuracy and style. \ Name as many as you can.\bigskip

\item Construct a well-designed figure that best displays the relationship
between household income and turnout. \ This will require recoding variables
and thinking carefully about the levels at which both variables are
measured. \ Provide the Stata (or, if you prefer to use it, \textit{R})
commands you used to recode variables and construct the figure. In a brief
paragraph, explain why you made the choices that you did.\bigskip

\item In a few sentences, describe the relationship you see between income
and turnout.\bigskip
\end{enumerate}

\item Which relationship--that between income and turnout, or education and
turnout--best approximates a linear relationship? \ Name the proper
statistic to be used in answering this question, calculate it, and answer
the question. \ The variable to use for educational attainment is \texttt{%
peeduca}. Note that it, like \texttt{hufaminc}, is coded at the ordinal
level but you want to analyze it as an interval-level variable. \ \bigskip

\item You want to investigate the relationship between country of birth and
current income. \ 

\begin{enumerate}
\item You wish to divide the sample into three groups: (1) those not born in
the U.S.; (2) those born in the U.S. but who have at least one parent not
born in the U.S.; and (3) those born in the U.S. with both parents born in
the U.S. \ Using the variables \texttt{hufaminc, penatvty}, \texttt{pemntvty}%
, and \texttt{pefntvty}, construct a boxplot which displays the distribution
of household income for each of these three groups. \ \textit{Hint}: doing
this will require creating new variables from \texttt{penatvty}, \texttt{%
pemntvty}, and \texttt{pefntvty.} \ You will also need to make choices about
how to deal with missing values. \ Justify any choices you make in a note
accompanying the figure.\bigskip

\item Using the proper statistical tests with an $\alpha =.05$, answer the
following questions. \ Note that additional recoding may be necessary.

\begin{enumerate}
\item Do native-born Americans have higher incomes than non-native born
Americans?

\item Do native-born Americans whose parents were born in the U.S. have
higher incomes than native-born Americans with at least one parent \textit{%
not }born in the U.S.?

\item Do Americans with a foreign-born father and a native-born mother have
lower incomes than Americans with a foreign-born mother and a native-born
father?\bigskip
\end{enumerate}

\item In a few sentences, describe your results. \ \texttt{\ \ }
\end{enumerate}
\end{enumerate}

\end{document}

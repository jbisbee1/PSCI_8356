
\documentclass[11pt]{article}
%%%%%%%%%%%%%%%%%%%%%%%%%%%%%%%%%%%%%%%%%%%%%%%%%%%%%%%%%%%%%%%%%%%%%%%%%%%%%%%%%%%%%%%%%%%%%%%%%%%%%%%%%%%%%%%%%%%%%%%%%%%%%%%%%%%%%%%%%%%%%%%%%%%%%%%%%%%%%%%%%%%%%%%%%%%%%%%%%%%%%%%%%%%%%%%%%%%%%%%%%%%%%%%%%%%%%%%%%%%%%%%%%%%%%%%%%%%%%%%%%%%%%%%%%%%%
\usepackage[abbr]{harvard}
\usepackage{amssymb}
\usepackage{setspace,graphics,epsfig,amsmath,rotating,amsfonts,mathpazo}

\setcounter{MaxMatrixCols}{10}
%TCIDATA{OutputFilter=LATEX.DLL}
%TCIDATA{Version=5.50.0.2953}
%TCIDATA{<META NAME="SaveForMode" CONTENT="1">}
%TCIDATA{BibliographyScheme=BibTeX}
%TCIDATA{Created=Thursday, September 11, 2008 15:11:56}
%TCIDATA{LastRevised=Wednesday, September 30, 2009 20:31:59}
%TCIDATA{<META NAME="GraphicsSave" CONTENT="32">}
%TCIDATA{<META NAME="DocumentShell" CONTENT="Articles\SW\article.egan">}

\topmargin=0 in \headheight=0in \headsep=0in \topskip=0in \textheight=9in \oddsidemargin=0in \evensidemargin=0in \textwidth=6.5in
\input{tcilatex}
\begin{document}


New York University

Wilf Family Department of Politics

Fall 2009

\begin{center}
{\large \textbf{Quantitative Research in Political\ Science I}}

(G53.1150)

Professor Patrick Egan

\bigskip

\textbf{PROBLEM\ SET 1: Due Monday, October 5 at beginning of class.}
\end{center}

\textit{A reminder: you may work with others in the class on this problem
set, and you are in fact encouraged to do so. \ However, the work you hand
in must be your own. \ Handwritten work is acceptable, but word-processed
work (e.g., using LaTeX) is preferred.}

\bigskip

\begin{enumerate}
\item On the class Blackboard site may be found a link to the raw data from
the latest \textit{New York Times}/CBS News poll, which covers American
public opinion on the current health care debate in the United States as
well as other topics. \ In a \textit{very} brief, two-paragraph mini-essay,
discuss some of the poll's findings (whatever interests you). \ In your
discussion, you must correctly use each of the following 8 terms. \ For
James's sake, please \textbf{underline} the terms as you use them in your
mini-essay and \textbf{cite }the survey questions to which you refer by
number. \ Don't worry too much about the content or flow of your essay: I
just want to see you using these terms correctly.

\qquad \qquad \qquad \qquad \qquad \qquad 
\begin{tabular}{ll}
\textit{interval-level variable} & \textit{a distribution skewed to the right%
} \\ 
\textit{nominal-level variable} & \textit{mode} \\ 
\textit{mean} & \textit{dichotomous variable} \\ 
\textit{symmetric distribution} & \textit{median}%
\end{tabular}

\bigskip 

\textit{The following two questions are designed to get you nimble with the
sorts of proofs we'll be doing in class with scalar algebra and summation
signs:}\bigskip 

\item Prove that the sum of deviations of a set of measurements from their
mean is equal to zero, that is:%
\begin{equation*}
\underset{i=1}{\overset{N}{\sum }}\left( y_{i}-\overline{y}\right) =0.
\end{equation*}%
\bigskip 

\item Show that the variance of an empirical distribution is equal to the
average of the squared observations minus the square of the average
observation, that is:

\begin{equation*}
\frac{1}{N}\underset{i=1}{\overset{N}{\sum }}\left( y_{i}-\overline{y}%
\right) ^{2}=\frac{_{_{^{^{\underset{i=1}{\overset{N}{\sum }}\left(
y_{i}\right) ^{2}}}}}}{N}-\left( \overline{y}\right) ^{2}
\end{equation*}%
\bigskip 

\textit{Most of the remaining problems come from WMS, \textbf{7th edition}.
\ Note that the answers to odd-numbered, non-asterisked problems may be
found in the back of the book. \ (In contrast to those answers, please show
all your work.)}\bigskip 

\item WMS Exercises 2.43 and 2.44.

\item WMS Exercises 2.46 and 2.47.

\item WMS Exercise 2.50.

\item This is an extension of the example we did in class:

There were 16 students in Quant I last fall. \ Of these students, 11 are
male. \ I assign the students at random to three teams consisting of 6, 5,
and 5 students respectively. \ Figure out the probability of having all
single-sex teams using the sample-point method.
\end{enumerate}

\bigskip

\end{document}

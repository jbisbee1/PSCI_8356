
\documentclass[11pt]{article}
%%%%%%%%%%%%%%%%%%%%%%%%%%%%%%%%%%%%%%%%%%%%%%%%%%%%%%%%%%%%%%%%%%%%%%%%%%%%%%%%%%%%%%%%%%%%%%%%%%%%%%%%%%%%%%%%%%%%%%%%%%%%%%%%%%%%%%%%%%%%%%%%%%%%%%%%%%%%%%%%%%%%%%%%%%%%%%%%%%%%%%%%%%%%%%%%%%%%%%%%%%%%%%%%%%%%%%%%%%%%%%%%%%%%%%%%%%%%%%%%%%%%%%%%%%%%
\usepackage[abbr]{harvard}
\usepackage{amssymb}
\usepackage{setspace,graphics,epsfig,amsmath,rotating,amsfonts,mathpazo}

\setcounter{MaxMatrixCols}{10}
%TCIDATA{OutputFilter=LATEX.DLL}
%TCIDATA{Version=5.50.0.2953}
%TCIDATA{<META NAME="SaveForMode" CONTENT="1">}
%TCIDATA{BibliographyScheme=BibTeX}
%TCIDATA{Created=Thursday, September 11, 2008 15:11:56}
%TCIDATA{LastRevised=Friday, October 08, 2010 18:32:40}
%TCIDATA{<META NAME="GraphicsSave" CONTENT="32">}
%TCIDATA{<META NAME="DocumentShell" CONTENT="Articles\SW\article.egan">}

\topmargin=0 in \headheight=0in \headsep=0in \topskip=0in \textheight=9in \oddsidemargin=0in \evensidemargin=0in \textwidth=6.5in
\input{tcilatex}
\begin{document}


New York University

Wilf Family Department of Politics

Fall 2010

\begin{center}
{\large \textbf{Quantitative Research in Political\ Science I}}

(G53.1250)

Professor Patrick Egan

\bigskip

\textbf{PROBLEM\ SET 2: Due Wednesday, October 13 at beginning of class.}
\end{center}

\textit{A reminder: you may work with others in the class on this problem
set, and you are in fact encouraged to do so. \ However, the work you hand
in must be your own. \ Handwritten work is acceptable, but word-processed
work (e.g., using LaTeX) is preferred. \ As usual, be sure that you're
consulting the \textbf{7th edition of WMS }when doing the problems.}

\bigskip

\begin{enumerate}
\item This is an extension of the example we did in class. \ 

There are 9 students in Quant I this fall. \ Of these students, 4 are
female. \ I assign the students at random to two teams consisting of 5 and 4
students respectively. \ Figure out the probability that one team is all
women and that the other is all men.

\begin{enumerate}
\item First answer this question using the \textit{sample-point }method. \
Be sure to explain what you are doing in a few sentences.\ 

\item Now, answer it using the \textit{event-composition} method. \ Again,
explain what you are doing.

\item Which method do you prefer? \ Why?
\end{enumerate}

\item WMS Exercise 2.156.

\item WMS Exercise 3.22.

\item WMS Exercise 3.30.

\item WMS Exercise 3.33.

\item WMS Exercise 3.48.

\item WMS Exercise 3.54.

\item WMS Exercises 3.64 and 3.65.
\end{enumerate}

\end{document}


\documentclass[11pt]{article}
%%%%%%%%%%%%%%%%%%%%%%%%%%%%%%%%%%%%%%%%%%%%%%%%%%%%%%%%%%%%%%%%%%%%%%%%%%%%%%%%%%%%%%%%%%%%%%%%%%%%%%%%%%%%%%%%%%%%%%%%%%%%%%%%%%%%%%%%%%%%%%%%%%%%%%%%%%%%%%%%%%%%%%%%%%%%%%%%%%%%%%%%%%%%%%%%%%%%%%%%%%%%%%%%%%%%%%%%%%%%%%%%%%%%%%%%%%%%%%%%%%%%%%%%%%%%
\usepackage[abbr]{harvard}
\usepackage{amssymb}
\usepackage{setspace,graphics,epsfig,amsmath,rotating,amsfonts,mathpazo}

\setcounter{MaxMatrixCols}{10}
%TCIDATA{OutputFilter=LATEX.DLL}
%TCIDATA{Version=5.50.0.2960}
%TCIDATA{<META NAME="SaveForMode" CONTENT="1">}
%TCIDATA{BibliographyScheme=BibTeX}
%TCIDATA{Created=Thursday, September 11, 2008 15:11:56}
%TCIDATA{LastRevised=Wednesday, November 20, 2013 12:51:45}
%TCIDATA{<META NAME="GraphicsSave" CONTENT="32">}
%TCIDATA{<META NAME="DocumentShell" CONTENT="Articles\SW\article.egan">}

\topmargin=0 in \headheight=0in \headsep=0in \topskip=0in \textheight=9in \oddsidemargin=0in \evensidemargin=0in \textwidth=6.5in
\input{tcilatex}
\begin{document}


New York University

Wilf Family Department of Politics

Fall 2013

\begin{center}
{\large \textbf{Quantitative Research in Political\ Science I}}

Professor Patrick Egan

\bigskip
\end{center}

\bigskip

\begin{center}
\textbf{MIDTERM\ EXAMINATION:\ NOVEMBER 8, 2013, 10 a.m. - 1 p.m.}
\end{center}

This exam is open-book, open-note. \ You will need either statistical tables
like those appearing in the back of your text or access to a statistical
software program that can provide similar information.\ \ Consultation of
Web resources is not permitted.\bigskip

Be sure to show all your work and to use complete sentences to provide
explanations.\bigskip

Point totals for each question are as follows. \ They sum to 92 points. \ 
\textbf{It is recommended that you read all parts to a question--that is,
parts (a), (b), (c), etc.--before beginning to answer it.}

\begin{itemize}
\item Question 1: \ 20 points.

\item Question 2: \ 20 points.

\item Question 3: \ 24 points.

\item Question 4: \ 28 points.
\end{itemize}

\begin{center}
\newpage
\end{center}

\begin{enumerate}
\item[1.] Any attempt to assess voter turnout with survey data is made more
complicated by the challenge of social desirability bias: survey respondents
tend to over-report whether they voted. \ For example, 58.3 percent of
voting-age Americans told the Census Bureau that they voted in the 2004
presidential election. \ But voting statistics show that only 55.5 percent
did. \ (This gap is even greater in years without presidential elections.)

You are conducting a post-election survey with which you'd like to get the
best possible estimate of aggregate voter turnout. \ You have the choice
between two different survey modes. \ The first is a traditional telephone
poll using professional interviewers (\textquotedblleft
live\textquotedblright ). \ The second is a telephone poll using the
interactive voice response (\textquotedblleft IVR\textquotedblright )
technique, in which questions are asked by a recorded voice and participants
enter their responses via their telephone keypad. \ Previous research has
found that live and IVR survey modes have different advantages and
disadvantages:

\begin{itemize}
\item The \textit{advantage }of IVR is that--since it is more removed from
the social context of a conversation between two people--it is less subject
to overreporting of the vote than surveys with live interviewers. \ 

\item The \textit{disadvantage }of IVR is that it is subject to more error
than surveys with live interviewers.
\end{itemize}

Your goal is to estimate $p$, the proportion of the voting-age population
that turned out to vote. \ We can consider the proportion of respondents
reporting they voted in the live and IVR survey modes as two different
estimates of $p.$ \ Call these estimates $\widehat{p}_{L}=\frac{Y_{L}}{n_{L}}
$ and $\widehat{p}_{I}=\frac{Y_{I}}{n_{I}},$ respectively.

\begin{enumerate}
\item The advantages and disadvantages of the live and IVR modes correspond
with two properties of estimators we've discussed in class. \ In a few
sentences, describe these two properties, and say how we would expect $%
\widehat{p}_{L}$ and $\widehat{p}_{I}$ to compare regarding these two
properties.\medskip

Assume it is the case that:%
\begin{gather*}
E\left( \widehat{p}_{L}\right) =ap+b;\text{ \ }VAR\left( \widehat{p}%
_{L}\right) =\frac{a^{2}}{n}p\left( 1-p\right) \text{ \ and} \\
E\left( \widehat{p}_{I}\right) =cp+d;\text{ \ }VAR\left( \widehat{p}%
_{I}\right) =\frac{c^{2}}{n}p\left( 1-p\right) . \\
\text{Further, assume that }a<c,b>d,\text{ and }\frac{b-d}{c-a}>1.
\end{gather*}

\item What is $B\left( \widehat{p}_{L}\right) ?$ \ What is $B\left( \widehat{%
p}_{I}\right) ?$ \ Show that $B\left( \widehat{p}_{L}\right) >B\left( 
\widehat{p}_{I}\right) .$\medskip

\item Show (trivially) that $VAR\left( \widehat{p}_{I}\right) >VAR\left( 
\widehat{p}_{L}\right) .$\medskip

\item We have a situation where one of our potential estimators suffers from
greater bias, while the other is subject to more error. \ In class, we
learned of a criterion often used to measure the tradeoff between bias and
error. \ What is this criterion called and what is its formula?\medskip

\item Now consider a case where $a=.01,b=.1;c=.02,d=.05$ and\ $p=\frac{1}{2}.
$ \ According to the criterion you identified in part $(d)$, which is the
better estimator?\newpage 
\end{enumerate}

\item[2.] Consider two independent large-sample means, $\overline{X}$ and $%
\overline{Y},$ where $\overline{X}<\overline{Y}.$ \ As discussed in class,
it is sometimes the case that the confidence intervals constructed about the
means $\mu _{X}$ and $\mu _{Y}$ overlap, but the confidence interval
constructed about the difference in these means $\left( \mu _{Y}-\mu
_{X}\right) $ does not contain zero.\bigskip\ \ 

\begin{enumerate}
\item Write the inequality that must hold if the CIs for $\mu _{X}$ and $\mu
_{Y}$ overlap. \ The expression should be in terms of $\overline{X}$,$%
\overline{Y,}$ $\sigma _{\overline{X}},$ $\sigma _{\overline{Y}}$.and some $%
z_{\frac{\alpha }{2},}$which for simplicity you may refer to as $z.$\bigskip

\item Write the inequality that must hold if the CI for $\mu _{Y}-\mu _{X}$
does not contain zero. \ This should be in terms of $\overline{X}$,$%
\overline{Y,}\sigma _{\overline{Y}-\overline{X}}$, and $z.$(the same $z$ as
in part (a)). \ \bigskip

\item Now show that a sufficient condition for the CIs for $\mu _{X}$ and $%
\mu _{Y}$ to overlap while the CI for $\mu _{Y}-\mu _{X}$ does not contain
zero is%
\begin{equation*}
z\sigma _{\overline{Y}-\overline{X}}<\overline{Y}-\overline{X}<z\left(
\sigma _{\overline{X}}+\sigma _{\overline{Y}}\right) \text{ .}
\end{equation*}

\item To reassure yourself that the inequality in (c) can be satisfied, show
that it is always the case that 
\begin{equation*}
z\sigma _{\overline{Y}-\overline{X}}<z\left( \sigma _{\overline{X}}+\sigma _{%
\overline{Y}}\right) .
\end{equation*}%
\newpage
\end{enumerate}

\item[3.] Indicate whether each of the following statements is true or
false. Explain each of your answers, using mathematics where necessary.

\begin{enumerate}
\item Our estimates of population means using large samples rely heavily on
assumptions about the shape of the distribution of the population.

\item Our estimates of population means using large samples rely heavily on
assumptions about the process giving rise to the sample.

\item All things being equal, Type I errors are more likely with small
samples than with large samples.

\item All things being equal, Type II errors are more likely with large
samples than with small samples.

\item 
\begin{equation*}
\frac{\partial \sigma _{\overline{Y}}^{2}}{\partial \sigma _{Y}^{2}}<0.
\end{equation*}

\item In an i.i.d. random sample of size $n$ drawn from the population $Y$,
the observation $Y_{1}$ is an unbiased estimate of $\mu _{Y}.$

\item All things being equal, I am more confident in a finding confirming
the null $H_{0}:\mu _{1}-\mu _{2}=0$ as $\alpha $ gets lower.

\item The Bernoulli, Binomial, and Poisson distributions bear important
similaries to one another.\newpage
\end{enumerate}

\item[4.] Answer each of the following questions. \ Show your work. \ 

\begin{enumerate}
\item You wish to determine whether women are more Democratic than men. \ 

\begin{enumerate}
\item You draw a random sample of 50 men and 50 women, and find that $43$
percent of men identify as Democrats, while $56$ percent of women do. \ How
sure does this make you that women are more Democratic than men in the
general population?

\item Now you draw a random sample of $n$ men and $n$ women, and find that $%
43$ percent of men identify as Democrats, while $56$ percent of women do. \
Find the largest $n$ at which you are \textbf{not} sure with 95 percent
confidence that women are more Democratic than men in the general
population. \ \bigskip
\end{enumerate}

\item You are the leader of an environmental group who hopes that major
environmental legislation is passed by Congress and signed into law at some
point during the 2013-2014 Congressional session. \ The number of major
environmental laws passed per session can be modeled as the random variable $%
Y$ with $E\left( Y\right) =1.$ \ What is the chance that at least one piece
of major environmental legislation gets passed during the session? \ Be
precise about how you are modeling this process. \bigskip

\item You have discovered some observation $Y_{i}$ of the random variable $Y$
that is two standard deviations greater than $Y$'s mean, $\mu _{Y}.$ \ 

\begin{enumerate}
\item Find \thinspace $P\left( Y\geq \mu _{Y}+2\sigma _{Y}|Y\symbol{126}%
\text{Standard Normal}\right) .$

\item Explain why it is non-sensical to model $Y$ as distributed Uniform. \
HINT: Show that \thinspace $P\left( Y\geq \mu _{Y}+2\sigma _{Y}|Y\symbol{126}%
\text{Uniform}\right) =0.$
\end{enumerate}
\end{enumerate}
\end{enumerate}

\end{document}
